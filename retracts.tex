\section{Disc retracts}\label{Metric-minimizing discs}

A disc retract as defined above is nothing but the image of strong deformation retraction of $\DD$;
the restriction of retraction to the boundary can be taken as the corresponding boundary curve.
We will not need this statement, but it follows from Moore's quotient theorem quoted below. 

\begin{thm}{Proposition}\label{prop:|D|}
Let $Y$ be a metric space and $s\:\DD\to\ Y$ be a metric-minimizing disc.
Then $|\DD|_s$ is a disc retract with boundary curve $\delta_s=\bar\pi_s|_{\SS^1}$.
Moreover, the map $\tau_s\:\<\DD\>_s\to |\DD|_s$ is injective and defines an isometry
$\<\DD\>_s\zz\to \<|\DD|\>_s$;
that is, $\tau_s$ is an isometry from $\<\DD\>_s$ to $|\DD|_s$ equipped with induced length metric.
\end{thm}

We need a little preparation before giving the proof.

Let $Y$ be a metric space and
$s\:\DD\to Y$ be a continuous map.
We say that $s$ has \label{page:no-bubble}\emph{no bubbles}
if for any point $p\in Y$ every connected component of the complement $\DD\backslash s^{-1}\{p\}$ contains a point from $\partial \DD$.

\begin{thm}{Lemma}\label{prop:point-complement}
Let $Y$ be a metric space and $s\:\DD\to Y$ be a metric-minimizing disc.
Then $s$ has no bubbles.
\end{thm}

\parit{Proof.}
Assume the contrary;
that is, there is $y\in Y$ such that the complement $\DD\backslash s^{-1}(y)$ contains a connected component $\Omega$ with $\partial \DD\cap \Omega=\emptyset$.

Let us define a new map $s'\:\DD\to\ Y$ by setting $s'(z)=y$ for any $x\in \Omega$ and $s'(x)=s(x)$ for any $x\notin \Omega$.

By construction, $s'$ and $s$ agree on $\partial\DD$. Moreover, $s\succcurlyeq s'$
because of the majorization $\mu\:\hat\pi_s(x)\zz\mapsto \hat\pi_{s'}(x)$.

Note that
\[\<x-x'\>_{s}>0=\<x-x'\>_{s'}\]
for a pair of distinct points $x,x'\in \Omega$.
In particular, $\mu$ is not an isometry, a contradiction.
\qeds



\begin{thm}{Lemma}\label{prop:disc-moore}
Let $Y$ be a metric space and assume that a map $f\:\DD\to Y$ has no bubbles.
Then $|\DD|_f$ is homeomorphic to a disc retract with boundary curve $\bar\pi_f|_{\SS^1}$.
\end{thm}

This lemma is nearly identical to \cite[Corollary 7.12]{LW3}; it could be considered a disc version of Moore's quotient theorem \cite{moore}, \cite{daverman}
which states that if a continuous map $f$ from the sphere $\SS^2$ to a Hausdorff space $X$
has acyclic fibers, then $f$ can be approximated by a homeomorphism;
in particular $X$ is homeomorphic to $\SS^2$.

\parit{Proof.}
From Lemma~\ref{lem:picont} we know that $\bar\pi_f$ is continuous and hence $|\DD|_f$
is a compact metric space. 


The mapping cone over $\DD$ along its boundary is homeomorphic to the sphere $\SS^2$;
denote by $\Sigma$ the mapping cone over $|\DD|_f$ with respect to $\bar\pi_s|_{\SS^1}$.
Let us extend the map $\bar\pi_f$ to a map between the mapping cones $\SS^2\to\Sigma$.
Note that this map satisfies Moore's quotient theorem, hence the statement follows.
\qeds

\parit{Proof of Proposition~\ref{prop:|D|}.}
The first two statements follow from Lemma~\ref{prop:point-complement} and Lemma~\ref{prop:disc-moore}.

Since $|\DD|_s$ is a disc retract, the mapping cylinder over the boundary curve of $|\DD|_s$ is homeomorphic to $\DD$.
Denote by $r \:\DD\to |\DD|_s$ the corresponding retraction.

Note that $\<\DD\>_{\bar s\circ r}$ is isometric to $|\DD|_s$ equipped with the induced length metric.
Recall that the map $\tau_s\:\<\DD\>_s\to|\DD|_s$ is short and the induced map $\mu\:\<\DD\>_s\zz\to\<\DD\>_{\bar s\circ r}$ is a majorization.
Since $s$ is metric-minimizing, $\mu$ is an isometry.
Hence the statement follows.
\qeds

Assume $W$ is a disc retract with a boundary curve $\delta$.
Recall that a point $p$ in a connected space $W$ is a \emph{cut point} if the complement $W\backslash\{p\}$ is not connected.

\begin{thm}{Lemma}\label{lem:discs}
Suppose that $W$ is a disc retract.
Let $\Delta\subset W$ be a maximal connected subset that contains no cut points.
Assume $\Delta$ has at least two points.
Then closure of $\Delta$ is homeomorphic to $\DD$.
\end{thm}

\parit{Proof.}
Let $\delta$ be a boundary curve of $W$, so the mapping cylinder $W_\delta$ is homeomorphic to $\DD$.

Note that $p$ is a cut point of $W$ if and only if $\delta^{-1}\{p\}$ has at least two connected components.
In particular, any cut point of $W$ lies on its boundary curve.

Denote by $\bar\Delta$ the closure of $\Delta$.
Note that for any $x\notin\bar\Delta$ there is a cut point $p\in\bar\Delta$ that cuts $x$ from $\Delta$.
Moreover, the map $\sigma\:x\mapsto p$ is uniquely defined on $W\backslash\bar\Delta$;
extend this map to whole $W$ by identity in $\bar\Delta$.
By Moore's theorem, $\bar\Delta$ is a disc retract with a boundary curve $\sigma\circ\delta$.
Namely we apply Moore's theorem to $\SS^2= W_\delta/(\SS^1\times 1)$ and the quotient map $\SS^2\to \SS^2/\sim$ for the minimal equivalence relation such that $x\sim y$ if $\sigma(x)=\sigma(y)$.

The space $\bar\Delta$ has no cut points, in other words $\sigma\circ\delta$ is monotonic.
It follows that $\sigma\circ\delta$ can be reparameterized into a simple closed curve.
By Jordan--Schoenflies theorem, the statement follows.
\qeds

The following lemma will be used in the final step in the proof of the main theorem, Section~\ref{Main theorem}.

\begin{thm}{Lemma}\label{lem:maj is isom}
Let $Y$ be a metric space and $s\:\DD\to Y$ be a metric-minimizing map.
Assume that there is a $\CAT(0)$ disc retract $W$ with boundary curve $\delta$ and a short map $f\:\<\DD\>_s \to W$
such that $f\circ \delta_s=\delta$. If there exists a short map 
$q\: W\to Y$ with $q\circ \delta=s|_{\partial \DD}$, then the map $f$ is an isometry.
\end{thm}

\parit{Proof.}
Let $r\:\DD\to W$ be the projection from the mapping cylinder $\DD=W_\delta$. 
Note that $r$ is a retraction, $r|_{\partial \DD}=\delta$ and the composition $\DD\xrightarrow{r}W\xrightarrow{q} Y$ fulfills \[q\circ r|_{\partial \DD}=s|_{\partial \DD}.\]

Note that  $\<W\>_q=\<\DD\>_{q\circ r}$ and the natural projection $\rho\: W\to \<W\>_q$ is short.
It follows that $\rho\circ f\: \<\DD\>_s\to \<\DD\>_{q\circ r}$ is a majorization.
Since $s$ is metric-minimizing, $\rho\circ f$ is an isometry. 

Therefore $f$ is an isometric embedding that contains $\delta$
in its image. 
By Lemma \ref{lem:geospace} and Proposition \ref{prop:|D|}, $\<\DD\>_s$ is a complete geodesic space.
So $f$ has to be surjective and therefore an isometry.
\qeds
