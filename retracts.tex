\section{Disc retracts}\label{Metric minimizing discs}

\begin{thm}{Proposition}\label{prop:|D|}
Let $s\:\DD\to\ Y$ be a metric minimizing disc.
Then $|\DD|_s$ is a disc retract with boundary curve $\delta_s=\bar\pi_s|_{\SS^1}$.
Moreover the map $\tau_s\:\<\DD\>_s\to |\DD|_s$ is injective and defines an isometry
$\<\DD\>_s\to \<|\DD|_s\>$;
that is $\tau_s$ is an isometry from $\<\DD\>_s$ to $|\DD|_s$ equipped with induced intrinsic metric.
\end{thm}

We need a little preparation before giving the proof.

Let $Y$ be a metric space and
$s\:\DD\to Y$ be a continuous map.
We say that $s$ has \label{page:no-bubble}\emph{no bubbles}
if for any point $p\in Y$ every connected component of the complement $\DD\backslash s^{-1}\{p\}$ contains a point from $\partial \DD$.

\begin{thm}{Lemma}\label{prop:point-complement}
Let $Y$ be a metric space and $s\:\DD\to Y$ be a metric minimizing disc.
Then $s$ has no bubbles.
\end{thm}

\parit{Proof.}
Assume the contrary;
that is, there is $y\in Y$ such that the complement $\DD\backslash s^{-1}(y)$ contains a connected component $\Omega$ 
such that $\partial \DD\cap \Omega=\emptyset$.

Let us define a new map $s'\:\DD\to\ Y$ by setting $s'(z)=y$ for any $x\in \Omega$ and $s'(x)=s(x)$ for any $x\notin \Omega$.

By construction, $s'$ and $s$ agree on $\partial\DD$. Moreover, $s\succcurlyeq s'$
because of the majorization $\mu\:\hat\pi_s(x)\zz\mapsto \hat\pi_{s'}(x)$.

Note that
\[\<x-x'\>_{s}>0=\<x-x'\>_{s'}\]
for a pair of distinct points $x,x'\in \Omega$.
In particular, $\mu$ is not an isometry. A contradiction.
\qeds



\begin{thm}{Lemma}\label{prop:disc-moore}
Let $Y$ be a metric space and assume that a map $f\:\DD\to Y$ has no bubbles.
Then $|\DD|_f$ is homeomorphic to a disc retract with boundary curve $\bar\pi_f|_{\SS^1}$.
\end{thm}

This lemma is nearly identical to \cite[Corollary 7.12]{LW3}; it could be considered a a disc version of Moore's quotient theorem \cite{moore}, \cite{daverman}
which states that if a continuous map $f$ from the sphere $\SS^2$ to a Hausdorff space $X$
has acyclic fibers, then $f$ can be approximated by a homemorphism; in particular $X$ is homeomorphic to $\SS^2$.

\parit{Proof.}
Form Lemma~\ref{lem:picont} we know that $\bar\pi_f$ is continuous and hence $|\DD|_f$
is a compact metric space. 


The mapping cone over $\DD$ along it boundary is homeomorphic to the sphere $\SS^2$;
denote by $\Sigma$ the mapping cone over $|\DD|_f$ with respect to $\bar\pi_s|_{\SS^1}$.
Let us extend the map $\bar\pi_f$ to a map between the mapping cones $\SS^2\to\Sigma$.
Note that this map satisfies Moore's quotient theorem, hence the statement follows.
\qeds

\parit{Proof of Proposition~\ref{prop:|D|}.}
The first two statements follow from Lemma~\ref{prop:point-complement} and Lemma~\ref{prop:disc-moore}.

Since $|\DD|_s$ is a disc retract, the mapping cylinder over the boundary curve of $|\DD|_s$ is homeomorphic to $\DD$.
Denote by $r \:\DD\to |\DD|_s$ the corresponding retraction.

Note that $\<\DD\>_{\bar s\circ r}$ is isometric to $|\DD|_s$ equipped with the induced intrinsic metric.
Recall that the map $\tau_s\:\<\DD\>_s\to|\DD|_s$ is short and the induced map $\mu\:\<\DD\>_s\to\<\DD\>_{\bar s\circ r}$ is a majorization.
Since $s$ is metric minimizing, $\mu$ is an isometry.
Hence the statement follows.
\qeds
