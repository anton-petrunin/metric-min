\section{About the old theorem}\label{sec:old-thm}

In this section we will show that the old theorem from \cite{petrunin-metric-min} follows from the new one.

Let us restate the main theorem in \cite{petrunin-metric-min} with a correction;
originally it assumed only compactness of $\<D\>_s$,
the proof actually used that the metric $(x,y)\zz\mapsto\<x-y\>_{s}$ is continuous,
which is a stronger condition.

\begin{thm}{Old theorem}
Let $Y$ be a $\CAT(0)$ space and $s\:\DD\to Y$ a continuous map that satisfies the following property: 
if a continuous map $s'\:\DD\to Y$ agrees with $s$ on $\partial\DD$ and
\[\<x-y\>_{s'}\le \<x-y\>_{s}\]
for any $x,y\in \DD$
then the equality holds for all pairs of $x$ and $y$.
Assume that the function $(x,y)\mapsto\<x-y\>_{s}$ is continuous.
Then $\<D\>_s$ is $\CAT(0)$.
\end{thm}

Note that the assumption in the old theorem is weaker than metric minimizing condition as defined in this paper;
that is, a metric minimizing map have to satisfy this assumption, but the other way around may not hold --- a majorization may not lift to $\DD$.

\begin{thm}{Lemma}
Let $Y$ be a metric space and $s\:\DD\to Y$ a map such that the function $(x,y)\mapsto\<x-y\>_{s}$ is continuous.
Assume that the fiber $s^{-1}\{z\}$ of some point $z\in Y$ contains a connected set $P$.
Then $\<x-y\>_{s}=0$ for any pair $x,y\in P$.
\end{thm}


The statement might is evident if $P$ is path connected; 
but there are examples (pseudoarc) of nontrivial connected compact sets that contains no arcs.
We present an argument of Taras Banakh \cite{banakh}.

\parit{Proof.}
It is sufficient to construct a path $\alpha$ connecting $x$ to $y$ of arbitrary small length.

The curve $\alpha$ will be constructed as a limit of an iteration procedure;
we define it on a collection of subintervals of $[0,1]$.
The restriction of $\alpha$ to each subinterval runs from $P$ to $P$;
the closure of the ends of the intervals forms a Cantor $C$ set in $[0,1]$; 
 $C$ interiors form the complement of a Cantor set.

Fix a converging to zero sequence of positive numbers $(\eps_n)$. 
Since $(x,y)\mapsto\<x-y\>_{s}$ is continuous, 
there are rectifiable curves $\gamma_1,\dots,\gamma_n$ that cut $\<\DD\>_s$ into closed sets $K_1,\dots,K_\ell$ of diameter at most $\eps_1$.
Indeed, ...

Replacing $P$ by a smaller set, we can assume that $P$ is irreducible; that is, it does not contain proper connected closed sets containing $x$ and $y$.

Set $\ell_i=\length(\gamma_i\cap P)$.
Note that for each $i$ the intersection $\gamma_i\cap P$ can be covered by a finite number of arcs with total length arbitrary close to $\ell_i$.
Denote by $\delta_1,\dots,\delta_k$ all these arcs of all $i$.

Note that one can chose a sequence of points $x=x_0,y_1,x_1,\dots,y_n=y$
such that 
(1) for any $i\in\{1,\dots,n-1\}$ both points $x_i$ and $y_i$ lie on one arc $\delta_j$ for some $j$ 
and 
(2) for any $i\in\{1,\dots,n\}$ the points $x_{i-1}$ and $y_i$ lie in one connected component $Q_k$ of $K_j\cap P$ for some $j$.

Let  $Q_1, \dots, Q_n$ be the required components; 
set $P'=\bigcup_k Q_k$. 
Include the subarcs from $x_i$ to $y_i$ in the curve $\alpha$ and repeat the procedure for each $Q_k$ and the diameter bound $\eps_2$.

This way we get a decresing sequence of sets $P\supset P'\supset P''$,
diameter of each connected component of 


Note that one can choose a finite collection $P_1,\dots,P_m$ of connected components of $K_i\cap P$ such that $P_1\ni x$, $P_m\ni y$
and the set $P_1\cup\dots\cup P_m\cup\delta_1\cup\dots\cup\delta_k$ contaians a connected set $P'\ni x,y$.

Pass to an irreducible subset $P'$; it is formed by a collection of subarcs of $\delta_i$ and connected subsets of $K_i\cap P$.

Now choose apply the same procedure for each $P_i$ for a smaller ...
and iterate.

At the end of the day we get a 

Pass to a 

Note that there is a chain of connected components $P_1,\dots,P_m$, of $K_i\cap P$ for different $i$ such that the union $P_1\cup\dots\cup P_m\cup\delta_1\cup\dots\cup\delta_k$ 

one can find finite collection of subarcs of $\gamma_i$ such that

*Proof.* Replacing $P$ by a smaller subcontinuum, we can assume that $P$ is irreducible between points $x$ and $y$, which means that any subcontinuum of $P$ that contains $x$ and $y$ coincides with $P$. 

The irreducibility of $P$ implies that for any rational numbers $a<b$ the (closed) set $\{x\in \mathbb R:\{x\}\times[a,b]\subset P\}$ is nowhere dense in $\mathbb R$. Consequently the set $$G_1:=\{x\in\mathbb R:P\cap(\{x\}\times\mathbb R)\mbox{ is zero-dimensional}\}$$is dense $G_\delta$ in $\mathbb R$. Choose an increasing sequence $a_0,\dots,a_n\in G_1$ such that $P\subset [a_0,a_n]\times\mathbb R$ and $|a_i-a_{i-1}|<\frac12\varepsilon$ for all positive $i\le n$. For every $i\in\{0,\dots,n\}$ the inclusion $a_i\in G_1$ implies that the set $Z_i:=\{y\in \mathbb R:(a_i,y)\in P\}$ is zero-dimensional and hence nowhere dense in $\mathbb R$.

By analogy, the set $$G_2:=\{y\in\mathbb R:P\cap(\mathbb R\times \{y\})\mbox{ is zero-dimensional}\}$$is dense $G_\delta$ in $\mathbb R$. Then we can choose a monotone sequence $b_0,\dots,b_m\in\mathbb G_2\setminus\bigcup_{i=0}^nZ_i$ such that $P\subset \mathbb R\times [b_0,b_m]$ and $|b_i-b_{i-1}|<\frac12\varepsilon$ for all positive $i\le m$.

For every $(i,j)\in\{1,\dots,n\}\times\{1,\dots,m\}$ consider the cube $Q_{i,j}:=[a_{i-1},a_i]\times[b_{j-1},b_j]$ and its boundary $\partial Q_{i,j}$ in the plane.
The choice of the points $a_{i-1}$, $a_i$, $b_{j-1}$, $b_j$ guarantees that the intersection $P\cap\partial Q_{i,j}$ is zero-dimensional and does not contain the vertices of the cube $Q_{i,j}$. 

Let $V:=\bigcup_{i=0}^n\{a_i\}\times [b_0,b_m]$ and $H:=\bigcup_{j=0}^m[a_0,a_n]\times\{b_j\}$ be the unions of vertical and horizontal sides of the cubes $Q_{i,j}$. Let $\mathcal C$ be the family of connected components of the symmteric difference $\Xi:=(V\cup H)\setminus (V\cap H)$. Each component $C\in\mathcal C$ is a vertical or horizontal interval of length $<\frac12\varepsilon$. By the regularity of the Lebesque measure on the real line, there is a disjoint finite family $\mathcal I$ of compact connected subsets of $\Xi\cup\{x,y\}$ such that

$\bullet$ $\{x\},\{y\}\in\mathcal I$;

$\bullet$ $P\cap\Xi\subset \bigcup\mathcal I$;

$\bullet$ $\lambda(\bigcup\mathcal I\setminus P)<\varepsilon$;

$\bullet$ each set $I\in\mathcal I$ meets the set $P$.

Now consider the finite graph $\Gamma$ whose set of vertices coincides with $\mathcal I$ and two distinct intervals $I,J\in\mathcal J$ form an edge of the graph $\Gamma$ if there exists a square $Q_{i,j}$ and a connected component of $P\cap Q_{i,j}$ that intersects both arcs $I$ and $J$. It can be shown that the graph $\Gamma$ is connected, so we can find a sequence $\{x\}=A_0,A_1,\dots,A_{k+1}=\{y\}$ of pairwise distinct vertices of the graph $\Gamma$ such that for every $i\le k$ the unordered pair $\{A_i,A_{i+1}\}$ is an edge of $\Gamma$. The latter means that $A_{i-1}$ and $A_i$ are intersected by some subcontinuum $P_i$ of $P$ that is contained in some square $Q_{j,l}$ and hence has diameter $<\varepsilon$.

It is clear that the sequences $P_1,\dots,P_k$ and $A_0,\dots,A_{k+1}$ have the required properties.
\qeds

\parit{Proof.}
It is sufficient to show that $s$ is metric minimizing.
\qeds
