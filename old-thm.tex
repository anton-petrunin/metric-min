\section{About the old theorem}\label{sec:old-thm}

In this section we formulate and prove two versions of the main theorem in \cite{petrunin-metric-min} with corrections.
The meaning of term \emph{metric minimizing} in this paper differ from the old paper.
Instead of using the older term we write a definition every time it is needed;
both conditions are weaker, so the statements require additional assumptions.

In Section~\ref{sec:defs}, we introduced two ways to pull back metric along a map $f\:X\to Y$ from a metric space $X$ to a length metric space $Y$:
$\langle x-y\rangle_f$ and $\langle| x-y|\rangle_f$.
Directly from the definition it follows that  $\langle x-y\rangle_f\ge \langle| x-y|\rangle_f$ for any $x,y\in X$.
The following diagram is extention of the diagram in Section~\ref{sec:defs} that includes $\langle|X|\rangle_f$, from above we have that both maps $\hat\tau_f$ and $\bar\tau_f$ are short and $\tau_f=\bar\tau_f\circ\hat\tau_f$.

\begin{figure}
\centering
\begin{tikzpicture}[scale=1.5]

\node (0) at (1,.5) {$\langle|X|\rangle_f$};
  \node (1) at (1,-.5) {$|X|_f$};
  \node (2) at (1,1.5){$\<X\>_f$};
  \node (11) at (3,1){$Y$};
  \node (12) at (-1,1) {$X$};
\draw[
    >=latex,
%   every node/.style={above,midway},% either
    auto=right,                      % or
    loop above/.style={out=75,in=105,loop},
    every loop,
    ]
   %(2) edge node{$\tau_f$}(1)
   (2) edge node{$\hat\tau_f$}(0)
   (0) edge node{$\bar\tau_f$}(1)
   (12) edge[bend right=10] node[swap]{$\hat{\bar \pi}_f$}(0)
   (0) edge[bend right=10] node[swap]{$\hat{\bar f}$}(11)
   (12) edge[bend left=90] node[swap]{$f$}(11)
   (12) edge[bend right=-10] node[swap]{$\hat\pi_f$}(2)
   (2) edge[bend right=-10] node[swap]{$ \hat f$}(11)
   (12) edge[bend right] node{$\bar \pi_f$}(1)
   (1) edge[bend right] node{$\bar f$}(11);
\end{tikzpicture}
\end{figure}

Our first formulation use $\langle|{*}-{*}|\rangle$-metric instead of $\langle{*}-{*}\rangle$; besides that it is identical to the old formulation.

\begin{thm}{Old theorem for $\bm{\langle|{*}-{*}|\rangle}$}\label{thm:old1}
Let $Y$ be a $\CAT(0)$ space and $s\:\DD\to Y$ a continuous map that satisfies the following property: 
if a continuous map $s'\:\DD\to Y$ agrees with $s$ on $\partial\DD$ and
\[\<|x-y|\>_{s'}\le \<|x-y|\>_{s}\]
for any $x,y\in \DD$
then the equality holds for all pairs of $x$ and $y$.
Assume that the space $\<|\DD|\>_s$ is compact.
Then $\<|\DD|\>_s$ is $\CAT(0)$.
\end{thm}

\parit{Proof.}
Note that $s$ has no bubbles, this can be proved the same way as Lemma~\ref{prop:point-complement}.
By Lemma~\ref{prop:disc-moore},  $|\DD|_s$ is a disc retract.

Note that the natural map $\bar\tau_s\:\<|\DD|\>_s\to |\DD|_s$ is injective and continuous.
Since $\<|\DD|\>_s$ is compact and $|\DD|_s$ is Hausdorff, the map $\bar\tau_s$ is a homeomorphism.

In particular $\<|\DD|\>_s$ is a disc retract as well as $|\DD|_s$.
Therefore the mapping cylinder of the boundary curve in $|\DD|_s$ is homeomorphic to $\DD$;
identify $\DD$ with the mapping cylinder and let $h\:\DD\to|\DD|_s$ be the projection that maps the cylinder to the boundary curve and does not move the rest of points.
(If $|\DD|_s$ is a disc, we can assume instead that $h$ is a homeomorphism.)
Note that $\<\DD\>_{\bar s\circ h}$ is isometric to $\<|\DD|\>_s$.

If $s$ is not metric minimizing,
then there is an other map $h'\:\DD\to Y$ such that $\bar s\circ h\succcurlyeq h'$ with nonisometric majorization $\mu\:\<|\DD|\>_s\to \<\DD\>_{h'}$.
For the composition $s'=\hat h\circ \mu\circ  \hat{\bar \pi}_s$, we have that 
\[\<x-y\>_{s'}\le \<|x-y|\>_{s}\quad\text{and therefore}\quad \<|x-y|\>_{s'}\le \<|x-y|\>_{s}\]
for any $x,y\in\DD$.

By the assumption, the last inequality is an equality.
Since $\<|\DD|\>_s$ is compact, we can assume that $s'$ is metric minimizing. %??? more here???
It remains to apply the main theorem.
\qeds

The second formulation use the metric $\langle{*}-{*}\rangle$ --- this is as in the original formulation,
and it requires that this pseudometric is continuous --- this is stronger than  
compactness of $\<\DD\>_s$ in the original formulation.
The sketch of proof given in \cite{petrunin-metric-min} implicitly used that the metric $(x,y)\zz\mapsto\<x-y\>_{s}$ is continuous.

\begin{thm}{Old theorem for $\bm{\langle{*}-{*}\rangle}$}\label{thm:old2}
Let $Y$ be a $\CAT(0)$ space and $s\:\DD\to Y$ a continuous map that satisfies the following property: 
if a continuous map $s'\:\DD\to Y$ agrees with $s$ on $\partial\DD$ and
\[\<x-y\>_{s'}\le \<x-y\>_{s}\]
for any $x,y\in \DD$
then the equality holds for all pairs of $x$ and $y$.
Assume that the function $(x,y)\mapsto\<x-y\>_{s}$ is continuous.
Then $\<\DD\>_s$ is $\CAT(0)$.
\end{thm}

This version follows from the following proposition and the old theorem for $\langle|{*}-{*}|\rangle$.
 
\begin{thm}{Proposition}\label{prop:<||>=<>}
Let $Y$ be a metric space and $s\:\DD\to Y$ is a continuous map without bubbles.
Assume that the function $(x,y)\mapsto \<x-y\>_s$ is continuous.
Then 
\[\<x-y\>_s=\<|x-|y\>_s\]
for any $x,y\in \DD$.
\end{thm}

Before going into the proof, let us give an example which shows that the statement is not trivial.
Consider a pseudo-arc $P\subset \DD$ and let $s$ be the quotient map $\DD\to \DD/P$.
Evidently 
\[\<|x-y|\>_s=0\]
for any $x,y\in P$.
However since $P$ contains no curves it is not at all evident that 
\[\<x-y\>_s=0\]
for any $x,y\in P$.
We present an argument of Taras Banakh \cite{banakh};
we do not know a generalization of the proposition to higher dimensions.

\begin{thm}{Lemma}\label{lem:subdivision}
Let $Y$ be a metric space and $s\:\DD\to Y$ is a continuous map without bubbles.
Assume that the function $(x,y)\mapsto \<x-y\>_s$ is continuous.
Then there is a subdivision of $\<\DD\>_s$ into subsets with arbitrary small diameter and rectifiable boundary.
\end{thm}

\parit{Proof.}
Note that given $\eps>0$ there is $\delta>0$ such that a set of diameter $\delta$ in $\<\DD\>_s$ can not separate a set of diameter at least $\eps$ from the boundary curve.
Indeed, otherwise passing to the limit we would get a one point set that separates a set from the boundary curve; that is, $s$ has a bubble --- a contradiction. 

Let us subdivide $\DD$ into small pieces by curves, say by vertical and horizontal lines.
Since the function $(x,y)\mapsto \<x-y\>_s$ is continuous,
we can assume that all pieces have small $\<{*}-{*}\>_s$-diameter;
that is, given $\eps>0$ we can assume that $\<x-y\>_s<\eps$ for any two points $x$ and $y$ in one piece.

It remains to modify the decomposition to make the boundary curves $\<{*}-{*}\>_s$-rectifiable.
Subdivide the curves into arcs with diameter smaller $\delta$ and exchange this piece by a curve of length smaller than $\delta$.
This way we might add to a piece a subset of diameter at most $\eps$.
Therefore the total $\<{*}-{*}\>_s$-diameter of each piece remains below $3\cdot\eps$.
The curves might cut more pieces from $\DD$, but by the same argument each of these pieces will have $\<{*}-{*}\>_s$-diameter below $3\cdot\eps$.
\qeds

\parit{Proof of \ref{prop:<||>=<>}.}
Fix point $x,y\in \DD$.
It is sufficient to construct a path $\alpha$ from $x$ to $y$ in $\DD$ such that the length of $s\circ\alpha$ is arbitrary close to $\<|x-y|\>_s$.

Since $(x,y)\mapsto \<x-y\>_s$ is continuous, $\<\DD\>_s$ and therefore $\<|\DD|\>_s$ are compact.
In particular, there is a minimzing geodesic $\gamma$ from $\hat{\bar x}$ to $\hat{\bar y}$ in $\<|\DD|\>_s$.
Denote by $\Gamma$ the inverse image of $\gamma$ in $\DD$;
this is a connected compact set which does nto have to be path connected.

To construct the needed path $\alpha$, it is sufficient to prove the following:

\begin{itemize}
 \item Given $\eps>0$ there is a finite collection of closed connected set $\Gamma_0,\dots,\Gamma_n$ in $\Gamma$ diameter ???
 and a collection of paths $\alpha_1,\dots,\alpha_n$ such that (1) the total length of $s\circ\alpha_i$ is at most $\eps$, (2) diameter of each $\Gamma_i$ is at most $\eps$ and (3) $x\in\Gamma_0$, $y\in\Gamma_1$ and the set formed by union of $\Gamma_i$ and $\alpha_i$ is connected.
\end{itemize}

Indeed, once the claim is proved, one can iterate it for a sequence of $\eps_n$ that converge to zero very fast and this way we obtain a curve made from the arcs of total length at most $\eps_1+\eps_2+\dots$ and a Cantor set in $\Gamma$; the Cantor set made by intersection of the nested closed sets in $\Gamma$.
The Cantor set can contribute at most $\<|x-y|\>_s$ to the length of curve; hence the result.

Choose a sudivision of $\DD$ provided by Lemma~\ref{lem:subdivision}....
\qeds



