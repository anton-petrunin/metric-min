\section{About the old theorem}\label{sec:old-thm}

In this section we formulate and prove two versions of the main theorem in \cite{petrunin-metric-min} with corrections.

Note that the meaning of term \emph{metric minimizing} in the present paper differs from its meaning in \cite{petrunin-metric-min} --- the old paper used a weaker definition and the theorem requires an additional assumption.

In Section~\ref{sec:defs}, we introduced three ways to pull back the metric along a map $f\:X\to Y$ from a topological space $X$ to a metric space $Y$: 
the induced length metric
$\langle x-y\rangle_f$,
the induced intrinsic metric $\langle| x-y|\rangle_f$
and the induced connecting metric $|x-y|_f$.
From the definitions we have that 
\[\langle x-y\rangle_f\ge \langle| x-y|\rangle_f\ge |x-y|_f\]
for any $x,y\in X$.
The second inequality is strict for generic maps, 
an example of a map $f$ with strict first inequality is given in \cite[4.2]{petrunin-intrinisic}.
(If $f$ is an embedding then equality holds \cite[4.5]{ledonne}.)
\begin{figure}[h]
\centering
\begin{tikzpicture}[scale=1.5]

\node (0) at (1,.5) {$\langle|X|\rangle_f$};
  \node (1) at (1,-.5) {$|X|_f$};
  \node (2) at (1,1.5){$\<X\>_f$};
  \node (11) at (3,1){$Y$};
  \node (12) at (-1,1) {$X$};
\draw[
    >=latex,
%   every node/.style={above,midway},% either
    auto=right,                      % or
    loop above/.style={out=75,in=105,loop},
    every loop,
    ]
   %(2) edge node{$\tau_f$}(1)
   (2) edge node{$\hat\tau_f$}(0)
   (0) edge node{$\bar\tau_f$}(1)
   (12) edge[bend right=10] node[swap]{$\hat{\bar \pi}_f$}(0)
   (0) edge[bend right=10] node[swap]{$\hat{\bar f}$}(11)
   (12) edge[bend left=90] node[swap]{$f$}(11)
   (12) edge[bend right=-10] node[swap]{$\hat\pi_f$}(2)
   (2) edge[bend right=-10] node[swap]{$ \hat f$}(11)
   (12) edge[bend right] node{$\bar \pi_f$}(1)
   (1) edge[bend right] node{$\bar f$}(11);
\end{tikzpicture}
\end{figure}
The diagram is an extension of the diagram on page~\pageref{diagram-page} that includes $\langle|X|\rangle_f$.
From above, both maps $\hat\tau_f$ and $\bar\tau_f$ are short and $\tau_f=\bar\tau_f\circ\hat\tau_f$.
Note that $\tau_f$ might be not injective while $\bar\tau_f$ is always injective.

Our first formulation uses $\langle|{*}-{*}|\rangle$-metric instead of $\langle{*}-{*}\rangle$ which was used in the old formulation.

\begin{thm}{Old theorem for $\bm{\langle|{*}-{*}|\rangle}$}\label{thm:old1}
Let $Y$ be a $\CAT(0)$ space and $s\:\DD\to Y$ a continuous map that satisfies the following property: 
if a continuous map $s'\:\DD\to Y$ agrees with $s$ on $\partial\DD$ and
\[\<|x-y|\>_{s'}\le \<|x-y|\>_{s}\]
for any $x,y\in \DD$
then the equality holds for all pairs of $x$ and $y$.
Assume that the space $\<|\DD|\>_s$ is compact.
Then $\<|\DD|\>_s$ is $\CAT(0)$.
\end{thm}

\parit{Proof.}
Note that $s$ has no bubbles; it can be proved the same way as Lemma~\ref{prop:point-complement}.
By Lemma~\ref{prop:disc-moore},  $|\DD|_s$ is a disc retract.

Note that the natural map $\bar\tau_s\:\<|\DD|\>_s\to |\DD|_s$ is injective and continuous.
Since $\<|\DD|\>_s$ is compact and $|\DD|_s$ is Hausdorff, the map $\bar\tau_s$ is a homeomorphism.
Since $\bar\pi_s$ is continuous, so is $\hat{\bar \pi}_s$.

In particular $\<|\DD|\>_s$ is a disc retract as well as $|\DD|_s$.
Therefore the mapping cylinder of the boundary curve in $|\DD|_s$ is homeomorphic to $\DD$.
Let us identify $\DD$ with the mapping cylinder of the boundary curve in $|\DD|_s$.
Denote by $h\:\DD\to|\DD|_s$ the natural projection;
it maps the cylinder to the boundary curve and does not move the points in $|\DD|_s$.
(If $|\DD|_s$ is a disc, we can assume instead that $h$ is a homeomorphism.)
Note that $\<\DD\>_{\bar s\circ h}$ is isometric to $\<|\DD|\>_s$.

If $s$ is not metric minimizing,
then there is an other map $h'\:\DD\to Y$ such that $\bar s\circ h\succcurlyeq h'$ with nonisometric majorization $\mu\:\<|\DD|\>_{\bar s\circ h}=\<|\DD|\>_s\to \<\DD\>_{h'}$.
For the composition $s'=\hat h\circ \mu\circ  \hat{\bar \pi}_s$, we have that 
\[\<x-y\>_{s'}
\le \<|x-y|\>_{s}\]
and therefore
\[\<|x-y|\>_{s'}
\le \<|x-y|\>_{s}\eqlbl{eq:x-y}\]
for any $x,y\in\DD$.

By the assumption, equality holds in \ref{eq:x-y} for any $x$ and $y$.
Since $\<|\DD|\>_s$ is compact, applying ultralimit+projection construction, we can assume that $s'$ is metric minimizing.
By the main theorem $\<|\DD|\>_{s'}$ is $\CAT(0)$ disc retract.

Note that we can assume that $\mu\:\<|\DD|\>_s\to \<\DD\>_{h'}$ is saddle;
otherwise it can be shorten which would lead to a strict inequality in \ref{eq:x-y} for some $x$ and $y$.
From the main theorem in \cite{petrunin-stadler} it follows that $\mu$ is monotonic
(we need to apply the theorem for each disc provided by Lemma~\ref{lem:discs} in the disc retract $\<\DD\>_{h'}$).
Since equality holds in \ref{eq:x-y}, $\mu$ has to be an isometry --- a contradiction.
\qeds

In the second formulation use the metric $\langle{*}-{*}\rangle$ as in the original formulation.

\begin{thm}{Old theorem for $\bm{\langle{*}-{*}\rangle}$}\label{thm:old2}
Let $Y$ be a $\CAT(0)$ space and $s\:\DD\to Y$ a continuous map that satisfies the following property: 
if a continuous map $s'\:\DD\to Y$ agrees with $s$ on $\partial\DD$ and
\[\<x-y\>_{s'}\le \<x-y\>_{s}\]
for any $x,y\in \DD$,
then the equality holds for all pairs of $x$ and $y$.
Assume that the function $(x,y)\mapsto\<x-y\>_{s}$ is continuous.
Then $\<\DD\>_s$ is $\CAT(0)$.
\end{thm}

Note that continuity of the function $(x,y)\mapsto\<x-y\>_{s}$
implies that $\<\DD\>_s$ is compact.
Therefore the former condition is stronger than the latter.  
The sketch of proof given in \cite{petrunin-metric-min} implicitly used that the metric $(x,y)\zz\mapsto\<x-y\>_{s}$ is continuous.
(We do not know if compactness of $\<\DD\>_s$ alone is sufficient.)

Theorem~\ref{thm:old2} follows from the following proposition and Theorem~\ref{thm:old1}.
 
\begin{thm}{Proposition}\label{prop:<||>=<>}
Let $Y$ be a metric space and $s\:\DD\to Y$ is a continuous map without bubbles.
Assume that the function $(x,y)\mapsto \<x-y\>_s$ is continuous.
Then 
\[\<x-y\>_s=\<|x-y|\>_s\]
for any $x,y\in \DD$.
\end{thm}

Before going into the proof, let us give an example showing that the proposition is not trivial.

Consider a pseudo-arc $P\subset \DD$ and let $s$ be the quotient map $\DD\to \DD/P$.
Evidently 
\[\<|x-y|\>_s=0\]
for any $x,y\in P$.
However since $P$ contains no curves it is not at all evident that 
\[\<x-y\>_s=0\]
for any $x,y\in P$.

We present an argument of Taras Banakh \cite{banakh};
it works for two-dimensional disc and we do not know a generalization of the proposition to higher dimensions.

\begin{thm}{Lemma}\label{lem:subdivision}
Let $Y$ be a metric space and $s\:\DD\to Y$ a continuous map without bubbles.
Assume that the function $(x,y)\mapsto \<x-y\>_s$ is continuous.
Then there is a finite collection of curves in $\DD$ with finite total $\<{*}-{*}\>$-length 
that divide $\DD$ into subsets with arbitrary small $\<{*}-{*}\>$-diameter.
\end{thm}

\parit{Proof.}
Note that given $\eps>0$ there is $\delta>0$ such that a set of diameter $\delta$ in $\<\DD\>_s$ can not separate a set of diameter at least $\eps$ from the boundary curve.
If this is not the case, 
then sets of arbitrary small diameter can separate a set of diameter at least $\eps$ from the boundary.
Passing to a limit we get a one point set that separates a set from the boundary curve; that is, $s$ has a bubble --- a contradiction. 

Let us subdivide $\DD$ into small pieces by curves, say by vertical and horizontal lines.
Since the function $(x,y)\mapsto \<x-y\>_s$ is continuous,
we can assume that all pieces have small $\<{*}-{*}\>_s$-diameter;
that is, given $\eps>0$ we can assume that $\<x-y\>_s<\eps$ for any two points $x$ and $y$ in one piece.

It remains to modify the decomposition to make the boundary curves $\<{*}-{*}\>_s$-rectifiable.

Subdivide the curves into arcs with $\<{*}-{*}\>_s$-diameter smaller $\tfrac{\delta}{5}$ and exchange this piece by a curve of $\<{*}-{*}\>_s$-length smaller than $\tfrac{\delta}{5}$.
The new arc together with the old one form a set of  $\<{*}-{*}\>_s$-diameter at most $\delta$.\footnote{Note that this arc might travel far in the Euclidean metric on $\DD$.}
Therefore we might add to a piece a subset of diameter at most $\eps$ and the total $\<{*}-{*}\>_s$-diameter of each piece remains below $3\cdot\eps$.
The curves might cut more pieces from $\DD$, but by the same argument each of these pieces will have $\<{*}-{*}\>_s$-diameter below $3\cdot\eps$.
\qeds

\parit{Proof of \ref{prop:<||>=<>}.}
Fix points $x,y\in \DD$.
It is sufficient to construct a path $\alpha$ from $x$ to $y$ in $\DD$ such that the length of $s\circ\alpha$ is arbitrary close to $\<|x-y|\>_s$.

Since $(x,y)\mapsto \<x-y\>_s$ is continuous, $\<\DD\>_s$ and therefore $\<|\DD|\>_s$ are compact.
In particular, there is a minimizing geodesic $\gamma$ from $\hat{\bar x}=\hat{\bar \pi}(x)$ to $\hat{\bar y}=\hat{\bar \pi}(y)$ in $\<|\DD|\>_s$.
Denote by $\Gamma$ the inverse image of $\gamma$ in $\DD$;
this is a connected compact set which does not have to be path connected.

To construct the needed path $\alpha$, it is sufficient to prove the following claim:

\begin{itemize}
 \item[$\bigstar$] Given $\eps>0$ there is a set $\Gamma'\subset \Gamma$ 
 and a collection of paths $\alpha_0,\dots,\alpha_n$ such that 
 (1) the total length of $s(\alpha_i\backslash\Gamma)$ is at most $\eps$, 
 (2) The set $\Gamma'$ is a union of finite collection of closed connected set $\Gamma_0,\dots,\Gamma_n$, 
 (3) diameter of each $\Gamma_i$ is at most $\eps$, 
 (4) $x\in\alpha_0$, $y\in\alpha_n$ and 
 (5) the union of $\Gamma'\cup \alpha_0\cup\dots\cup\alpha_n$ is connected.
\end{itemize}

Indeed, once the claim $\bigstar$ is proved, one can apply it recursively for a sequence of $\eps_n$ that converge to zero very fast.
Namely we can apply the claim to each of the subsets $\Gamma_n$ and take as $\Gamma''$ the union of all closed subsets provided by the claim.
This way we obtain a nested sequence of closed sets $\Gamma\supset \Gamma'\supset\Gamma''\supset\dots$ which break into finite union of closed connected subsets of arbitrary small diameter
and a countable collection of arcs with total length at most $\eps_1+\eps_2+\dots$ 
outside of $\Gamma$.
Set 
\[\Phi=\Gamma\cap \Gamma'\cap\Gamma''\cap\dots\]
Note that there is a simple curve from $x$ to $y$ that runs in the constructed arcs and $\Phi$.
The part of the curve in $\Gamma$ contributes at most $\<|x-y|\>_s$ to its $\<{*}-{*}\>_s$-length.
Therefore the total length of the curve can not exceed 
\[\<|x-y|\>_s+\eps_1+\eps_2+\dots;\]
hence the result will follow.

It remains to prove $\bigstar$.

Fix a subdivision $\Upsilon_1,\dots,\Upsilon_k$ of $\DD$ provided by Lemma~\ref{lem:subdivision} for the given $\eps$.
Denote by $\Delta$ the union of all the cutting curves.

By the regularity of $\<{*}-{*}\>_s$-length, we may cover $\Delta\cap\Gamma$ by a finite collection of arcs with total  $\<{*}-{*}\>_s$-length arbitrary close to $\<{*}-{*}\>_s$-length of $\Delta\cap\Gamma$.
Denote these arcs by $\alpha_0,\dots,\alpha_n$.
Without loss of generality we may assume that $x\in\alpha_0$ and $y\in\alpha_n$.

Consider a finite graph with the vertexes labeled by $\alpha_0,\dots,\alpha_n$;
two vertexes $\alpha_i$ and $\alpha_j$ are connected by an edge if the there is a connected set $\Theta\subset \Gamma\cap\Upsilon_k$ for some $k$ such that $\Theta$ intersects $\alpha_i$ and $\alpha_j$.
Note that the graph is connected therefore we may choose a path from $\alpha_0$ to $\alpha_n$ in the graph.

The path corresponds to a sequence of arcs $\alpha_i$ and a sequence of $\Theta$-sets.
The $\Theta$-sets that correspond to the edges in the path can be taken as $\Gamma_i$ in the claim.
Hence the claim and therefore the proposition follow. 
\qeds



