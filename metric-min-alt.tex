\documentclass{article}
\usepackage{metric-min}
%\usepackage{showkeys}

\begin{document}
\title{Metric minimzing surfaces revisited}
\author{Anton Petrunin and Stephan Stadler}
%\address{A. Petrunin\newline\vskip-4mm Math. Dept. PSU,University Park, PA 16802,USA}
%\email{petrunin@math.psu.edu}
%\address{S. Stadler\newline\vskip-4mm Math. Inst.,Universit\"at M\"unchen, Theresienstr. 39, D-80333 M\"unchen, Germany}\email{stadler@math.lmu.de}
%\thanks{A.~Petrunin was partially supported by NSF grant DMS 1309340.}


\date{}

\maketitle

\begin{abstract}
A surface which does not admit a length nonincreasing deformation is called \emph{metric minimizing}.
We show that reasonable metric minimizing surfaces in $\CAT[0]$ spaces are $\CAT[0]$ with respect to their intrinsic metric. 

\end{abstract}

\section{Introduction}

Let $\DD$ be the unit disc in the plane and $Y$ a $\CAT[0]$ space.
We will be interested in the metric minimizing maps $s\:\DD\to Y$;
that is maps which minimize the induced intrinsic metric.

To define \emph{metric minimizing map}, we first need a definition of induced intrinsic metric.
Assume $s\:\DD\to Y$ is a map, not necessary continuous.
Given two points $x,y\in \DD$, set 
\[\|x-y\|_s=\inf\{\,\length (s\circ\gamma)\,\}\] 
where the infimum is taken for all paths $\gamma\:[0,1]\to\DD$ 
connecting $x$ to $y$ such that the composition $s\circ\gamma$ is rectifiable path (in prticular continous).
The value $\|x-y\|_s$ is called \emph{intrinsic distance} between the points $x$ and $y$ \emph{induced} by $s$; 
it may take infinite value.

The value $\|x-y\|_s$ can vanish at distinct points, in other words $\|{*}-{*}\|_s$ is a \emph{pseudometric} on $\DD$.
The corresponding metric space will be denoted as $\|\DD\|_s$;
that is points in $\|\DD\|_s$ are the equivalence classes $[x]$ for the equivalence relation $x\sim y$ $\iff$ $\|x-y\|_s=0$
and the distance defined as $\|[x]-[y]\|_s=\|x-y\|_s$.

If $s'\:\DD\to Y$ is an other map which agree with $s$ on the boundary $\partial \DD$,
we will write $s\succcurlyeq s'$ if there is a short map 
\[i\:(\DD,\|{*}-{*}\|_s)\to (\DD,\|{*}-{*}\|_{s'})\] 
which is identical on the boundary. 

Finally, if $s\succcurlyeq s'$ implies $s'\succcurlyeq s$ we say that $s$ is \emph{metric minimizing}.

\begin{thm}{Main theorem}\label{thm:mainintro}
Let $Y$ be a $\CAT[0]$ space and $s\:\DD\to Y$ is metric minimzing map.
Assume the boundary curve $s|_{\partial\DD}$ is rectifiable.
Then the space $\|\DD\|_s$ is a $\CAT[0]$ space homeomorphic to a disc retract.

Moreover if $s|_{\partial\DD}$ is injective then $\|\DD\|_s$ is homeomorphic to a disc.
\end{thm}

\section{Planar spaces}

In this section we will reformulate the main theorem, in terms of \emph{planar spaces} which is more suitable for the proof.  

For a general map the space $\|\DD\|_s$ defined above might have few \emph{metric components};
that is the maximal subsets of points on finite distance from each other.
These components might be bad fractal, 
but the metric on one component is planar in the sense of the following definition.

\begin{thm}{Definition}
A geodesic metric space is called \emph{planar} if it admits an injective continuous map into the plane.
\end{thm}

The existence of an injective continuous map into the plane is evident;
the existence of geodesics between points follow  from the lemma below;
this is a classical statement (see for example \cite[II-\S8 Thm. 3]{KF}).


\begin{thm}{Lemma}\label{lem:geospace}
Let $X$ be a compact geodesic space, $Y$ a metric space and $f\:X\to Y$ a map. 
Then each metric component of $\|X\|_f$ is a geodesic space.
\end{thm}

\parit{Proof.}
Assume $\gamma_n$ is a sequence of constant speed paths from $x$ to $y$ in $|X|_f$
such that $\length\gamma_n\to \|x-y\|_f$.
Since $|X|_f$ is compact, we can pass to its partial limit $\gamma$.
Clearly $\gamma$ is a geodesic from $x$ to $y$ for the metric $\|x-y\|_f$.
\qeds


Let 
$\gamma\:\SS^1\to Y$ be a closed curve,
$X$ be a planar space 
and $s\:X\to Y$ be a map.
We say that $s$ majorize $\gamma$ if $s$ is short and $\gamma$ factors through $s$;
that is $\gamma=s\circ\bar\gamma$ for a curve $\bar\gamma\:\SS^1\to X$.

A majorization $\SS^1\xrightarrow{\bar\gamma}X\xrightarrow{s}Y$ of $\gamma$ is called \emph{minimizing} 
if there is no short map $s'\:X\to Y$ which agree with $s$ on $\bar\gamma(\SS^1)$ and distinct from $s$.

A majorization $\SS^1\xrightarrow{\bar\gamma}X\xrightarrow{s}Y$ of $\gamma$ is called \emph{totally minimizing}
if for any other majorization $\SS^1\xrightarrow{\bar\gamma'}X'\xrightarrow{s'}Y$ of $\gamma$
there is no short map $i\:X\to X'$ such that $\gamma'=i\circ\gamma$ which is strictly short on a pair of pints in $X$. 

Assume $s\:\DD\to Y$ is a metric minimizing map as in the main theorem \ref{thm:mainintro}.
Denote by $X$ the metric component of $\partial \DD$ in $\|\DD\|_s$.
Note that the map $X\to Y$ induced by $s$ is a totally minimizing majorization of $\gamma=s|_{\partial \DD}$.
Note that the map  which sends all the metric components in $\|\DD\|_s$ other than $X$ to one point in $X$ is short.
Therefore the ain theorem can be reformulated the following way.

\begin{thm}{Reformulation of the main theorem}\label{thm:reformulation}
Let $Y$ be a $\CAT[0]$ space and $\SS^1\xrightarrow{\bar\gamma}X\xrightarrow{s}Y$ be a totally minimizing majorization of a closed rectifiable curve $\gamma\:\SS^1\to Y$.
Then $X$ is a $\CAT[0]$ space homeomorphic to a disc retract.

Moreover if in addition $\gamma$ is a simple curve, then $X$ is homeomorphic to the disc.
\end{thm}

\section{Finite-whole extension lemma}\label{Finite-whole extension lemma}

\begin{thm}{Finite-whole extension lemma}\label{lem:finite-whole}
Let $X$ and $Y$ be a metric spaces, $A\subset X$ and $f\:A\to Y$ is a short map.
Assume $Y$ is compact and for any finite set $F\subset X$ there is a short map $F\to Y$ which agrees with $f$ in $F\cap A$.
Then there is a short map $X\to Y$ which agrees with $f$ in $A$.
\end{thm}


\parit{Proof.}
Given a finite set $F\subset X$,
denote by $\mathfrak{S}_F$ the set of all short maps $h\: F\to Y$ which agree with $f$ in $F\cap A$.
Given $x\in F$ consider the set
\[K(F,x)=\set{h(x)\in Y}{h\in \mathfrak{S}_F}.\]

By assumption of lemma 
\[K(F,x)\ne\emptyset\eqlbl{Kne0}\] for any finite set $F\subset X$ and $x\in F$.
Further, note that $K(F,x)$ is closed and
\[K(F',x)\supset K(F,x)\eqlbl{KinK}\]
for any other finite set $F'$ such that 
$x\in F'\subset F$.

Without loss of generality we can assume that $A$ (the domain of $f$) is a maximal set with respect to inclusion such that $f$ satisfies the assumptions of the lemma;
that is, we can not add to the domain of $f$ a single point in such a way that the assumptions of the lemma still holds.
(It follows that $A$ is closed subset of $X$, but we will not use it.)

Arguing by contradiction, assume $A\ne X$; fix $x\in X\backslash A$.
Given $y\in Y$ there is a finite set $F\ni x$ such that $y\notin K(F,x)$.
Or equivalently the intersection of all $K(F,x)$ for all finite sets $F$ including $x$ is empty.

By finite intersection property, 
we can choose a finite collection of finite sets $F_1,\dots F_n$ containing $x$ such that 
\[K(F_1,x)\cap \dots\cap K(F_n,x)=\emptyset.\eqlbl{K(F,x)}\]

Since the union $F=F_1\cup\dots\cup F_n$ is finite, \ref{Kne0} and \ref{KinK}
imply
\[K(F_1,x)\cap \dots\cap K(F_n,x)\supset K(F,x)\ne \emptyset\]
which contradicts \ref{K(F,x)}.
\qeds

\begin{thm}{Finite-whole extension lemma}\label{lem:finite-whole}
Let $X$, $Y$ and $Z$ be a metric spaces, 
$A\subset X$ and $f\:A\to Z$ is a short map.
Assume $Y$ and $Z$ are compact and for any finite set $F\subset X$ there is a short maps 
$\alpha_F\:F\to Y$ 
and $\beta_F\:F\to Z$ such that 
$\beta_F$ agrees with $f$ in $F\cap A$
and 
\[|\alpha_F(x)-\alpha_F(y)|_Y\ge |\beta_F(x)-\beta_F(y)|_Z\]
for any $x,y\in F$.
Then there is a short maps $\alpha\:X\to Y$ and $\beta\:X\to Y$ which agrees with $f$ in $A$
and
\[|\alpha(x)-\alpha(y)|_Y\ge |\beta(x)-\beta(y)|_Z\]
for any $x,y\in X$.
\end{thm}

The proof is nearly identical.

\parit{Proof.}
Given a finite set $F\subset X$,
denote by $\mathfrak{S}_F$ the set of all pairs of short maps $(\alpha_F,\beta_F)$ satisfying the assumption of lemma.
Given $x\in F$ consider the set
\[K(F,x)=\set{\alpha_F(x),\beta_F(y)\in Y\times Z}{(\alpha_F,\beta_F)\in \mathfrak{S}_F}.\]

By assumption of lemma 
\[K(F,x)\ne\emptyset\eqlbl{Kne0}\] for any finite set $F\subset X$ and $x\in F$.
Further, note that $K(F,x)$ is closed and
\[K(F',x)\supset K(F,x)\eqlbl{KinK}\]
for any other finite set $F'$ such that 
$x\in F'\subset F$.

Without loss of generality we can assume that $A$ (the domain of $f$) is a maximal set with respect to inclusion such that $f$ satisfies the assumptions of the lemma;
that is, we can not add to the domain of $f$ a single point in such a way that the assumptions of the lemma still holds.
(It follows that $A$ is closed subset of $X$, but we will not use it.)

Assume $A\ne X$; fix $x\in X\backslash A$.
Given $(y,z)\in Y\times Z$ there is a finite set $F\ni x$ such that $(y,z)\notin K(F,x)$.
Or equivalently the intersection of all $K(F,x)$ for all finite sets $F$ including $x$ is empty.

By finite intersection property, 
we can choose a finite collection of finite sets $F_1,\dots F_n$ containing $x$ such that 
\[K(F_1,x)\cap \dots\cap K(F_n,x)=\emptyset.\eqlbl{K(F,x)}\]

Since the union $F=F_1\cup\dots\cup F_n$ is finite, \ref{Kne0} and \ref{KinK}
imply
\[K(F_1,x)\cap \dots\cap K(F_n,x)\supset K(F,x)\ne \emptyset\]
which contradicts \ref{K(F,x)}.
It follows that we can assume that $A=X$.

That is, for any finite set $F\in X$ there is a map $\alpha\:F\to Y$ such that 
\[|x-y|_X \ge |\alpha_F(x)-\alpha_F(y)|_Y \ge |f(x)-f(y)|_Z\]
for any $x,y\in F$.

\qeds


\section{Length minimizing graphs}\label{Metric minimizing graphs}

Assume $\Gamma$ is a  finite graph and $A$ is a subset of its vertices.
Let $Y\in\CAT[0]$ and $f\:\Gamma\to Y$ be an arbitrary map.
The map $f$ is called \emph{edge minimizing} relative to $A$ if there is no map $f'\:\Gamma\to Y$
such that 
\[\length f\circ\gamma\ge \length f'\circ\gamma\] 
for any path $\gamma\:[0,1]\to\Gamma$ and for some path the inequality is strict. 
We say that $f\:\Gamma\to Y$ is \emph{strictly edge minimizing} if  there is no map $f'\:\Gamma\to Y$ distinct from $f$
such that 
\[\length f\circ\gamma\ge \length f'\circ\gamma\] 
for any path $\gamma\:[0,1]\to\Gamma$.

\begin{thm}{Proposition}\label{prop:metric-min-graph}
Let $Y$ be a $\CAT[0]$ space, 
$\Gamma$ a finite  graph and $A$ a subset of its vertices.
Assume $f\:\Gamma\to Y$ is weakly  minimizing relative to $A$.
Then
\begin{itemize}
\item each edge of $\Gamma$ maps to a geodesic
\item for any vertex $v\notin A$ and any $x\ne f(v)$
there is an edge  $[vw]$ in $\Gamma$ such that
$\measuredangle[f(v)\,^{f(w)}_x]\ge \tfrac\pi2$.
\end{itemize}
Moreover, $f$ is strictly??? metric minimizing relative to $A$. 
\end{thm}

\begin{wrapfigure}{r}{22 mm}
\begin{lpic}[t(-0 mm),b(-0 mm),r(0 mm),l(0 mm)]{pics/not-sufficient(1)}
%\lbl[lb]{12.5,11;$W_0$}}
\end{lpic}
\end{wrapfigure}

As one may see from the diagram,
the two conditions in the proposition do not guarantee that the map $f$ is metric minimizing,
the solid points form the set $A$.

\parit{Proof.}
The first condition should be obvious.

Assume the second condition does not hold at a vertex $v\zz\notin A$;
that is, there is a point $x\in Y$ such that
$\measuredangle[f(v)^{f(w)}_x]< \tfrac\pi2$
for any adjacent vertex $w$.
In this case moving $f(v)$ toward $x$ along $[f(v)x]$ decrease the lengths of all edges adjacent to $v$, a contradiction.

%???HERE IS A COPY OF SEC 10

To prove the last statement, assume there is a map $f'$ distinct from $f$ such that $f|_A=f'|_A$ and $f\succcurlyeq f'$.
Denote by $g(x)$ the midpoint of $f(x)$ and $f'(x)$ for any $x\in \Gamma$. 
By comparison $f\succcurlyeq g$.
It follows that the tautological map $\|\Gamma\|_f\to \|\Gamma\|_g$ is an isometry.
The later implies that the distance $|f(v)-g(v)|$ is the same for all the vertices $v$ in $\Gamma$.
Since we have $|f(v)-g(v)|=0$ for any $v\in A$,
we get $f(v)=g(v)$ for any vertex $v$ in $\Gamma$.
Hence $f=f'$, a contradiction.
\qeds

Assume $\Gamma$ is a finite graph embedded in the plane $\RR^2$;
in particular $\Gamma$ is planar.
The complement to the unbounded connected component of $\RR^2\backslash\Gamma$ will be called filling of $\Gamma$;
it will be denoted as $\Fill\Gamma$.

The vertex of $\Gamma$ will be called \emph{boundary vertex}
if it lies in the boundary $\partial_{\RR^2}[\Fill\Gamma]$,
otherwise it will be called \emph{interior vertex}.

\begin{thm}{Corollary}\label{cor:planar-minimizing-graph}
Let $Y$ be a $\CAT[0]$ space and
$\Gamma$ an embedded finite graph in $\RR^2$.
Assume $f\:\Gamma\to Y$ is a edge minimizing relative to the boundary vertices. 
Then there is a $\CAT[0]$ disc retract $W$,
a short map $q\:W\to Y$ so that $f$ factorize as
and the maps $\Gamma\xrightarrow{f'} W \xrightarrow{q} Y$.
\end{thm}


\parit{Proof.}
Fix a cycle $\gamma$ in $\|\Gamma\|_f$ which bounds one of the discs in the complement $\RR^2\backslash \|\Gamma\|_f$.
Set $\ell=\length \bar f\circ\gamma$.

By Reshetnyak's majorization theorem, there is a convex polygon $P$ (possibly degenerate) with perimeter $\ell$ which admits 
a short map to $Y$ in such a way that $f\circ\gamma$ is formed by the image of the boundary.
Note that each angle of $P$ is at least as big as 
the angle between the corresponding edges.

Prepare a polygon as above for each disc in the complement of $\Gamma$
and glue these polygons into $\Gamma$ along the natural map.
The obtained space $D$ which is simply connected.
Therefore in order to show that $D$ is $\CAT[0]$,
we need to check that the sum of the angles around each interior vertex in $\|\Gamma\|_f$ is at least $2\cdot\pi$.


Assume the contrary, that is, 
the sum of the angles around a fixed interior vertex $v$ is less than $2\cdot\pi$.
The space of directions $\Sigma_{f(v)}$ is a $\CAT[1]$ space.
The directions of the edges from $v$ have a natural
cyclic order say $\xi_1,\dots,\xi_k$
such that
\[\measuredangle(\xi_1,\xi_2)+\dots+\measuredangle(\xi_k,\xi_1)<2\cdot\pi.\]
By Reshetnyak's majorization theorem,
the closed broken line $\xi_1,\dots,\xi_k$ is majorized by a convex spherical polygon $P$.
Note that $P$ lies in an open hemisphere with the pole  at some point in $P$.
Choose $x\in Y$ so that the direction form $f(v)$ to $x$ coincides with the image of the pole in $\Sigma_{f(v)}$.
This choice of $x$ contradicts the condition in Proposition~\ref{prop:metric-min-graph}.

The short maps from Reshetnyak's majorization theorem fit together to yield the existance of the required short maps.
\qeds

\section{Key Lemma}\label{Key Lemma}


\begin{thm}{Key Lemma}\label{lem:key}
Let $Y$ be a $\CAT[0]$ space and $X$ 
be a metric minimizing filling of a closed curve $\gamma$ in $Y$.
Given a finite set $F\subset X$ 
there is 
(1) a $\CAT[0]$ space $W$, which is a disc retract,
and (2) short maps $p\:F\to W$ and $q\:W\to Y$ such that
\[\bar s(x)=q\circ p(x)\] 
for any $x\in F\cap \partial |\DD|_s$
and 
\[|p(x)-p(y)|_W\le |x-y|_X\] 
for any $x,y\in F$.
\end{thm}

\parit{Proof.}
Let us connect each pair $x,y$ of points in $F$ by geodesics.

We can assume that 
every pair of the constructed geodesics 
are either disjoint, or their intersection is formed by finite collections of arcs and points.

Indeed, if some number of geodesics $\gamma_1,\dots,\gamma_n$ has this property and we are given points $x$ and $y$, then
we choose a minimizing geodesic $\gamma_{n+1}$ from $x$ to $y$ which maximizes the time it spends in $\gamma_1,\dots,\gamma_n$  in the order of importance.
Namely, 
\begin{itemize}
\item  among all minimizing geodesics connecting $x$ to $y$
choose one which spends maximal time in $\gamma_1$ --- in this case $\gamma_{n+1}$ intesects $\gamma_1$ along the empty set, one-point set or a closed arc.
\item among all minimizing geodesics as above
choose one which spends maximal time in $\gamma_2$ --- in this case $\gamma_{n+1}$ intesects $\gamma_2$ along at most two arcs and points.
\item and so on.
\end{itemize}

%IT SHOULD BE POSSIBLE TO ENSURE THAT INTERSECTION OF ANY TWO GEODESIC IS EMPTY OR CONNECTED, BUT I FAILED TO MAKE IT FORMALLY???

In particular the set of all these geodesics forms a finite graph, say $\Gamma$,
embedded in $X$. 

According to Proposition~\ref{prop:|D|},
$|\DD|_s$ admits an embedding into the plane.
Therefore $\Gamma$ can be considered as a graph embedded into the plane.
By Proposition~\ref{prop:exist}, there is a map $u\:\Gamma\to Y$ metric minimizing relative to $\Gamma\cap\partial|\DD|_{\bar s}$ with
$\bar s|_\Gamma\succcurlyeq u$.
Hence there is a short map $\alpha\:\|\Gamma\|_{\bar s|_\Gamma}\to\|\Gamma\|_u$. 
By Corollary~\ref{cor:planar-minimizing-graph} there is a $\CAT[0]$ disc retract $W$, a length preserving embedding
$\|\Gamma\|_u\hookrightarrow W$ and a map $\hat u\:W\to Y$ extending $\bar u$. %???why not to say that q= \hat u???
Set $p=\alpha|_F$ and $q=\hat u$.
\qeds

\section{Compactness of planar CAT[0] spaces}\label{Compactness}

Let $\mathcal{K}_\ell$ be the set of isometry classes of $\CAT[0]$ metrics on a disc retract with rectifiable
boundary curves of length at most $\ell$.


Here is the main statement in this section.

\begin{thm}{Compactness lemma}\label{lem:compact}
$\mathcal{K}_\ell$ is compact in the Gromov--Hausdorff topology.
\end{thm}

It follows immediately from Lemmas \ref{lem:precompact} and \ref{lem:closed} proven below.

\begin{thm}{Lemma}\label{lem:precompact}
$\mathcal{K}_\ell$ is precompact in the Gromov--Hausdorff topology.
\end{thm}

Further $\area K$ denotes the two-dimensional Hausdorff measure of a metric space $K$. 

\parit{Proof.}
Let $K$ be a metric space with metric $d$ and isometry class in $\mathcal {K}_\ell$.
By Reshetnyak's theorem there is a short map from a convex plane figure $F$ with perimeter at most $\ell$ onto $K$.
In particular, $\area K \le \area F \le \tfrac{\ell^2}{4\pi}$.

Fix $\eps>0$. 
Set $m=\lceil 10\cdot\tfrac\ell\eps\rceil$.
Choose $m$ points $y_1,\dots,y_m$ on $\partial K$
which divide $\partial K$ into arcs of equal length.

Consider the maximal set of points $\{x_1,\dots,x_n\}$ such that $d(x_i,x_j)>\eps$ and $d(x_i,y_j)>\eps$.

Note that the set $\{x_1,\dots,x_n,y_1,\dots,y_m\}$
forms an $\eps$-net in $(K,d)$.

Further note that the balls $B_i=B_{\eps/2}(x_i)$
do not overlap.
By comparison,
\[\area B_i\ge \tfrac{\pi\cdot\eps^2}{4}.\]

It follows that $n\le \tfrac{1}{\pi^2}\cdot\left(\tfrac\ell\eps\right)^2$.
That is, there is a integer valued function $N(\eps)$,
such that for  
$(K,d)$ contains an $\eps$-net
with at most $N(\eps)$ points.

In other words, $\mathcal{K}_\ell$ is uniformly totally bounded.
Any class of metrics with this property is precompact in Gromov--Hausdorff topology; 
see for example \cite[7.4.15]{BBI}.
\qeds





\begin{thm}{Lemma}\label{lem:closed}
$\mathcal{K}_\ell$ is closed in the Gromov--Hausdorff topology.
\end{thm}

\parit{Proof.}
Let $(X_n)$ be a sequence in $\mathcal{K}_\ell$ with $X_n\to X_\infty$. 
Set $r_n=\frac{\length\partial X_n}{2\pi}$ 
and $B_n=B_{r_n}(0)\subset \mathbb{R}^2$.
Let $\gamma_n:\partial B_n\to\partial X_\infty$ be an arc length parametrization.
Choose a point $p_n\in X_n$ with $X_n\subset B_{r_n}(p_n)$ and define
$f_n:B_n\to X_n$ by sending the geodesic $[0\theta]$ for $\theta\in\partial B_n$ to the geodesic $[p_n\gamma_n(\theta)]$ with constant speed. 
Note that $f_n$ is short.
We obtain a limit map $f_\infty:B_\infty\to X_\infty$ which is short and maps radial geodesics in $B_\infty$ to geodesics in $X_\infty$ joining the limit point $p_\infty$
to a boundary point.
Note that if two points $x$  and $y$ map to the same point $q$ under $f_\infty$, then $q$ is a branch point in $X_\infty$ and there is a path $c$ connecting 
$x$ and $y$ which also maps to $q$. 
In particular, $f^{-1}(q)$ is connected.
On the other hand, since we can lift radial geodesics from $X_\infty$ to $B_\infty$ we see that $X_\infty-f^{-1}(q)$
is connected as well.
As in Proposition~\ref{prop:disc-moore} the claim now follows from Moore's theorem in \cite{moore}.
\qeds

\section{Geodesic homotopy}

Given two points $p_0$ and $p_1$ in a $\CAT[0]$ space $Y$,
denote by $p_t$ the point $\gamma(t)$ on the 
the geodesic path $\gamma\:[0,1]\to Y$ from $p_0$ to $p_1$;
that is, $\gamma(0)=p_0$ and $\gamma(1)=p_1$.

Note that the map $(p_0,p_1,t)\mapsto p_t$ is continuous.
Therefore given two continuous maps $f_0,f_1\:K\to Y$,
the one parameter family of maps $f_t\:K\to Y$ 
is a homotopy;
further it will be called \emph{geodesic homotopy}.

Let $\gamma_{0}, \gamma_1\:\II\to Y$ be two rectifiable curves parametrized by arc length. 
We say that  $\gamma_{0}$ is {\em parallel} to $\gamma_{1}$, if the function $t\mapsto |\gamma_{0}(t)-\gamma_{1}(t)|_Y$ is constant.


\begin{thm}{Lemma}\label{lem:parpaths}
Let $\gamma_0,\gamma_1\:[0,\ell]\to Y$ be two rectifiable curves. 
Denote by $\gamma_t\:[0,\ell]\to Y$ their geodesic homotopy.
Then
\[\length\gamma_{\frac12}
\le \tfrac12\cdot\length\gamma_0 +\tfrac12\cdot\length\gamma_1\]
for any $t\in [0,1]$.

If in addition $\gamma_0$ and $\gamma_1$ are parametrized by arc length and 
\[\length\gamma_{\frac12}
=\length\gamma_0
=\length\gamma_1,\] 
then $\gamma_{0}$ and $\gamma_{1}$ are parallel. 
\end{thm}

\parit{Proof.}
%Ref to existance of |\dot\gamma_0(t)|???
From comparison, we get
\[2\cdot|\dot\gamma_{\frac12}(t)|^2
\le
|\dot\gamma_{0}(t)|^2
+|\dot\gamma_{0}(t)|^2
-\tfrac12\cdot(|\gamma_{0}(t)-\gamma_{1}(t)|')^2\] 
for almost all $t$.
Taking the integral, we get the first statement.

If in addition $\gamma_0$ and $\gamma_1$ are parametrized by arc length, 
then $|\dot\gamma_{\frac12}(t)|\le 1$.

By assumption, the length of $\gamma_{\frac12}$ equals $\ell$. 
Hence $|\dot\gamma_{\frac12}(t)|=1$ for almost all $t\in[0,\ell]$. 
In particular, $\gamma_{\frac12}$
is parametrized by arc length. 

It follows that $|\gamma_{0}(t)-\gamma_{1}(t)|'=0$ for almost all $t$.
Since the function 
\[t\mapsto |\gamma_{0}(t)-\gamma_{1}(t)|\] 
is Lipschitz,
the latter  implies that it is constant,
and therefore $\gamma_{0}$ and $\gamma_{1}$ are parallel.
\qeds

\parbf{Remark.}
In fact, it is possible to show in the above situation that the induced intrinsic metric on $[0,1]\times[0,\ell]$
for the map $(t,\tau)\mapsto \gamma_t(\tau)$ is a product metric.

\begin{thm}{Corollary}\label{cor:parpaths}
Let $\gamma_0,\gamma_1\:[0,1]\to Y$ be two rectifiable paths both starting in a point $p\in Y$ and of equal length $\ell>0$. 
If the lengths in their geodesic homotopy $\gamma_t$ is constant, then $\gamma_{0}=\gamma_{1}$.
\end{thm}

\begin{thm}{Uniqueness}\label{prop:strict-mm}
Let $X$ be a geodesic space, 
$A\subset X$ be a closed subset 
and $Y$ be a $\CAT[0]$ space.
Assume that the map $s_0\:X_0\to Y$ is continuous and metric minimizing relative to $A$ and 
$s_1\preccurlyeq s_0$ for an other map $s_1\:X_1\to Y$.
Then there is $i\:X_0\to X_1$ such that $s_0=s_1\circ i$.
\end{thm}

\parit{Proof.}
Let $i$ be as in the definition of preorder  $\preccurlyeq$.
Note that it is distance preserving map $X_0\to X_1$.

For $t\in[0,1]$ denote $s_t\:X\to Y$ the geodesic homotopy between $s_0$ and $s_1\circ i$. 
Since $s_0$ is metric minimizing, $s_t\:X\to Y$ are length preserving for any $t\in[0,1]$.

From Corollary \ref{cor:parpaths} we see that $s_0$ agrees with $s_1$ on any rectifiable curve starting at $A$.
Hence the statement follows.
\qeds

\section{Main theorem}\label{Main theorem}

\begin{thm}{Main Theorem}\label{thm:main}
Let $Y$ be a $\CAT[0]$ space, 
and $\SS^1\xrightarrow{\bar\gamma}X\xrightarrow{s}Y $ is a metric minimizing majorization of a closed rectifiable curve $\gamma\:\SS^1\to Y$.
Then $X$ is a $\CAT[0]$ space homeomorphic to a disc retract. 
\end{thm}

\parit{Proof.}
Given a finite set $F\subset X$,
denote by $\mathcal{W}_F$
the set of isometry classes of spaces $W$ which meet the conditions of the Key Lemma~\ref{lem:key}
for $F$;
according to the key lemma~\ref{lem:key} $\mathcal{W}_F\ne\emptyset$.
Note that for two finite sets $F\subset F'$ in $X$,
we have $\mathcal{W}_F\supset \mathcal{W}_{F'}$.

According to the Lemma on compactness (\ref{lem:compact}) $\mathcal{W}_F$ is compact.
Therefore 
\[\mathcal{W}
=
\bigcap_{F}\mathcal{W}_F\ne \emptyset\]
where the intersection is taken over all finite subsets $F\subset X$. 


Fix a space $W$ from $\mathcal{W}$.
Note that for any finite set $F\subset X$
there are short maps 
$p_F\:F\to W$ 
and
$q_F\:W\to Y$ such that
$q_F\circ p_F(x)=s(x)$ for any $x\in \bar\gamma(\SS^1)\cap F$.


Recall that by the key lemma~\ref{lem:key}, the space $W$ is a $\CAT[0]$ disc retract
and the map  satisfy the assumptions in the finite-whole lemma (\ref{lem:finite-whole}).
It follows that there are maps $p\: X\to W$ and $q\:W\to Y$
such that $q\circ p(x)=s(x)$ for any $x\in \bar\gamma(\SS^1)$

Since $s$ is metric minimizing, 
both maps $p$ and $q$ are metric minimizing and $p$ is distance preserving.

Denote by $K$ the convex hull of $p\circ\bar\gamma(\SS^1)$ in $W$.
Let us show that 
\[p(X)=K\eqlbl{XinK}\]
Note that $p(X)\supset p\circ\bar\gamma(\SS^1)$ and $p(X)$ is a convex.
Therefore $p(X)\supset K$.
Further since the closest point projection $\pi_K\:W\to K$ is short,
so is $q \circ \pi_K \circ p\:X\to Y$.
If $p(X)\backslash K\ne\emptyset$ then $\pi_K$ shortens some curves in $p(X)$;
the latter contradicts metric minimality of $s$.
Hence \ref{XinK} follows.

By \ref{XinK}, $X$ is isometric to $K$.
hence the statement follows.
\qeds



\begin{thebibliography}{52}

\bibitem{A} Alexandrov, A. D. ``Ruled  surfaces  in  metric  spaces,'' Vestnik Leningrad. Univ., 12:5-26, 1957 (Russian).

\bibitem{bressan} Bressan, A.
``Hyperbolic systems of conservation laws.
The one-dimensional Cauchy problem.'' 
Oxford Lecture Series in Mathematics and its Applications, 20. 
Oxford University Press, Oxford, 2000. 
xii+250 pp.

\bibitem{BBI}Burago, D.; Burago, Y.; Ivanov, S.
A course in metric geometry.
Graduate Studies in Mathematics, 33. American Mathematical Society, Providence, RI, 2001. xiv+415 pp.

\bibitem{GS} Gromov, Mikhail, and Richard Schoen. "Harmonic maps into singular spaces and p-adic superrigidity for lattices in groups of rank one." Publications Mathématiques de l'IHÉS 76.1 (1992): 165-246.

\bibitem{H} Hamilton, R. S. ``Harmonic Maps of Manifolds with Boundary,'' Lecture Notes in Mathematics, Springer, 1975, ISBN 978-3-540-37530-2.

\bibitem{HKST} Heinonen, J.;  Koskela, P.;  Shanmugalinga, N.; Tyson, J. ``Sobolev spaces on metric measure spaces,''
volume 27 of New Mathematical Monographs. Cambridge University Press, Cambridge, 2015.

\bibitem{KF}
\begin{otherlanguage}{russian}
Колмогоров, А. Н.;
Фомин, С. В.,
\textit{Элементы теории функций и функционального анализа.}
Издание седьмое, 2004.
\end{otherlanguage}

\bibitem{KS}Korevaar, N. J.; Schoen, R. M. ``Sobolev spaces and harmonic maps for metric space targets,'' Comm. Anal. Geom., 1(3-4):561-659, 1993.

\bibitem{LSW} Lytchak, A.; Stadler, S.; Wenger, S.  ``On conformal changes of CAT(0) spaces'', in preparation.

\bibitem{LW}Lytchak, A.; Wenger, S. ``Area minimizing discs in metric spaces,'' preprint arXiv:1502.06571, 2015.

\bibitem{LW2}Lytchak, A.; Wenger, S. ``Energy and area minimizers in metric spaces,'' preprint  arXiv:1507.02670, 2015.

\bibitem{LW3}Lytchak, A.; Wenger, S. ``Intrinsic structure of minimal discs in metric spaces ,'' preprint  arXiv:1602.06755, 2016.

\bibitem{LW4}Lytchak, A.; Wenger, S. ``Regularity of harmonic discs in spaces with quadratic isoperimetric inequality  ,'' preprint  arXiv:1512.01060, 2016.

\bibitem{moore}
Moore, R. L.,
``Concerning upper semi-continuous collections of continua,''
Trans. Amer. Math. Soc. 27 no. 4 (1925) pp. 416--428.

\bibitem{petrunin-metric-min} Petrunin, A.
``Metric minimizing surfaces.''
Electron. Res. Announc. Amer. Math. Soc. 5 (1999), 47--54 

\bibitem{petrunin-intrinisic} Petrunin, A.
``Intrinsic isometries in Euclidean space.''
St. Petersburg Math. J. 22 (2011), no. 5, 803--812 

\bibitem{petrunin-orthodox} Petrunin, A. 
``Exercises in Orthodox Geometry''
{\tt arXiv:0906.0290 [math.HO]}

\bibitem{R}Reshetnyak, Yu. G. ``Sobolev classes of functions with values in a metric space,'' II, Sibirsk. Mat. Zh. 45 (2004), no. 4, 855-870. MR 2091651 (2005e:46055)

\bibitem{Se} Serbinowski,  T. ``Boundary regularity of harmonic maps to nonpositively curved metric spaces,''
Comm. Anal. Geom. , 2(1):139-153, 1994.

\bibitem{S}Schoen, R. ``Analytic Aspects of The Harmonic Map Problem'' Chapter IX in  
``Lectures on harmonic maps'' by Schoen, R.; Yau, S. T.,  
Conference Proceedings and Lecture Notes in Geometry and Topology, II. International Press, Cambridge, MA, 1997. vi+394 pp. ISBN: 1-57146-002-0

\bibitem{shefel-2D} 
\begin{otherlanguage}{russian}
Шефель, С. З.,
\textit{О седловых поверхностях ограниченной спрямляемой кривой.}
Доклады АН СССР, 162 (1965) №2, 
294---296.
\end{otherlanguage}
%\v{S}efel', S.,
%\textit{On saddle surfaces bounded by a rectifiable curve,} 
%Dokl. Akad. Nauk SSSR 
%162 
%(1965), 
%294--296.

\bibitem{shefel-3D} 
\begin{otherlanguage}{russian}
Шефель, С. З., 
\textit{О внутренней геометрии седловых поверхностей.}
Сибирский математический журнал, 5 (1964), 1382---1396
\end{otherlanguage}
%\v{S}efel', S., 
%\textit{On the intrinsic geometry of saddle surfaces,} Sibirsk. Mat. \v{Z}. 
%5 
%(1964), 
%1382--1396

\bibitem{St} Stadler, S. ``Harmonic discs in CAT(0) spaces'', in preparation.

%\bibitem{W1}Whyburn, G. T., ``On sequences and limiting sets,'' Fund. Math. vol. 25 (1935) pp. 408-426.

%\bibitem{W2}Whyburn, G. T., ``Analytic topology,'' Amer. Math. Soc. Colloquium Publications, vol. 28, 1942.

%\bibitem{Wi}Wilder, R. L., ``Topology of Manifolds,'' American Mathematical Society Colloquium Publications, vol. 32. American Mathematical
%Society, New York, N. Y., 1949.
\end{thebibliography}



\end{document}