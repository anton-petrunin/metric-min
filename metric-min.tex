\documentclass[a4paper,10pt]{amsart}
\usepackage{metric-min}

\begin{document}
\title{Metric minimzing surfaces revisited}
\author{Anton Petrunin}
\address{A. Petrunin\newline\vskip-4mm
Math. Dept. PSU,
University Park, PA 16802,
USA}
\email{petrunin@math.psu.edu}
\author{Stephan Stadler}
\address{S. Stadler\newline\vskip-4mm
Math. Inst.,
Universit\"at M\"unchen, Theresienstr. 39, D-80333 M\"unchen, Germany}
\email{stadler@math.lmu.de}
%\thanks{A.~Petrunin was partially supported by NSF grant DMS 1309340.}


\date{}

\begin{abstract}
We say that a surface is metric minimizing if it does not admit a metric decreasing deformation.
We show that metric minimizing surfaces in $\CAT[0]$ spaces are $\CAT[0]$.
\end{abstract}
\maketitle



\section{Definitions}

Let $X$ be a set.
A \emph{pseudometric} $|{*}-{*}|$ on a set $X$ 
is a function $X\times X\to[0,\infty]$
such that 
\begin{itemize}
\item $|x-x|=0$, for any $x\in X$;
\item $|x-y|=|y-x|$, for any $x,y\in X$;
\item $|x-y|+|y-z|\ge|x-z|$ for any  $x,y,z\in X$.
\end{itemize}

If in addition $|x-y|=0$ implies $x=y$ then $|{*}-{*}|$ is called \emph{metric}.

Note that by our definition, the distance between points might be infinite.%
\footnote{Let us mention an other construction to remove a possible psychological barrier with infinity --- it will not be used further.

Consider an equivalence relation ``$\approx$'' on a metric space defined as \[x\approx y\  \iff\  |x-y|<\infty.\]
Note that its equivalence classes form \emph{usual metric spaces}; 
that is $|x-y|<\infty$ for any pair of points $x$ and $y$ in the class.
It implies that metric spaces in our definition 
can be thought of disjoint union of some collection \emph{usual metric spaces} with infinite distance from each other.}


A set with a pseudometric or metric will be called \emph{pseudometric} or correspondingly \emph{metric space}.

For any pseudometric on a set 
there is an equivalence relation ``$\sim$''
such that $x\sim y$ if and only if $|x-y|=0$.
The pseudometric $|{*}-{*}|$ defines a genuine metric on the set of equivalence classes of $\sim$.
We will say that obtained metric space is \emph{defined} by the  original pseudometric space.%???IS IT A GOOD WAY TO SAY??? 

\parbf{Induced  pseudometrics.}
Let $X$ and $Y$ be metric spaces.
Given a map $f\:X\to Y$,
define a \emph{connecting pseudometric} $|{*}-{*}|_f$ on $X$ in
the following way
\[|x-y|_f=\inf\{\diam f(K)\},\]
where the infimum is taken for all connected sets $K\subset X$ which contain $x$ and $y$;
if there is no such set we set $|x-y|_f=\infty$.

The intrinsic metric induced by $|{*}-{*}|_f$ will be denoted as 
$\|{*}-{*}\|_f$. 
That is, 
\[\|x-y\|_f=\liminf_{\eps\to0+}\left\{\sum_{i=1}^n|x_i-x_{i-1}|_f\right\}.\]
where the infimum is taken for all arrays of points 
$x=x_0,x_1,\dots,x_n=y$ such that 
$|x_i-x_{i-1}|_f<\eps$ for any $i$.

The metric spaces defined by the pseudometrics $|{*}-{*}|_f$ 
and $\|{*}-{*}\|_f$ on $X$ will be denoted as $|X|_f$ and $\|X\|_f$ correspondingly.

Note that $|x-y|_f=0$ if and only if $\|x-y\|_f=0$;
that is,  both pseudometrics $|{*}-{*}|_f$ and  $\|{*}-{*}\|_f$
define the same equivalence classes on $X$.
The equivalence class of $x\in X$ will be denoted as $[x]_f$; 
it can be regarded as a point in  $|X|_f$ and in $\|X\|_f$.

Given two maps $f,h\:X\to Y$ we will write $f\succcurlyeq h$ if 
\[\|x-y\|_f\ge \|x-y\|_h\]
for any pair of points $x,y\in X$.
We will write $f\succ h$ 
if in addition the inequality is strict for at least one pair of points then.



\parbf{Metric minimizing map.}
Let $X$ and $Y$ be metric spaces and $A\subset X$ be a closed subset.

The map $f\:X\to Y$ is called \emph{metric minimizing relative to $A$}
if there is no map $h\:X\to Y$ such that $f\succ h$
and $h$ agrees with $f$ on $A$;
that is, $h|_A=f|_A$.

We say that $f\:X\to Y$ is \emph{strict metric minimizing relative to $A$}
if there is no map $h\:X\to Y$ distinct from $f$
such that $f\succcurlyeq h$, 
and $h|_A=f|_A$.

\begin{thm}{Proposition}\label{prop:point-complement}
Let $X$ and $Y$ be metric spaces 
and $A\subset X$ be a closed subset.
Assume $X$ is connected and $f\:X\to Y$ is a metric minimizing map relative to $A$.
Then for any point $x\in X$ any connected component of $X\backslash [x]_f$ intersects $A$.

\end{thm}

\parit{Proof.}
Assume contrary.
Denote by $W$ the connected component of $X\backslash [x]_f$ such that $A\cap W=\emptyset$.
Let us define the new map $h\:X\to\ Y$ 
by setting $h(z)=f(x)$ for any $z\in W$
and $h(z)=f(z)$ for any $z\notin W$.

By construction $f$ and $h$ agree on $A$ and $f\succcurlyeq h$.

Note that $\|x-y\|_f>0=\|x-y\|_h$ for any $y\in W$.
Therefore $f\succ h$, a contradiction.
\qeds

The following two propositions follow directly from the definition of metric minimizing maps.

\begin{thm}{Proposition}\label{prop:subset}
Let $X$ and $Y$ be metric spaces and $A\subset X$ be a closed subset.
Assume $f\:X\to Y$ is a metric minimizing map relative to $A$.

Given a closed subset $W\subset X$, set 
\[A_W=\partial W\cup (A\cap W).\]
Then the restriction $f|_W$ is metric minimizing relative to $A_W$.
\end{thm}

\begin{thm}{Proposition}\label{prop:factor}
Let $X, X'$ and $Y$ be metric spaces, 
$A\subset X$ be a closed subset.
Assume $f\:X\to Y$ is a metric minimizing map relative to $A$
which factors through a continuous surjective map $\phi\:X\to X'$;
that is $f=f'\circ\phi$ for a map $f'\:X'\to Y$.
Then $f'\:X'\to Y$ is metric minimizing relative to $\phi(A)$.
\end{thm}

\section{Metric minimizing graphs}



\begin{thm}{Proposition}
Assume $\Gamma$ is a  finite graph and $A$ is the collection of its vertices.
Let $Y\in\CAT[0]$ and $f\:\Gamma\to Y$ be an arbitrary map.
Then there is a metric minimizing map $f'\:\Gamma\to Y$ relative to $A$ such that
$f'|_A=f|_A$ and 
$f\succcurlyeq f'$.
\end{thm} 

\parit{Proof.} Note that there is a sequence of maps $f=f_0\succcurlyeq f_1\succcurlyeq\dots$ such that $f_n|_A=f|_A$ and there is no map $h\:\Gamma\to Y$ such that $h|_A=f|_A$ and $f_n\succ h$ for each $n$.

Indeed, the sequence $f_n$ can be chosen so that
\[\length f_n<\inf\set{\length h}{f_{n-1}\succcurlyeq h,\ h|_A=f|_A},\]
where $\length f_n$ denotes the sum of lengths of all edges of $\Gamma$. %???BETTER WORDING??? 
Since $f_n\succ h$ implies $\length f_n>\length h$, 
the condition above holds for $f_n$.

Fix an ultra-filter $\omega$ on the set of natural numbers.
Let $Y^\omega$ be the ultra-power of $Y$;
recall that $Y$ can be (and will be) considered as a subspace of $Y^\omega$.
Pass to the ultra-limit $f_\omega\:\Gamma\to Y^\omega$ 
of $f_n\:\Gamma\to Y$.
Note that $f_\omega$ is a metric minimizing map.

It remains to show that $f_\omega(\Gamma)\subset Y$.
Assume contrary, then there is a subsequence of $f_n$ which $\omega$-converges to a metric minimizing map, say $v_\omega\:\Gamma\to Y^\omega$ distinct from $f_\omega$.
Denote by $g_\omega(x)$ the midpoint of $[v_\omega(x)f_\omega(x)]$.
Note that $f_\omega\succ g_\omega$ and $g_\omega|_A=f|_A$, a contradiction.
\qeds

\begin{thm}{Proposition}\label{prop:metric-min-graph}
Let $Y$ be a $\CAT[0]$ space, 
$\Gamma$ a finite  graph and $A$ a subset of its vertices.
Assume $f\:\Gamma\to Y$ is metric minimizing relative to $A$.
Then
\begin{itemize}
\item each edge of $\Gamma$ maps to a geodesic
\item for any vertex $v\notin A$ and any $x\ne f(v)$
there is an edge  $[vw]$ in $\Gamma$ such that
$\measuredangle[f(v)^{f(w)}_x]\ge \tfrac\pi2$.
\end{itemize}
Moreover, $f$ is strictly metric minimizing relative to $A$ 
\end{thm}

\begin{wrapfigure}{r}{22 mm}
\begin{lpic}[t(-5 mm),b(-0 mm),r(0 mm),l(0 mm)]{pics/not-sufficient(1)}
%\lbl[lb]{12.5,11;$W_0$}}
\end{lpic}
\end{wrapfigure}

As one may see from the diagram,
the two conditions in the proposition do not guarantee that the map $f$ is metric minimizing.

\parit{Proof.}
By Proposition~\ref{prop:subset},
the restriction of $f$ to any edge $[vw]$ of $\Gamma$
is metric minimizing relative to $\{v,w\}$.
Hence the first condition follows.

Assume the second condition does not hold at a vertex $v\notin A$;
that is, there is a point $x\in Y$ such that
$\measuredangle[f(v)^{f(w)}_x]< \tfrac\pi2$
for any adjacent vertex $w$.
In this case moving $f(v)$ toward $x$ along $[f(v)x]$ decrease the lengths of all edges adjacent to $v$, a contradiction.

To prove the last statement, assume there is a map $f'$ distinct from $f$ such that $f|_A=f'|_A$ and $f\succcurlyeq f'$.
Denote by $g(x)$ the midpoint of $f(x)$ and $f'(x)$ for any $x\in \Gamma$. 
By comparison $f\succcurlyeq g$.
It follows that the tautological map $\|\Gamma\|_f\to \|\Gamma\|_g$ is an isometry.
The later implies that the distance $|f(v)-g(v)|$ is the same for all the vertices $v$ in $\Gamma$.
Since we have $|f(v)-g(v)|=0$ for any $v\in A$,
we get $f(v)=g(v)$ for any vertex $v$ in $\Gamma$.
Hence $f=f'$, a contradiction.
\qeds

Assume $\Gamma$ is a finite graph embedded in the plane,
in particular $\Gamma$ is planar.

The complement to the unbounded connected component of $\RR^2\backslash\Gamma$ will be called filling of $\Gamma$;
it will be denoted as $\Fill\Gamma$.

The vertex of $\Gamma$ will be called \emph{boundary vertex}
if it lies in the boundary $\partial\Fill\Gamma$,
otherwise it will be called \emph{interior vertex}.

\begin{thm}{Corollary}\label{cor:planar-minimizing-graph}
Let $Y$ be a $\CAT[0]$ space and
$\Gamma$ an embedded graph in $\RR^2$.
Assume $f\:\Gamma\to Y$ is a metric minimizing map relative to the boundary vertices. 
Then 
one can equip $\Fill\Gamma$ with a $\CAT[0]$ pseudometric 
and extend $f$ to a short map $\bar f\:\Fill\Gamma\to Y$ which is length preserving on $\Gamma$.
\end{thm}

\parit{Proof.}
Fix a cycle $\gamma$ in $\Gamma$ which bounds one of the discs in the complement $\RR^2\backslash \Gamma$.
Set $\ell=\length\gamma$.

By Reshetnyak's majorization theorem, there is a convex polygon $P$ (possibly degenerate) with perimeter $\ell$ which admits a short map to $Y$ in such a way that $\gamma$ is formed by the image of the boundary.
Note that each angle of $P$ is at least as big as 
the angle between the corresponding edges.

Prepare a polygon as above for each disc in the complement of $\Gamma$
and glue these polygons into $|\Gamma|_f$ along the natural map.
The obtained space $D$ is simply connected.
Therefore in order to show that $D$ is $\CAT[0]$,
we need to check that the sum of the angles around each interior vertex in $\Gamma$ is at least $2\cdot\pi$.

Fix an interior vertex $v$.
Assume that the sum of the angles around $v$ is less than $2\cdot\pi$.
The space of directions $\Sigma_{f(v)}$ is a $\CAT[1]$ space.
The directions of the edges from $v$ have a natural
cyclic order say $\xi_1,\dots,\xi_k$
such that
\[\measuredangle(\xi_1,\xi_2)+\dots+\measuredangle(\xi_k,\xi_1)<2\cdot\pi.\]
By Reshetnyak's majorization theorem,
the closed broken line $\xi_1,\dots,\xi_k$ is majorized by a convex spherical polygon $P$.
Note that $P$ lies in an open hemisphere with the pole  at some point in $P$.
Choose $x\in Y$ so that the direction form $f(v)$ to $x$ coinsides with the image of the pole in $\Sigma_{f(v)}$.
This choice of $x$ contradicts the condition in Proposition~\ref{prop:metric-min-graph}.\qeds







\section{Metric minimizing discs}



Let us denote by $\DD$ the closed unit disc in the plane,
its boundary $\partial \DD$ is a unit circle.

Let $X$ be a Hausdorff space and
$f\:\DD\to X$ be a continuous map.
We say that $f$ is a \emph{no-bubble map}
if for any point $p\in X$ every connected component of the complement $\DD\backslash f^{-1}\{p\}$ has points in the boundary $\partial \DD$.


We will need the following disc version of Moore's theorem proved in \cite{moore}.

\begin{thm}{Proposition}\label{prop:disc-moore}
Let $X$ be a Hausdorff space and
$f\:\DD\to X$ be a no-bubble map.
Then $|\DD|_f$ is homeomorphic to a \emph{disc retruct};
that is, $|\DD|_f$ is a compact simply connected space which admits an embedding into the plane.
\end{thm}


\parit{Proof.}
Note that $|\DD|_f$ is a Hausdorff space.
and the forgetful map $h\:\DD\to |\DD|_f$ is continuous.
Since and $\DD$ is compact, the space $|\DD|_f$ comes with the quotient topology for $h$. 

Consider the disc $\DD$ as a subset in the plane $\SS^2$.
For any $x\in \DD$, set 
\[[x]=h^{-1}\{h(x)\}\]
and
if $x\in \SS^2\backslash \DD$ we assume $[x]=\{x\}$.
Denote by $\SS^2_f$ the set of classes $[x]$ with the quotient topology induced by the map $\iota_f\:x\mapsto [x]$.

Note that for any point $x$, the set $[x]$ is connected and compact, with connected complement.
Therefore by Moore's theorem \cite{moore},
$\SS^2_f$ is homeomorphic to $\SS^2$.

It remains to note that $|\DD|_f$ is embedded into $\SS^2_f$ and its complement is homeomorphic to an open disc.\qeds

Given a disc retract $X$,
define its interior as a maximal open set which is homeomorphic to an open set in the plane.
By Domain Invariance Theorem, interior is well defined.
The complement to the interior of $X$ will be called boundary and denoted as $\partial X$.

Applying \ref{prop:point-complement} and \ref{prop:disc-moore}, we get the following.

\begin{thm}{Proposition}\label{prop:|D|}
Let $f\:\DD\to\ Y$ be a metric minimizing map relative to $\partial \DD$.
Then $f$ is a no-buble map.

In particular,  $|\DD|_f$ is homeomorphic to a disc retract.
Moreover $\partial|\DD|_f$ is the image of $\partial \DD$
under the tautological map $x\mapsto [x]_f$.
\end{thm}

Note that the metric minimizing map $f\:\:\DD\to\ Y$ factors through a map $f'\:|\DD|_f\to Y$
and by \ref{prop:factor} and \ref{prop:|D|} 
the map $f'$
is metric minimizing relative to $\partial|\DD|_f$.

\begin{thm}{Proposition}
Let $Y\in\CAT(0)$ 
and 
$f\:\DD\to Y$ be a metric minimizing disc.
Then $\|\DD\|_f$ is a complete metric space.
\end{thm}

\parit{Proof.}
Denote by $W$ be completion of $\|\DD\|_f$.

By Proposition \ref{prop:|D|},
the space $|\DD|_f$ is compact.
Therefore the tautological map $f\:\|\DD\|_f\to |\DD|_f$ 
extends to a continuous map $\bar f\:W\to |\DD|_f$.

If $\|\DD\|_f$ is a proper subset of $W$ 
then from above we get that
then $\bar f$ is not injective.

In this case there is a rectifiable path $\gamma\:[0,1]\to W$,
such that $\gamma(t)\in\|\DD\|_f$ for all $t<1$ and $\gamma(1)\notin \|\DD\|_f$.
Note that the path $\bar f\circ \gamma$ is also rectifiable and in particular continuous.
???
\qeds








\section{Key Lemma}


\begin{thm}{Key Lemma}\label{lem:key}
Let $Y$ be a $\CAT(0)$ space and $s\:\DD\to Y$ 
be a metric minimizing disc relative to the boundary $\partial \DD$.
Given a finite set $F\subset \DD$
there is a $\CAT(0)$ space $W$, which is homeomorphic to a compact subset of the plane,
and maps $p\:F\to W$ and $q\:W\to Y$ such that
\[s(x)=q\circ p(x)\] 
for any $x\in F\cap \partial \DD$
and 
\[\|p(x)-p(y)\|_q\le \|x-y\|_s\] 
for any $x,y\in F$.
\end{thm}

\parit{Proof.}
Let us connect each pair $x,y$ of points in $F$ by geodesics
if $\|x-y\|_s<\infty$.

We can assume that 
every pair of the constructed geodesics 
are either disjoint, or their intersection is formed by finite collections of arcs and points.

Indeed, assuming some number of geodesics $\gamma_1,\dots\gamma_n$ which meets the above property is already chosen and we need to choose one more geodesic connecting $x$ to $y$.
Choose a minimizing geodesic $\gamma_{n+1}$ which maximize the time it spends in $\gamma_1,\dots\gamma_n$  in the order of importance.
Namely, 
\begin{itemize}
\item  among all minimizing geodesics connecting $x$ to $y$
choose one which spends maximal time in $\gamma_1$ --- in this case $\gamma_{n+1}$ intesects $\gamma_1$ along the empty set, one-point set or a closed arc.
\item among all minimizing geodesics as above
choose one which spends maximal time in $\gamma_2$ --- in this case $\gamma_{n+1}$ intesects $\gamma_2$ along at most two arcs and points.
\item and so on.
\end{itemize}

%IT SHOULD BE POSSIBLE TO ENSURE THAT INTERSECTION OF ANY TWO GEODESIC IS EMPTY OR CONNECTED, BUT I FAILED TO MAKE IT FORMALLY???

In particular the set of all these geodesics forms a finite graph, say $\Gamma$,
embedded in $|\DD|_s$. 

According to Proposition~\ref{prop:|D|},
$|\DD|_s$ admits an embedding into the plane.
Therefore $\Gamma$ can be considered as a graph embedded into the plane.

It remains to apply Corollary~\ref{cor:planar-minimizing-graph}.
\qeds

\section{Compactness of planar CAT(0) spaces}

We recall some terminology. Let $X$ be a connected topological space. A point $c\in X$
is called a {\em cut point}, if $X\setminus \{c\}$ is disconnected. The set of cut points in $X$ 
will be denoted by $cut(X)$. If $X$ is connected and $x,y\in X$ are such that no cut point in $X$
seperates $x$ and $y$, then they are called {\em conjugate}. For a non-cut point $p\in X$
we define $conj(p)$ as the set of all points conjugate to $p$. A point $e\in X$ is called an {\em endpoint},
if $conj(e)=\{e\}$. The conjugacy set $conj(p)$ for non-endpoints
are called {\em cyclic elements}. Every cyclic element is a closed set\cite[IV(1.4)]{W2}. If $X$ is a 
Peano continuum\footnote{A {\em Peano continuum} is a compact
connected and locally connected metric space.}, then every cyclic element $conj(p)$ is {\em cyclically connected},\ i.e.
any two points $x,y\in conj(p)$ lie in a Jordan curve $J\subset conj(p)$\cite[III(3.32)]{Wi}. Any two different 
cyclic elements $conj(p)$ and $conj(q)$ intersect at most in a single point. Any point common to two 
different cyclic elements is a cut point\cite[IV(1.4)]{W2}.

\begin{thm}{Lemma}
 Let $X$ be a CAT(0) space. The conjugacy set for any point $p\in X$ is convex, and it is given by
 $$
conj(p):=\{q\in X|\ (pq)\cap cut(p)=\emptyset \}.
$$
\end{thm}

\parit{Proof.}
If two points are separated by a cut point, then any path connecting those points has to go through the cut point.
Hence the particular representation follows. Let $x,y\in conj(p)$ and $z\in cut(X)\setminus\{x,y\}$. Then a dense subset
of the geodesic cone from $p$ over $[xy]$ is contained in the same component of $X\setminus \{z\}$ as $p$. 
The claim follows since $conj(p)$ is closed.
\qeds

If $C$ is a locally connected continuum such that 
each cyclic element in $C$ is homeomorphic to a circle, then $C$ is
called a {\em boundary curve}. A {\em base set} is a locally connected continuum which
embeds into the plane $\mathbb{R}^2$ such that its image is not separating. A locally connected
continuum $B$ is a base set if and only if every cyclic element in $B$ is homeomorphic to a disc,
whose interior is open in $B$. If we remove the interiors of all cyclic elements in $B$ the remaining
space is a boundary curve $C$. In this case we say that $B$ is bounded by $C$.


\begin{thm}{Lemma}
 Let $X$ be a compact CAT(0) space which embeds into $\mathbb{R}^2$. Then $X$
 is homeomorphic to a base set.
\end{thm}

\parit{Proof.}
Clearly $X$ is connected and locally connected. Let $\iota :X\hookrightarrow\mathbb{R}^2$ be an embedding.
If $\iota(X)$ separates $\mathbb{R}^2$, then it would contain a Jordan curve $J$ which still separates
$\mathbb{R}^2$. This is impossible since $J$ is contractible in $X$.
\qeds

We need Whyburn's notion of {\em regular convergence}\footnote{Our definition of regular convergence coincides with Whyburn's regular convergence relative
to 0- and 1-cycles.}. A sequence of compact metric spaces $(K_n,d_n)$ converges regularly to a 
metric space $(K,d)$, if they converge with respect to Gromov--Hausdorff distance and the following two conditions are fulfilled:
\begin{itemize}
 \item[(i)] For any sequence of pairs of points $x_n,y_n\in K_n$ with $d_n(x_n,y_n)\to 0$, there is a sequence of paths $\gamma_n$
 connecting $x_n$ to $y_n$ such that $\diam(\gamma_n)\to 0$.
 \item[(ii)] For any sequence of Jordan curves $J_n\subset K_n$ with $\diam(J_n)\to 0$, there is a sequence of continuous discs $D_n\subset K_n$
 with $\partial D_n=J_n$ and $\diam D_n\to 0$. 
\end{itemize}

Note that if all spaces $K_n$ are geodesic, then condition (i) is automatically fulfilled. Moreover, if all 
spaces $K_n$ are CAT(0), then condition (ii) is fulfilled as well because any Jordan curve of diameter $a$ in a CAT(0)
space is contractible in a ball of radius $a$. Hence any Gromov--Hausdorff converging sequence of CAT(0) spaces is 
regularly convergent.

Let $\mathcal{K}_\ell$ be the set of isometry classes of $\CAT(0)$ metrics on base sets with rectifiable
boundary curves of length at most $\ell$.



\begin{thm}{Lemma}
$\mathcal{K}_\ell$ is precompact in the Gromov--Hausdorff topology.
\end{thm}

Further $\area K$ denotes the two-dimensional Hausdorff measure of a metric space $K$. 

\parit{Proof.}
Let $K$ be a metric space with isometry class in $\mathcal {K}_\ell$.
By Reshetnyak's theorem there is a short map from a convex plane figure $F$ with perimeter at most $\ell$ onto $K$.
In particular, $\area K \le \area F 
\le \ell^2$.

Fix $\eps>0$. 
Set $m=\lceil 10\cdot\tfrac\ell\eps\rceil$.
Choose $m$ points $y_1,\dots,y_m$ on $\partial D$
which divide $\partial D$ into arcs of equal length.

Consider the maximal set of points $\{x_1,\dots,x_n\}$ such that $d(x_i,x_j)>\eps$ and $d(x_i,y_j)>\eps$.

Note that the set $\{x_1,\dots,x_n,y_1,\dots,y_m\}$
forms an $\eps$-net in $(\DD,d)$.

Further note that the balls $B_i=B_{\eps/2}(x_i)$
do not overlap.
By comparison,
\[\area B_i\ge \tfrac{\pi\cdot\eps^2}{4}.\]

It follows that $n\le 2\cdot\left(\tfrac\ell\eps\right)^2$.
That is, there is a function $N(\eps)$,
which returns a positive integer for any $\eps>0$
such that for any 
$(\DD,d)$ contains an $\eps$-net
with at most $N(\eps)$ points.

In other words, $\mathcal{K}_\ell$ is uniformly totally bounded.
Any class of metrics with such property is precompact in Gromov--Hausdorff topology; 
see for example \cite[7.4.15]{BBI}.
\qeds





\begin{thm}{Lemma}
$\mathcal{K}_\ell$ is closed in the Gromov--Hausdorff topology.
\end{thm}
\parit{Proof.}
Let $K_n\to K$ be a Gromov--Hausdorff convergent sequence  where every $K_n$ represents an element in  $\mathcal{K}_\ell$. Then
$K$ is a compact CAT(0) space. Let $p\in K$ be a point which is neither a cut point not an end point. We need to show that the cyclic 
element $conj(p)$ is homeomorphic to a disc. Let $J\subset conj(p)$ be a Jordan curve with $p\in J$. Choose Jordan curves $J_n\subset K_n$
with $J_n\to K_n$ \cite[2.2]{W1}. Then there is a unique cyclic element $C_n\subset K_n$ such that $J_n\subset C_n$. By assumption $C_n$ is homeomorphic
to a disc. The limit $C_\infty:=\lim C_n$ is a convex subset of $K$ containing $J$. By \cite[6.3]{W1}, $C_\infty$ is a base set and therefore contains a
unique cyclic element $D_\infty$ such that $J\subset D_\infty$. Since $D_\infty$ is homeomorphic to a disc, it is clear that $D_\infty\subset conj(p)$.

{\it Claim:} $D_\infty=conj(p)$.

\noindent If this is not the case, then there is an element $q\in conj(p)\setminus D_\infty$. Choose a point $p'\in D_\infty\setminus J$ which is surrounded 
by $J$. Using the cyclic connectedness we find a Jordan curve $J'\subset conj(p)$ such that $p',q\in J'$. Approximate $J'$
by Jordan curves $J_n'\subset K_n$ and the point $p'$ by points $p_n'\in K_n$. Since $p_n'$ is then surrounded by the $J_n'$, the intersection
$J_n'\cap J_n$ contains at least two points. Consequently $J_n'$ lies in the same cyclic element as $J_n$,\ i.e. $J_n'\subset C_n$. As
above we conclude that there is a unique cyclic element $D_\infty'\subset C_\infty$ such that $J'\subset D_\infty'$. But 
$J'$ intersects the interior of $D_\infty$ in the point $p'$. Hence $D_\infty=D_\infty'$, a contradiction.
\qeds

Combining the last two statements we arrive at:

\begin{thm}{Lemma on compactness}\label{lem:compact}
$\mathcal{K}_\ell$ is compact in the Gromov--Hausdorff topology.
\end{thm}

\section{Main theorem}

\begin{thm}{Main Theorem}
Let $Y$ be a $\CAT[0]$ space 
and $s\:\DD\to Y$ be a metric minimizing map relative to the boundary $\partial\DD$.
Assume $\|\DD\|_s$  is separable. 
Then $\|\DD\|_s$ is a $\CAT[0]$ space.
\end{thm}


\parit{Proof of Main theorem.}
Note that it is sufficient to prove the theorem 
in the case that $\partial |\DD|_f$ formed by a rectifiable simple closed curve.

Indeed, fix a triangle $\triangle$ in $\|\DD\|_f$.
If the above case has been proven, 
then closure of each connected open coponent bounded by $\triangle$ is $\CAT[0]$.
In particular $\triangle$ is thin.
Since $\triangle$ is arbitrary, the statement follows.

Given a finite set $F\subset \DD$,
denote by $\mathcal{W}_F$
the set of isometry classes of spaces $W$ which meet the conditions of the Key Lemma~\ref{lem:key}
for $F$.
Note that for two finite sets $F\subset F'$ in $\DD$,
we have $\mathcal{W}_F\supset \mathcal{W}_{F'}$.

According to Lemma on compactness (\ref{lem:compact}) $\mathcal{W}_F$ is compact.
Therefore 
\[\mathcal{W}
=
\bigcap_{F}\mathcal{W}_F\ne \emptyset\]
where the intersection is taken over all finite subsets $F$ in $\DD$.

Fix a space $W$ from $\mathcal{W}$
and a dense sequence of points $\{x_1,x_2,\dots\}$ in $\DD$,
such that the subsequence of points in $\partial \DD$
is also dense in $\partial \DD$.
Set $F_n=\{x_1,\dots,x_n\}$.
Denote by $p_n\:F_n\to W$ a map satisfying the conditions in the Key Lemma~\ref{lem:key}.

Passing to a subsequence by $n$ we can ensure that the sequence
$p_n(x_k)$ converges as $n\to\infty$ for every fixed $k$.
Set 
\[p(x_k)=\lim_{n\to\infty} p_n(x_k).\]

Note that $p$ is short.
Since $\{x_k\}$ is dense in $\DD$,
the map $p$ can be extended to whole $\DD$ 
as a short map.
The obtained map will be still denoted as $p$.

Summarizing, the space $W$ is $\CAT[0]$ space 
homeomorphic to a planar continuum.
By construction, therere are two short maps 
$\|\DD\|_s\xrightarrow{p} W \xrightarrow{q} Y$
such that 
\[q\circ p|_{\partial\DD}=s|_{\partial\DD}.\]
Since $s$ is metric minimizing, both maps $p$ and $q$ are length preserving.

It remains to show that $p$ is injective, it is similar to the the problem ``Saddle surface'' in \cite{petrunin-orthodox}.

Assume  $w=p(x)=p(y)$ for distinct points $x,y\in\|\DD\|_s$
Note that  $w$ lies in the interior of $W$.
Choose a geodesic $\gamma$ which pass through $w$ and goes 
from boundary to boundary of $W$.
The inverse image $p^{-1}(\gamma)$ is a contractabe set with two ends at $\partial\|\DD\|_s$, say $a$ and $b$.
It follows that $p^{-1}(\gamma)$ can be arbitrary well approximated by a simple curve from $a$ to $b$. 
In particular the there is well defined order of  points on $p^{-1}(\gamma)$.
We can assume that the points $a,x,y,b$ appear in the same order on $p^{-1}(\gamma)$. 

Note that there is a continuous one parameter family of geodesics $\gamma_t$ passing through $w$ with the ends at $\partial W$
such that $\gamma=\gamma_0$ and $\gamma_1$ is $\gamma$ with reversed parametrization.
Note that the order of $x$ and $y$ on $p^{-1}(\gamma_t)$ does not change in $t$.
On the other hand the order on $\gamma_0$ and on $\gamma_1$ are opposite, a contradiction.
\qeds

\section{Energy of Sobolev maps}


Let $\Omega$ be an open and bounded subset of $\mathbb{R}^n$ and $(X,d)$ a metric space.
We say that a measurable map $u:\Omega\to X$ lies in 
$L^2(\Omega,X)$ if and only if
\begin{enumerate}
 \item there is a full measure subset $M\subset \Omega$ such that $u(M)\subset X$ is separable;
 \item for any $x\in X$ the map $z\mapsto d(u(z),x)$ belongs to $L^2(\Omega)$.
\end{enumerate}

Following Reshetnyak \cite{R}, we define the Sobolev space $W^{1,2}(\Omega,X)$, of finite energy maps, as follows. 
A map $u\in L^2(\Omega,X)$ is contained in $W^{1,2}(\Omega,X)$ if and only if

\begin{enumerate}
 \item for any $x\in X$ the map $u_x(z):=d(u(z),x)$ belongs to $W^{1,2}(\Omega)$;
 \item there exists a uniform bound for the weak gradients $|\nabla u_x|$,\ i.e. there is a function $h\in L^2(\Omega)$ 
 such that $|\nabla u_x|\leq h$ holds for all $x$ almost everywhere in $\Omega$. 
\end{enumerate}

If for a map $u:\Omega\to X$ the  approximate limit
$$
ap\lim_{t\searrow 0}\frac{d(u(z+tv),u(z))}{t}
$$
exists, then we call this limit the {\em approximate metric derivative} at the point $z$ in the direction $v\in\mathbb{R}^n$. It will be denoted
as $|du_z(v)|$. According to
\cite[Prop.4.3]{LW}, if $u\in W^{1,2}(\Omega,X)$, then its approximate metric derivative exists and defines a semi-norm $|du_z|$ on $\mathbb{R}^n$, 
in every direction at almost every point $z\in\Omega$. Moreover, the function $z\to\int_{S^1}|du_z(v)|^2 d\mathcal{L}^1$ lies in $L^2(\Omega)$.
We can then define the {\em energy} of a map $u\in W^{1,2}(\Omega,X)$ as
$$
E(u):=\int_\Omega\left(\frac{1}{\omega_n}\int_{S^{n-1}}|du_z(v)|^2 d\mathcal{L}^1(v)\right)d\mathcal{H}^{n-1}(z).
$$
It was shown in \cite{LW} that this definition coincides, up to a factor, with the Korevaar-Schoen 2-energy \cite{KS}.

From now on, we will assume that $X$ is a CAT(0) space. Recall that between any two maps $u_0,u_1:\Omega\to X$ there is 
a unique {\em geodesic homotopy} $u_t$. Namely for $t\in[0,1]$ we define $u_t(z)$ as the point on $[u_0(z)u_1(z)]$ at distance $t$
from $u_0(z)$. If $u_0, u_1\in W^{1,2}(\Omega,X)$, then energy is convex along geodesic homotopies \cite[(2.2vi)]{KS}:
$$
E(u_t)\leq (1-t)E(u_0)+tE(u_1)-t(1-t)\int_\Omega|\nabla d(u_0,u_1)|^2d\mathcal{H}^2.
$$


\section{Area preserving short maps between CAT(0) surfaces}

Let $X$ be a CAT(0) disc. For any subset $A\subset X$ we denote $N_\epsilon (A)$ its open $\epsilon$-neighbourhood.
Let $x,y\in X$ be different points and $[xy]$ the geodesic segement between them. We define an $\epsilon$-half-neighbourhood
$N^+_\epsilon ([xy])$ in the following way. Note first that $[xy]$ seperates the ball $B$ of radius $\frac{|x-y|}{2}$ around
the midpoint of $[xy]$ into two components $B^+$ and $B^-$. Let $f$ be the distance function to $[xy]$. Choose gradient curves 
$\gamma_x$ and $\gamma_y$ of $f$ starting in $x$ respectively $y$ and such that they are tangential to $B^+$. The concatenation
of $\gamma_x$, $[xy]$ and $\gamma_y$ seperates $N_\epsilon([xy])$ into two components and we choose $N^+_\epsilon ([xy])$ to be the one
with interior angles equal to  $\frac{\pi}{2}$. The curves $\gamma_x$ and $\gamma_y$ will be called {\em boundary curves} of the 
half-neighbourhood.


For $t<\epsilon$ set $\gamma_t:=\partial N_t([xy])\cap N^+_\epsilon ([xy])$. If $\epsilon$
is small enough, then $\gamma_t$ defines a rectifiable simple arc.


\begin{thm}{Proposition}\label{prop:length continuity}
The length of $\gamma_t$ converges to $|x-y|$ for $t\to 0$.
\end{thm}

In order to prove this, we need a little preparation. For $\epsilon>0$ we denote
$X_\epsilon$ the subset of points in $X$ at distance $>\epsilon$ from the boundary.

\begin{thm}{Lemma}\label{lem:extension}
For any geodesic segment $[xy]$ of length $L>0$ in $X_\epsilon$ there is a point 
$x_y\in X$ such that $[x x_y]$ extends $[xy]$ and has length $L+\epsilon$.
\end{thm}
\parit{Proof.}
We use an argument from \cite{Kleiner}. Note that there exists a radius $r\geq\epsilon$, such 
that the radius $r$ distance sphere around $y$ contains a Jordan curve $J$. The Jordan domain bounded by $J$
will be called $D_J$.
Denote $cone_x(J)$ the geodesic cone over $J$ with respect to $x$. Then, 
for any small $\delta>0$, $cone_x(J)$ and $D_J$
represent the same element in $H_2(X,N_\delta(J))$. Hence $D_J\subset cone_x(J)$. 
In particular $y\in cone_x(J)$.
\qeds

A {\em geodesic quadrangle} $\square$ consits of an ordered tuple of four pairwise 
distinct points $x_k\in X$, $k\in\mathbb{Z}_4$, 
called vertices, and the four geodesic segments $[x_k,x_{k+1}]$, called the sides. 
If no side is separating,\ i.e. no maximal extension of that side as a geodesic 
separates the other two vertices, $\square$ is said to be {\em regular}.

\begin{thm}{Lemma}\label{lem:large angles}
Let $\epsilon>0$ be fixed. For some $\delta>0$ let $\square_\delta$ be a regular geodesic quadrangle with
vertices $x,y,u$ and $v$ in $X_\epsilon$ such that
\begin{enumerate}
	\item $\angle_x(u,y)=\angle_y(v,x_y)=\frac{\pi}{2};$
	\item $|x-y|\leq\delta;$
	\item $|x-u|\geq 4\delta\text{ and }|y-v|\geq 4\delta;$
	\item there is a point $m\in [uv]$ with $|m-[xy]|\leq \delta.$
\end{enumerate}
Then 
\begin{itemize}
\item $\angle_u(x_u,v)\geq\frac{2\pi}{3};$
\item $\angle_v(u_v,y)\geq\frac{2\pi}{3}.$
\end{itemize}
\end{thm}

\parit{Proof.}
We will use the notation for extensions of geodesics as in Lemma \ref{lem:extension}.
Let $f$ be the distance function to $[x x_y]$ restricted to $[u u_v]$. Then we have
$$
f'(u)\leq\frac{f(m)-f(u)}{|m-u|}\leq-\frac{1}{2}\text{  and  }f'(v)\geq\frac{f(v)-f(m)}{|v-m|}\geq\frac{1}{2}.
$$
By condition (1) the claim now follows from the first variation formula.
\qeds

\begin{thm}{Corollary}
Let $\epsilon>0$ be fixed. For a sequence $\delta_k\to 0$ let $\square_k$ be a converging sequence of quadrangles
as in Lemma \ref{lem:large angles} with $\square_k\to \{z\}$. Denote by $x^\infty_u, u^\infty_v$ and $v^\infty_y$
the limits of $\epsilon$-extensions of the respective sides of $\square_k$. Then
\begin{itemize}
\item $\angle_z(x^\infty_u,u^\infty_v)\geq\frac{2\pi}{3};$
\item $\angle_z(u^\infty_v,v^\infty_y)\geq\frac{2\pi}{3};$
\item $\angle_z(v^\infty_y,x^\infty_u)\geq\pi.$
\end{itemize}
In particular $\mathcal{H}^1(\Sigma_z(X))\geq\frac{7}{3}\pi$.
\end{thm}
\parit{Proof.}
The inequalities follow from semi-continuity of angles together with Lemma \ref{lem:large angles}.
For the supplement note that $[x^\infty_u u^\infty_x]$ induces a decomposition of $\Sigma_z X$ into two convex subset, both of length at least $\pi$.
But as a result of the first two inequlities, one of those convex sets has length at least $\frac{4}{3}\pi$, hence the claim.
\qeds

\begin{thm}{Corollary}\label{cor:level distance 1}
For every $\epsilon>0$ there exists $\bar\delta>0$ such that for any nondegenerated geodesic $[xy]$ 
in $X_\epsilon$ with $\mathcal{H}^1(\Sigma_z X)< \frac{7}{3}\pi$
for every $z$ in $[xy]$, the following holds for every $\delta\leq\bar\delta$. 
If $[x'y']\subset[xy]$ is a subgeodesic of length $\leq\delta$ and $N^+_r([xy])$ is some half-neighbourhood 
of $[x'y']$ with boundary curves $\gamma_{x'}$ and $\gamma_{y'}$, 
then the geodesic $c_\delta$ between  $\gamma_{x'}(4\delta)$ and $\gamma_{y'}(4\delta)$ has distance $>\delta$
from $[x'y']$.
\end{thm}

\begin{thm}{Corollary}\label{cor:level distance 2}
Let $p$ be a point in the interior of $X$ and $(c,c^\perp)$ a {\em rightangled hinge} at $p$, \ i.e. $c,c^\perp:[0,T]\to X$ are unit speed geodesics starting in $p$
with $\angle_p(\dot c(0),\dot c^\perp (0))=\frac{\pi}{2}$. For every $t\in[0,T]$ let
$c^\perp_t$ be a gradient curve of the distance function to $c$ starting in $c(t)$ and lying on the same side
of $c$ as $c^\perp$. Then there exists $\hat\delta>0$ such that for all $\delta\leq\hat\delta$ the distance between 
$[c^\perp(4\delta)c^\perp_\delta(4\delta)]$
and $c$ is larger than $\delta$.
\end{thm}








\parit{Proof of Proposition\ref{prop:length continuity}.}
Let $[xy]$ be a geodesic in $X_\epsilon$ for some $\epsilon>0$. Call a point
$z\in X$ with $\mathcal{H}^1(\Sigma_z X)\geq \frac{7}{3}\pi$ a {\em fat point}. 
Note that there is only a finite number of 
fat points in $X$. We will prove the statement by induction on  the number 
of fat points lying on $[xy]$. If there is no such point, then we choose $\bar\delta>0$ 
as in Corollary \ref{cor:level distance 1} and a point array
$x=x_0,x_1,\ldots,x_n=y$ on $[xy]$ such that $|x_j-x_{j+1}|=\delta\leq\bar\delta$. Let $N_r^ +([x_j x_{j+1}])$ 
be a half-neighbourhood contained in $N_r^ +([x y])$
with boundary curves $\gamma_{x_j^ +}$ and $\gamma_{x_{j+1}^ -}$. Then by Corollary \ref{cor:level distance 1}, 
the geodesic between $\gamma_j^ -(4\delta)$
and $\gamma_{j+1}^ +(4\delta)$ has distance more than $\delta$ from $[xy]$. We connect the point 
$\gamma_j^ +(4\delta)$ to $\gamma_j^ -(4\delta)$ by a shortest path $\rho_j$
contained in $\partial B_{4\delta}(x_j)$. Note that the length $L_j$ of $\rho_j$ is $O(\delta)$. 
Moreover, since $X$
is a CAT(0) disc, the contribution of single points to the total curvature of $X$ is finite. Hence 
even $\sum_{j=0}^ n L_j$ is $O(\delta)$.
Therefore, since half-neighbourhoods are convex and projections onto convex subsets are short, the 
length $L_\delta$ of $\gamma_\delta$ can be
estimated by 
$$
L_\delta\leq |x-y|+n\cdot o(\delta)+O(\delta).
$$
Since $|x-y|=n\delta$, the claim follows in this case.

For the general case, we choose $s\in[xy]$ to be the fat point closest to $x$. We assume that $s\neq x$,
the other case is similar. Let $\gamma_s^\pm\subset N_r^+([x y])$ be geodesics starting in $s$ and such that 
$\angle_s(x,\gamma_s^-(r))=\frac{\pi}{2}$ and $\angle_s(y,\gamma_s^+(r))=\frac{\pi}{2}$. Applying Corollary 
\ref{cor:level distance 2}
to the rightangled hinges $([xs],\gamma_s^-)$ and $([sy],\gamma_s^+)$ we obtain constants $\hat\delta^\pm>0$. 
Set $\hat\delta:=\min\{\hat\delta^-,\hat\delta^+\}$.
Next choose $s^-\in[xs]$ and $s^+\in[sy]$ with $|s-s^\pm|\leq\hat\delta$. Then the induction hypothesis applies 
to $[xs^-]$ and $[s^+y]$ since we have reduced 
the number of fat points. If $N^+_r([s^- s^+])$ is a half-neighbourhood with boundary curves $\gamma_{s^\pm}$, 
then for $\delta\leq\hat\delta$ we can estimate the length of $\gamma_\delta$ as before. 
Namely choose a shortest path $\rho$ in $\partial B_{4\delta}(s)$ from $\gamma_s^-(4\delta)$ to $\gamma_s^+(4\delta)$ 
and concatenate it 
with $[\gamma_{s^-}(4\delta)\gamma_s^-(4\delta)]$ and $[\gamma_s^+(4\delta)\gamma_{s^+}(4\delta)]$. The same estimate 
as above applies now because, by Corollary \ref{cor:level distance 2}, this path has distance $>\delta$ from $[s^- s^+]$.
\qeds






\begin{thm}{Theorem}\label{thm:short+area=isom}
Let $X$ be a 2-rectifiable space and $Y$ a planar CAT(0) space. 
Let $\varphi:X\to Y$ be a surjective short map. If $\area(X)=\area(Y)$,
then $\varphi$ is an isometry. 
\end{thm}

\parit{Proof.}
Recall that each cyclic element of $Y$ is homeomorphic to a closed disc. 
Pick points $p,q\in Y$ which ly in the interior of the same cyclic element $C$. 
Then choose a half-neighbourhood $N_r^+$ of $[pq]$ which is still contained
in the interior of $C$. Denote $f$ the distance function of $[pq]$.
Let $\hat N_r^+$ be the inverse image of $N_r^+$
under $\varphi$. 
By the coarea formula we obtain
$$
\area(N_r^+)=\int_0^\delta \mathcal{H}^1(f^ {-1}(t))dt
$$
and
$$
\area(\hat N_r^+)\geq\int_{\hat N_r^+}|\nabla (f\circ\varphi)|=\int_0^\delta \mathcal{H}^1((f\circ\varphi)^ {-1}(t))dt.
$$
Hence $\mathcal{H}^1((f\circ\varphi)^ {-1}(t))\leq \mathcal{H}^1(f^ {-1}(t))$ for almost all $t\leq r$.
Since for almost all $t\leq r$ $\mathcal{H}^1((f\circ\varphi)^ {-1}(t))$ contains a full measure subset which is a disjoint union of paths ....
 
\qeds











\section{Existence and uniqueness}

Let $\gamma_{\pm 1}:I\to X$ be two rectifiable paths, parametrized by arc length. We say that  $\gamma_{-1}$ is {\em parallel} to $\gamma_{+1}$, if $d(\gamma_{-1},\gamma_{+1})$
is constant on $I$.


\begin{thm}{Lemma}
Let $\gamma_{\pm 1}:I\to X$ be two rectifiable paths of equal length $L>0$. Set $\gamma_t:=h(\cdot,t)$ where $h:[-1,1]\times[0,L]\to X$ denotes the geodesic homotopy
between their arclength parametrizations. If $\gamma_0$ has length $L$, then $\gamma_{-1}$ and $\gamma_{+1}$ are parallel. Moreover, $h([-1,1]\times[0,L])$ is intrinsically 
flat.
\end{thm}
\parit{Proof.}
Convexity of $d$ implies $|\dot\gamma_0(t)|\leq 1$. By assumtion, length of $\gamma_0$ equals $L$, hence $|\dot\gamma_0(t)|=1$ for almost all $t\in[0,L]$. So $\gamma_0$
is parametrized by arc length. Since the energy of a unit speed path equals its length, we conclude from energy convexity 
$$
\int_0^L|\nabla d(\gamma_{-1},\gamma_{+1})|^2 d\mathcal{H}^1=0.
$$
Therefore, $\gamma_{-1}$ is parallel to $\gamma_{+1}$.

For the second claim, denote $H$ the surface $h([-1,1]\times[0,L])$ equipped with the pull-back metric. By \ref{main}, $H$
is CAT(0).
Now let $\epsilon>0$ and choose an array of points $x_0^{0},x_1^{0},\ldots,x_k^{0}$ on $\gamma_{0}$ 
such that $\sum_{i=0}^k d(x_i^{0},x_{i+1}^{0})\geq L-\epsilon$. For $t\in[-1,1]$ denote by $x_i^{t}$ the
point $\gamma_t$ corresponding to $x_i^{0}$. Now use the comparison triangles
$\triangle(x_{j}^{-1},x_{j+1}^{-1},x_{j+1}^{+1})$ and $\triangle(x_{j}^{+1},x_{j+1}^{+1},x_{j+1}^{-1})$, $j=0,\ldots k-1$,
to glue a flat comparison surface $S_\epsilon$ for $H$. From Reshetnyak's majorization theorem, we obtain a short map 
$f_\epsilon:S_\epsilon\to H$. Denote by by $\hat x_i^{t}$ the point on on $S_\epsilon$ corresponding to $x_i^{t}$. Then
$$
L-\epsilon\leq\sum_{i=0}^k d(x_i^{0},x_{i+1}^{0})\leq\sum_{i=0}^k d(\hat x_i^{0},\hat x_{i+1}^{0})\leq
\frac{1}{2}\sum_{i=0}^k d(x_i^{-1}, x_{i+1}^{-1})+\frac{1}{2}\sum_{i=0}^k d(x_i^{+1}, x_{i+1}^{+1})\leq L.
$$
Hence we can choose a sequence $\epsilon_j\to 0$ such that $S_{\epsilon_j}$ converges to a flat surface $S$ and $f_{\epsilon_j}$
converges to a short map $f:S\to H$. Note that $f$ is surjective. Moreover, $S$
has two transversal foliations, one by parallel paths of constant length $L$ and one by geodesic segments of constant length.
It follows that the Jacbian of $f$ is almost everywhere equal to one. Thus $f$ is area preserving and therefore an isometry by \ref{}.
\qeds


%MAYBE IT WILL BE A COR OR PART OF AN OTHER STATEMENT, BUT WE NEED IT???
\begin{thm}{Proposition}\label{prop:strict-mm}
Any metric minimizing map in $\CAT[0]$ space is strict metric minimizing.
\end{thm}


\begin{thebibliography}{52}
\bibitem{moore}
Moore, R. L.,
``Concerning upper semi-continuous collections of continua,''
Trans. Amer. Math. Soc. 27 no. 4 (1925) pp. 416--428.

\bibitem{BBI}Burago, D.; Burago, Y.; Ivanov, S.
A course in metric geometry.
Graduate Studies in Mathematics, 33. American Mathematical Society, Providence, RI, 2001. xiv+415 pp.

\bibitem{KS}Korevaar, N. J.; Schoen, R. M. ``Sobolev spaces and harmonic maps for metric space targets,'' Comm. Anal. Geom., 1(3-4):561-659, 1993.

\bibitem{LW}Lytchak, A.; Wenger, S. ``Area minimizing discs in metric spaces,'' preprint arXiv:1502.06571, 2015.

\bibitem{petrunin-orthodox} A. Petrunin
``Exercises in Orthodox Geometry''{\tt arXiv:0906.0290 [math.HO]}

\bibitem{R}Reshetnyak, Yu. G. ``Sobolev classes of functions with values in a metric space,'' II, Sibirsk. Mat. Zh. 45 (2004), no. 4, 855-870. MR 2091651 (2005e:46055)

\bibitem{shefel-2D} S. Z. \v{S}efel′, On saddle surfaces bounded by a rectifiable curve, Dokl. Akad. Nauk SSSR 162 (1965), 294--296.

\bibitem{shefel-3D} S. Z. \v{S}efel′, On the intrinsic geometry of saddle surfaces, Sibirsk. Mat. Ž. 5 (1964), 1382--1396

\bibitem{W1}Whyburn, G. T., ``On sequences and limiting sets,'' Fund. Math. vol. 25 (1935) pp. 408-426.

\bibitem{W2}Whyburn, G. T., ``Analytic topology,'' Amer. Math. Soc. Colloquium Publications, vol. 28, 1942.

\bibitem{Wi}Wilder, R. L., ``Topology of Manifolds,'' American Mathematical Society Colloquium Publications, vol. 32. American Mathematical
Society, New York, N. Y., 1949.
\end{thebibliography}


\end{document}