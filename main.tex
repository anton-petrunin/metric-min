\section{Proof assembling}\label{Main theorem}

The following lemma will estabish the final step in the proof of the main theorem.

\begin{thm}{Lemma}\label{lem:maj is isom}
Let $Y$ be a $\CAT[0]$ space and $s\:\DD\to Y$ be a metric minimizing map.
Assume that there is a $\CAT[0]$ disc retract $W$ with boundary curve $\delta$ and a short map $f\:\<D\>_s \to W$
such that $f\circ \delta_s=\delta$. If there exists a short map 
$q\: W\to Y$ with $q\circ \delta=s|_{\partial \DD}$, the $f$ is an isometry.
\end{thm}

\parit{Proof.}
Let $r\:\DD\to W$ be the natural projection from the mapping cylinder of $\delta$. Then $r$ 
is a retraction with $r|_{\partial \DD}=\delta$.
The composition $\DD\xrightarrow{r}W\xrightarrow{q} Y$ fulfills $q\circ r|_{\partial \DD}=s|_{\partial \DD}$.

Note that  $\<W\>_q=\<\DD\>_{q\circ r}$ and the natural projection $\rho\: W\to \<W\>_q$ is short.
It follows that $\rho\circ f\: \<\DD\>_s\to \<\DD\>_{q\circ r}$ is short and therefore an isometry
because $s$ is metric minimizing. Then $f$ is an isometric embedding which contains $\delta$
in its image. By Lemma \ref{lem:geospace} and Proposition \ref{prop:|D|} $\<\DD\>_s$ is a complete geodesic space.
So $f$ has to be surjective and therefore an isometry.
\qeds

\parit{Proof of the main theorem.}
It is sufficient to consider
the case that $|\DD|_s$ is  homeomorphic to a disc.
Indeed, it is sufficient to show that a triangle $[xyz]$ in $\<\DD\>_s$ is thin. 
Once the above case has been proven, the closure of each connected open component of $\<\DD\>_s\backslash ([xy]\cup[yz]\cup[zx])$
surrounded by the triangle is $\CAT[0]$.
From this, the thinness of the triangle $[xyz]$ follows; see \cite{bishop}.

%Since $|\DD|_s$ is homeomorphic to the disc, we can identify $\DD$ and $|\DD|_s$ and therefore we can assume that the map $s$ is light; that is, inverse image of any point is zero-dimensional.

Given a finite set $F\subset \DD$,
denote by $\mathcal{W}_F$
the set of isometry classes of spaces $W$ which meet the conditions of the Key Lemma~\ref{lem:key}
for $F$;
according to the Key Lemma~\ref{lem:key} $\mathcal{W}_F\ne\emptyset$.
Note that for two finite sets $F\subset F'$ in $\DD$,
we have $\mathcal{W}_F\supset \mathcal{W}_{F'}$.

According to the Compactness Lemma \ref{lem:compact} $\mathcal{W}_F$ is compact.
Therefore 
\[\mathcal{W}
=
\bigcap_{F}\mathcal{W}_F\ne \emptyset\]
where the intersection is taken over all finite subsets $F$ in $\DD$. 


Fix a space $W$ from $\mathcal{W}$;
the space $W$ is a $\CAT[0]$ disc retract,
such that given a finite set $F\subset \DD$ there is a map $h_F\:F\to W$ which is short with 
respect to $\<{*}-{*}\>_s$ 
and a short map $q_F\:W\to Y$ such that $q_F\circ h_F$ agrees with $s$ on $\partial\DD\cap F$.

Given a finite set $F\subset \DD$,
denote by $\mathfrak{S}_F$ the set of all maps $h_F\:F\to W$ described above.
Note that $\mathfrak{S}_F$ satisfies the assumption of the finite-whole extension lemma \ref{lem:finite-whole}.

Indeed, $\mathfrak{S}_F$ is closed by Lemma \ref{lem:closed} and the condition on the restriction of $h_F\in  \mathfrak{S}_F$ is evident.

Applying the finite-whole extension lemma \ref{lem:finite-whole},
we get a map $h\:\DD\to W$ such that $h|_F\in \mathfrak{S}_F$
for any finite set $F\subset \DD$.

Our next aim is to show that there is a single map $q$ such that
for all finite sets $F$ the composition $q\circ h|_F$ agrees with
$s$ on $\partial\DD\cap F$.
This is done by applying the ultralimit+projection construction.

Choose a sequence of finite sets $F_n$ such that the intersections $F_n\cap\partial \DD$ get denser and denser in $\partial \DD$, 
denote by $q_n$ the corresponding maps.
Let $q_\omega\:W\to Y^\omega$ be the ultralimit of $q_n$ and set $q=\nu\circ q_\omega$,
where $\nu\:Y^\omega\to Y$ is the closest-point-projection.
By construction $q\:W\to Y$ is short and $q\circ h$ agrees with $s$ on $\partial \DD$.
Note that we can't quite conclude $s\succcurlyeq q\circ h$ because $h$ might not be continuous.

However, by construction, the map $h$ induces a short map $\hat h\:\<\DD\>_s\to W$ 
such that $\hat h\circ\delta_s$ is the boundary curve of $W$. By Lemma \ref{lem:maj is isom}
$\hat h$ is an isometry and the statement follows.

\qeds