\section{Discs}


\begin{thm}{Proposition}\label{prop:mono-disc}
Let $W$ is a $\CAT[0]$ disc retract with boundary curve $\delta$.
Assume $f\:\DD\to W$ is a continuous map such that $f|_{\partial\DD}=\delta$ 
and for any geodesic $\gamma$ in $W$ each connected component of $\DD\backslash f^{-1}\gamma$ intersects the boundary $\partial \DD$.
Then $f$ is monotone.
\end{thm}

As a corollary we get the following elementary statement which is equally hard to prove.

\begin{thm}{Corollary}
Let $f\:\DD\to \DD$ be a light map such that restriction $f|_{\partial\DD}$ is the identity map.
Assume that for any line segment $I\subset \DD$ each connected component of $\DD\backslash f^{-1}I$ intersects the boundary $\partial \DD$.
Then $f$ is a homeomorphism.
\end{thm}

\parbf{Remark.} 
Before getting into the proof, let us present a \emph{fake} one based on the idea from the problem ``Saddle surface'' in \cite{petrunin-orthodox}.
We say where we cheat in the footnote; 
it is hard (if at all possible) to fix it.

\parit{Fake proof.}
Note that  $\deg f=1$;
in particular $f$ is onto.
It remains to show that $f$ is injective.

Assume  $w=f(x)=f(y)$ for distinct points $x,y\in\DD$
Note that  $w$ lies in the interior of $\DD$.
Choose a geodesic $\gamma$ which passes through $w$ and goes 
from boundary to boundary of $\DD$.
The inverse image $p^{-1}(\gamma)$ is a contractible set with two ends at $\partial\|\DD\|_s$, say $a$ and $b$.
We can assume that the points $a,x,y,b$ appear in the same order on $p^{-1}(\gamma)$.\footnote{This is where we are cheating: the inverse image $p^{-1}(\gamma)$ might be as terrible as psedoarc, where the order points has no sense.}

Note that there is a continuous one parameter family of geodesics $\gamma_t$ passing through $w$ with the ends at $\partial \DD$
such that $\gamma=\gamma_0$ and $\gamma_1$ is $\gamma$ with reversed parametrization.
Note that the order of $x$ and $y$ on $p^{-1}(\gamma_t)$ does not change in $t$.
On the other hand the orders on $\gamma_0$ and on $\gamma_1$ are opposite, a contradiction.\qeds

\begin{center}
\begin{lpic}[t(-2 mm),b(-0 mm),r(0 mm),l(0 mm)]{pics/mapping-cylinder(1)}
\lbl{53,17;$W$}
\lbl{60,6;$\hat W$}
\lbl[rb]{67,22.5;$\delta$}
\lbl[rb]{18,16;$\DD$}
\lbl[rb]{18,3;$\hat\DD$}
\lbl[rb]{41,17;$\xrightarrow{\hat f}$}
\end{lpic}
\end{center}

\parit{Proof.}
Denote by $\hat W$ the mapping cylinder  of $W$ over the boundary curve $\delta\:\SS^1\to W$;
that is
\[\hat W=W\bigsqcup_{\delta(x)\sim (x,0)} \SS^1\times[0,1].\]
Recall that $\hat W$ is homeomorphic to the disc.
Consider the mapping cylinder $\hat \DD$ of $\DD$ over the boundary;
$\hat \DD$ can be identified with a disc of radius $2$.
The identity map $\SS^1\times [0,1]$ induce an extension of $f\:\DD\to W$ to the map 
$\hat f\:\hat \DD\to \hat W$.
Note that $f$ is monotonic if and only if $\hat f$ is.

Given $u,v\in\SS^1$ three paths $u\times [0,1]$, $v\times[0,1]$ with the geodesic between $\delta(u)$ and $\delta(v)$ in $W$;
the concatenation of these paths in $\hat W$ as well as any closed arc in this concatenation will be called \emph{test-curve}.
Note that any two points $x$ and $y$ in $\hat W$ are joined by a unique test-curve
and for any test-curve which will be denoted $\langle xy\rangle$.

By the assumption in the proposition, each connected component of $\hat \DD\backslash \hat f^{-1}\langle xy\rangle$ for any test curve $\langle xy\rangle$ intersects $\partial \hat \DD$.

Note that $\hat f\:\hat \DD\to \hat W$ has degree 1 and in particular it is onto.

Assume $\hat f$ is not monotone;
that is, there is a point $x\in \hat W$ such that the inverse image $\hat f^{-1}\{x\}$ is not connected.
Clearly $x\notin\partial \hat W$.

Consider two open subsets $\Theta\subset\hat \DD$ and $\Omega\subset \hat \DD\times\SS^1$.
\begin{align*}
\Theta&=\hat \DD\backslash f^{-1}\{x\},
\\
\Omega&=\set{(z,s)\in \hat \DD\times\SS^1}{f(z)\not\in \langle x\,h(s)\rangle}.
\end{align*}

The restriction of the projection $\hat \DD\times\SS^1\to \hat \DD$ sends $\Omega$ to $\Theta$.
Note that indunced homomorphism $\pi_1\Omega\to \pi_1\Theta$ is onto.
Indeed, given $z\in \Theta$, we can choose a $s_z\in\SS^1$ such that 
$f(z)\notin \langle x\,s_z\rangle$;
moreover the function $z\to s_z$ can be chosen to be continuous.
In particular for any loop $\lambda\:[0,1]\to\Theta$, the loop 
\[\bar\lambda\:t\mapsto (\lambda(t),s_{\lambda(t)})\]
is a lift of $\lambda$ in $\Omega\subset \DD\times\SS^1$.

Since $x\notin\partial \hat W$, we get $\partial\DD\subset \Theta$.
Since $\{x\}=\langle x\,x\rangle$, every connected component of $\Theta$ have to intersect $\partial\DD$.
Since $\partial\DD$ is connected so is $\Theta$.

Since $f^{-1}\{x\}$ is not connected, it can be divided into two subsets by a curve in $\Theta$.
Therefore $\pi_1\Theta$ contains a free group with two generators.

On the other hand, the restriction of the projection $\DD\times\SS^1\to \SS^1$ to $\Omega$ has fiber 
$\Phi_s=\DD\backslash f^{-1}\langle xs\rangle$ at point $s\in\partial \hat W$.
Note that $\Phi_s$ is simply connected and therefore the projection $\Omega\to \SS^1$ induce an isomorphism of fundamental groups.
In particular 
\[\pi_1\Omega=\ZZ,\]
a contradiction.
\qeds

