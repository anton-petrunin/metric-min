\section{Discs}

Recall that a map between topological spaces is called monotone if the inverse image of any point is connected.
Note that by the definition, any monotone map is onto.

\begin{thm}{Proposition}\label{prop:mono-disc}
Let $f\:\DD\to \DD$ be a continuous map such that restriction $f|_{\partial\DD}$ is a monotone map $\partial\DD\to\partial\DD$.
Assume that for any closed convex set $K\subset \DD$ each connected component of $\DD\backslash f^{-1}K$ intersects the boundary $\partial \DD$.
Then $f$ is monotone.
\end{thm}

\parbf{Remark.} 
One may think that proof can be done along the same lines as for the problem ``Saddle surface'' in \cite{petrunin-orthodox}. 
Unfortunately the proof requires some extra regularity of $f$. 

Here is the \emph{fake proof}, we say where we cheat in the footnote; 
it is hard (if at all possible) to fix its precise meaning.
In the fake proof we assume that $f$ is light, that is inverse image of point is zero-dimensional --- the general case can reduced this case.

\parit{Fake proof.}
Assume  $w=f(x)=f(y)$ for distinct points $x,y\in\DD$
Note that  $w$ lies in the interior of $\DD$.
Choose a geodesic $\gamma$ which passes through $w$ and goes 
from boundary to boundary of $\DD$.
The inverse image $p^{-1}(\gamma)$ is a contractible set with two ends at $\partial\|\DD\|_s$, say $a$ and $b$.
We can assume that the points $a,x,y,b$ appear in the same order on $p^{-1}(\gamma)$.\footnote{This is where we are cheating: the inverse image $p^{-1}(\gamma)$ might be as terrible as psedoarc, where the order points has no sense.}

Note that there is a continuous one parameter family of geodesics $\gamma_t$ passing through $w$ with the ends at $\partial \DD$
such that $\gamma=\gamma_0$ and $\gamma_1$ is $\gamma$ with reversed parametrization.
Note that the order of $x$ and $y$ on $p^{-1}(\gamma_t)$ does not change in $t$.
On the other hand the orders on $\gamma_0$ and on $\gamma_1$ are opposite, a contradiction.\qeds 

\parit{Proof.}
Note that any monotone map $\SS^1\to\SS^1$ has degree $\pm1$.
If follows $f|_{\partial\DD}\:\partial\DD\to\partial\DD$ has degree $\pm1$
and therefore the same holds for $f\:\DD\to\DD$.
In particular, $f\:\DD\to\DD$ is onto.

Assume $f$ is not monotone.
Then there is a point $x\in \DD$ such that the inverse image $f^{-1}\{x\}$ is not connected.

Consider two open subsets $\Theta\subset\DD$ and $\Omega\subset \DD\times\SS^1$.
\begin{align*}
\Theta&=\DD\backslash f^{-1}\{x\},
\\
\Omega&=\set{(z,s)\in \DD\times\SS^1}{f(z)\not\in [xs]}.
\end{align*}

The restriction of the projection $\DD\times\SS^1\to \DD$ sends $\Omega$ to $\Theta$.
Note that indunced homomorphism $\pi_1\Omega\to \pi_1\Theta$ is onto.
Indeed, given $z\in \Theta$, we can choose a $s_z\in\SS^1$ such that 
$f(z)\notin [xs_z]$;
moreover the function $z\to s_z$ can be chousen to be contiouous.
In particular for any loop $\lambda\:[0,1]\to\Theta$, the loop 
\[\bar\lambda\:t\mapsto (\lambda(t),s_{\lambda(t)})\]
is a lift of $\lambda$ in $\Omega\subset \DD\times\SS^1$.

Assume $x\notin\partial\DD$.
Then $\partial\DD\subset \Theta$.
Since $\{x\}$ is convex, every connected component of $\Theta$ have to intersect $\partial\DD$.
Since $\partial\DD$ is connected so is $\Theta$.

Since $f^{-1}\{x\}$ is not connected, it can be divided into two subsets by a curve in $\Theta$.
Therefore $\pi_1\Theta$ contains a free group with two generators.

The restriction of the projection $\DD\times\SS^1\to \SS^1$ to $\Omega$ has fiber 
$\Phi_s=\DD\backslash f^{-1}[xs]$ at point $s\in\SS^1$.
Note that $\Phi_s$ is simply connected and therefore the projection $\Omega\to \SS^1$ induce an isomorphism of fundamental groups.
In particular 
\[\pi_1\Omega=\ZZ,\]
a contradiction.

The case $x\in\partial\DD$ is done in a similar fashion.

The fiber $\Phi_s$ has one or two connected components each is simply connected.
The fiber $\Phi_x$ has one component.
Further any closed curve in $\Omega$ can be contracted to the fiber $\Phi_x$ and then to a point.
Therefore $\Omega$ is simply connected.

On the other hand, since $f^{-1}\{x\}$ is not connected, there is a closed curve in $\Theta$ which separates a nonempty subset in $f^{-1}\{x\}$ and $\partial \DD$. 
This curve generates an infinite cyclic subgroup in $\pi_1\Theta$.
Since $\pi_1\Omega\to \pi_1\Theta$ is onto, the latter leads to a contradiction.
\qeds

