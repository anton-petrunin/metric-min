\section{Metric minimizing graphs}\label{Metric minimizing graphs}

\emph{Metric minimizing graphs} are defined analog to metric minimizing discs.

Namely, let $Y$ be a metric space, $\Gamma$ a graph and $A$ a subset of its vertexes.
Given two maps $f,f'\:\Gamma\to Y$, we write $f\succcurlyeq f'\rel A$ if $f$ and $f'$ agree on $A$ 
and there is a majorization $\mu\:\<\Gamma\>_f\to \<\Gamma\>_{f'}$
such that $f(a)\zz=f'(\mu(a))$ for any $a\in A$.

A map $f$ is called \emph{metric minimizing relative to $A$} if $f\succcurlyeq f'\rel A$ implies that the majorization $\mu$ is an isometry.

\begin{thm}{Proposition}\label{prop:metric-min-graph-exist}
Let $Y$ be a $\CAT[0]$ space, 
$\Gamma$ a finite  graph and $A$ a subset of its vertexes.

Given a map $f\:\Gamma\to Y$ there is a metric minimizing map $h\:\Gamma\to Y$
such that $f\succcurlyeq h\rel A$.
\end{thm}

\parit{Proof.}
Note that if we replace $f$ by a map $f'$ which agrees with $f$ on the vertexes of $\Gamma$ and
restricts on edges to constant speed parametrizations of the corresponding geodesics, then
$f\succcurlyeq f'\rel A$. Since $\Gamma$ is finite we therefore may assume that $f$ is Lipschitz to begin with.
Consequently, the sublevel $\mathcal{M}_f:=\{f\succcurlyeq g\rel A\}$ consists of Lipschitz maps $g\: \Gamma\to Y$ with the same Lipschitz
constant as $f$. By Arzel\`a-Ascoli $\mathcal{M}_f$ is compact.
 We will use Zorn's Lemma to show that 
$\mathcal{M}_f$ has a minimal element.

Let $\mathcal{O}$ be a well-ordered subset of $\mathcal{M}_f$. 
We need to show that $\mathcal{O}$ has a lower bound.
If we view $\mathcal{O}$ as a net indexed by itself, then by compactness of 
$\mathcal{M}_f$ there exists a limit point $s$ of $\mathcal{O}$. Clearly, $s$
is a lower bound of $\mathcal{O}$.
\qeds

\begin{thm}{Proposition}\label{prop:metric-min-graph}
Let $Y$ be a $\CAT[0]$ space, 
$\Gamma$ a finite  graph and $A$ a subset of its vertexes.
Assume the map $v\mapsto v'$ is metric minimizing map $\Gamma\to Y$ relative to $A$.
Then
\begin{enumerate}[(a)]
\item each edge of $\Gamma$ maps to a geodesic;
\item for any vertex $v\notin A$ and any $x\ne f(v)$
there is an edge  $[vw]$ in $\Gamma$ such that
$\measuredangle[v'\,^{w'}_x]\ge \tfrac\pi2$;
\item\label{sum>=2pi} for any vertex $v\notin A$ and any cyclic order $w_1,\dots,w_n$ of adjacent vertexes we have
\[\measuredangle[v'\,^{w'_1}_{w'_2}]+\dots+\measuredangle[v'\,^{w'_{n-1}}_{w'_n}]+\measuredangle[v'\,^{w'_n}_{w'_1}]\ge 2\cdot\pi.\]
\end{enumerate}
\end{thm}

\begin{wrapfigure}{r}{22 mm}
\begin{lpic}[t(-8 mm),b(-0 mm),r(0 mm),l(0 mm)]{pics/not-sufficient(1)}
%\lbl[lb]{12.5,11;$W_0$}}
\end{lpic}
\end{wrapfigure}

\parit{Remark.}
As one may see from the diagram, the conditions in the proposition do not guarantee that the map $f$ is metric minimizing,
the solid points form the set $A$.

\parit{Proof.}
The first condition is evident.

Assume the second condition does not hold at a vertex $v\zz\notin A$;
that is, there is a point $x\in Y$ such that
$\measuredangle[v'\,^{w'}_x]< \tfrac\pi2$
for any adjacent vertex $w$.
In this case moving $v'$ toward $x$ along $[v',x]$ decrease the lengths of all edges adjacent to $v$, a contradiction.

Assume the third condition does not hold, that is, 
the sum of the angles around a fixed interior vertex $v'$ is less than $2\cdot\pi$.

Recall that the space of directions $\Sigma_{v'}$ is a $\CAT[1]$ space.
Denote by $\xi_1,\dots,\xi_n$ the directions of $[v'w'_1],\dots, [v',w'_n]$ in $\Sigma_{v'}$.
By the assumption, we have
\[|\xi_1-\xi_2|_{\Sigma_{v'}}+\dots+|\xi_n-\xi_1|_{\Sigma_{v'}}<2\cdot\pi.\]
By Reshetnyak's majorization theorem,
the closed broken line $[\xi_1,\dots,\xi_k]$ is majorized by a convex spherical polygon $P$.

Note that $P$ lies in an open hemisphere with the pole  at some point in $P$.
Choose $x\in Y$ so that the direction form $v'$ to $x$ coincides with the image of the pole in $\Sigma_{f(v)}$.
This choice of $x$ contradicts the condition in Proposition~\ref{prop:metric-min-graph}.
\qeds