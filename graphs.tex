\section{Metric minimizing graphs}\label{Metric minimizing graphs}

\emph{Metric minimizing graphs} are defined analog to metric minimizing discs.

Namely, let $Y$ be a metric space, $\Gamma$ be a finite graph and $A$ be a subset of its vertexes.
Given two maps $f,f'\:\Gamma\to Y$, we write $f\succcurlyeq f'\rel A$ if $f$ and $f'$ agree on $A$ 
and there is a majorization $\mu\:\<\Gamma\>_f\to \<\Gamma\>_{f'}$
such that $f(a)\zz=f'(\mu(a))$ for any $a\in A$.

A map $f\:\Gamma\to Y$ is called \emph{metric minimizing relative to $A$} if $f\succcurlyeq f'\rel A$ implies that the majorization $\mu$ is an isometry.

\begin{thm}{Proposition}\label{prop:metric-min-graph-exist}
Let $Y$ be a $\CAT[0]$ space, 
$\Gamma$ be a finite graph and $A$ be a subset of its vertexes.

Given a continuous map $f\:\Gamma\to Y$ there is a map $h\:\Gamma\to Y$ 
which is metric minimizing relative to $A$ and $f\succcurlyeq h\rel A$.
\end{thm}

\parit{Proof.}
Let us parametrize each edge of $\Gamma$ by $[0,1]$.
A map $h\:\Gamma\to Y$ will be called \emph{straight} if it
sends each edge of $\Gamma$ to a geodesic path in $Y$.

If $h\:\Gamma\to Y$ is straight then $f\succcurlyeq h\rel A$ if and only if 
$|f(v)-f(w)|_Y\zz\ge |h(v)-h(w)|_Y$
for any two adjacent vertexes $v$ and $w$ in $\Gamma$.
In particular, we can assume that the given map $f$ is straight.

Let $\set{f_\alpha}{\alpha\in \mathcal{O}}$ be a well-ordered subset of straight maps with respect to ``$\succcurlyeq\rel A$'' 
with $f$ as the maximal element. 
We need to show that $\{f_\alpha\}$ has a lower bound.

Assume contrary.
Note that
$$
\mathcal{F}
=
\set{Q\subset\mathcal{O}}{\{f_\alpha\}_{\alpha\notin Q}\text{ has a lower bound } f_\beta}.
$$
is a nonprincipal filter.
Complete $\mathcal{F}$ to an ultrafilter $\omega$ on $\mathcal{O}$ and apply the \emph{ultralimit+projection construction}.

Namely, denote by $Y^\omega$ the ultrapower of $Y$. 
Then $Y^\omega$
is a $\CAT[0]$ space which contains $Y$ as a closed convex subset. 
The $\omega$-limit $s_\omega\:\Gamma\to Y^\omega$ is well defined, since
$f$ and, therefore, all $f_\alpha$ are Lipschitz continuous. 
Denote $f'$ the composition of $f_\omega$ with the nearest point projection $Y^\omega\to Y$.
The nearest point projection to a closed convex set in $\CAT[0]$ space is short.
Therefore $f_\alpha\succcurlyeq f'$ for any $\alpha\in \mathcal{O}$. Let $f''$ denote the straightening of $f'$.
Then $f'\succcurlyeq f''$ and the claim follows from Zorn's Lemma.
\qeds


\begin{thm}{Proposition}\label{prop:metric-min-graph}
Let $Y$ be a $\CAT[0]$ space, 
$\Gamma$ be a finite  graph and $A$ be a subset of its vertexes.
Assume that the assignment $v\mapsto v'$ is a metric minimizing map $\Gamma\to Y$ relative to $A$.
Then
\begin{enumerate}[(a)]
\item each edge of $\Gamma$ maps to a geodesic;
\item\label{prop:metric-min-graph:x} for any vertex $v\notin A$ and any $x\ne v'$
there is an edge  $[v,w]$ in $\Gamma$ such that
$\measuredangle[v'\,^{w'}_x]\ge \tfrac\pi2$;
\item\label{sum>=2pi} for any vertex $v\notin A$ and any cyclic order $w_1,\dots,w_n$ of adjacent vertexes we have
\[\measuredangle[v'\,^{w'_1}_{w'_2}]+\dots+\measuredangle[v'\,^{w'_{n-1}}_{w'_n}]+\measuredangle[v'\,^{w'_n}_{w'_1}]\ge 2\cdot\pi.\]
\end{enumerate}
\end{thm}

\begin{wrapfigure}{r}{22 mm}
\begin{lpic}[t(-8 mm),b(-0 mm),r(0 mm),l(0 mm)]{pics/not-sufficient(1)}
\end{lpic}
\end{wrapfigure}

\parit{Remark.}
The conditions in the proposition do not guarantee that the map $f$ is metric minimizing.
An example can be guessed from the diagram, where the solid points form the set~$A$. 


\parit{Proof.}
The first condition is evident.

Assume the second condition does not hold at a vertex $v\zz\notin A$;
that is, there is a point $x\in Y$ such that
$\measuredangle[v'\,^{w'}_x]< \tfrac\pi2$
for any adjacent vertex $w$.
In this case moving $v'$ toward $x$ along $[v',x]$ decreases the lengths of all edges adjacent to $v$, a contradiction.

Assume the third condition does not hold, that is, 
the sum of the angles around a fixed interior vertex $v'$ is less than $2\cdot\pi$.

Recall that the space of directions $\Sigma_{v'}$ is a $\CAT[1]$ space.
Denote by $\xi_1,\dots,\xi_n$ the directions of $[v',w'_1],\dots, [v',w'_n]$ in $\Sigma_{v'}$.
By assumption, we have
\[|\xi_1-\xi_2|_{\Sigma_{v'}}+\dots+|\xi_n-\xi_1|_{\Sigma_{v'}}<2\cdot\pi.\]
By Reshetnyak's majorization theorem,
the closed broken line $[\xi_1,\dots,\xi_k]$ is majorized by a convex spherical polygon $P$.

Note that $P$ lies in an open hemisphere with pole  at some point in $P$.
Choose $x\in Y$ so that the direction from $v'$ to $x$ coincides with the image of the pole in $\Sigma_{f(v)}$.
This choice of $x$ contradicts (\emph{\ref{prop:metric-min-graph:x}}).
\qeds