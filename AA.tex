



\section{2-dimensional case}

Let $W$ be a metric space.
A coninous map $s\:\DD\to W$ is called \emph{saddle} if for any closed convex subset $K\subset W$ 
any connected component of the complement $\DD\backslash s^{-1}(K)$
has a point from the boundary $\partial \DD$.

Applying this definition to the one-point sets,
we get that any saddle map has no bubbles.


\begin{thm}{Proposition}
Let $W$ be a $\CAT[0]$ space then any metric minimizing map $s\:\DD\to W$ is saddle.
\end{thm}

\parit{Proof.}
Assume contrary, let $K$ be a convex set in $W$
and $\Omega$ be the violating component of the complement $\DD\backslash s^{-1}(K)$.
Redefine $s$ for each $x\in\Omega$ by moving 
$s(x)$ to its closest point projection on $K$.
Denote by $s'$ the new map.

Since $K$ is convex the closest point projection is short,
therefore $s\succcurlyeq s'$.
That is $s$ is not strictly metric minimizing.
It remains to apply Proposition~\ref{prop:strict-mm}.
\qeds


\begin{thm}{Shefel's theorem}\label{thm:shefel-2D}
Let $W$ be a $\CAT[0]$ space which is a disc retract.
Assume $s\:\DD\to W$ be a saddle map. 
Then $\|\DD\|_s$ is $\CAT[0]$.
\end{thm}

We almost repeat the proof of Shefel in \cite{shefel-2D}.
The main difference is instead of ``plane'' one has to say ``$W$''.
However, Shefel's proof was written rather tight, so we have through in few details.

In the proof we will need the following lemmas.

\begin{thm}{Cutting hat lemma}\label{lem:cutting-hat}
Let 
$W$ be a $\CAT[0]$ space, 
$K\subset W$ be a closed convex set 
and $\eps>0$.
Assume $s\:\DD\to W$ is a saddle map 
and $u\:\DD\to K$ is a continuous map such that 
\[|s(x)-u(x)|<\eps\]
and $u(x)\in K$ for any point in the boundary of some open set $\Omega\subset \DD$.

Then there is a continuous map $v\:\DD\to K$ such that 
\begin{enumerate}[(i)]
\item $|s(x)-v(x)|<\eps$ holds for any $x\in\DD$.
\item $u(x)=v(x)$ for any $x\notin\Omega$ and
$v(x)\in K$ for any $x\in\Omega$.
\end{enumerate}
Moreover, we can assume that the constructed map $v$ has no bubbles
and 
\[v(\Omega)\subset \Conv u(\partial\Omega),\]
where $\Conv S$ denotes the minimal closed convex set containing $S$.
\end{thm}

\parit{Proof.}
Fix a continuous function $\lambda\:\Omega\to [0,1]$
which is $0$ on $\partial \Omega$ and $1$ outside a sufficiently small neighborhood of $\partial \Omega$.
Note that the closest point projection of $W$ to $K$ is a short map.

For $x\notin \Omega$ set $v(x)=u(x)$ and if 
$x\in \Omega$, set $v(x)$ to be the point which minimum point of the function 
\[\lambda\cdot\dist^2_p+(1-\lambda)\cdot\dist^2_q,\]
where $p$ and $q$ are correspondingly the closest point projections of  $u(x)$ and $s(x)$ on $u(\partial\Omega)$.
Note that for some choice of $\lambda$ the map $v\:\DD\to W$ satisfies
both conditions.

To prove the first \emph{moreover} statement we need to remove bubbles form the constructed map $v$.
Namely any $x$ we can choose the maximal open set $\Upsilon_x$ such that for some point $p\in W$ the complement $\DD\backslash v^{-1}\{p\}$ has $\Upsilon_x$ as a
connected component with no points from $\partial \DD$.
Redefine $v$ by setting $v(x)=p$.

Finally note that $\Conv u(\partial\Omega)\subset K$. 
Therefore to prove the second \emph{moreover} statement 
we need to apply lemma for the convex set $K'=\Conv u(\partial\Omega)$.
\qeds





\parit{Proof of Shefel's theorem \ref{thm:shefel-2D}.}
Fix a fine triangulation $\tau$ of $\DD$.
Map the vertices of $\tau$ by $s$
and extend it to whole $\DD$ by using baricentric map on each simplex.

The obtained map $u\:\DD\to W$ can be made arbitrary close to $s$ assuming that the triangulation is fine.
Say given $\eps>0$ we can assume that 
\[|s(x)-u(x)|_X<\eps\eqlbl{eq:|s-u|<e}\]
for any $x\in\DD$.

Note that the image of each triangle $\triangle$
is the closed region bounded by the geodesic triangle in $W$ with the corresponding vertices.
Moreover,  $\|\triangle\|_u$ is isometric to the image $u(\triangle)$.

It follows that, the space $\|\DD\|_u$ can be reconstructed by gluing solid geodesic triangles in $W$ along their common sides.

Denote by $N$ the number of triangles in $\tau$.
Consider the class of metric spaces $\mathcal{S}_N$ 
which are homeomorphic to a disc retract
and can be constructed gluing at most $N$ convex sets in $W$ 
along some arcs in their boundaries.
Note that $\mathcal{S}_N$ is compact in Gromov--Hausdorff topology.

From above, $\|\DD\|_u$ belongs to $\mathcal{S}_N$.  

Note that the space $\Gamma$ of all geodeisics in $W$ withthe ends on $\partial W$ is compact. 
Fix a dense sequence of geodesics $\gamma_1,\gamma_2,\dots$ in $\Gamma$.
The geodesics are convex form sets in $W$, so we can apply Cutting hat lemma \ref{lem:cutting-hat}
recursively.
We obtain a sequence of maps, say $u=u_0,u_1,u_2,\dots$ such that 
\[|u_n(x)-s(x)|<\eps\]
for any $x$.

Further each space $\|\DD\|_{u_n}$
belong to $\mathcal{S}_N$.
Denote by $w_n\:\|\DD\|_{u_n}\to W$ the its projection.

Since $\mathcal{S}_N$ is compact, we can pass to its partial limit as $n\to\infty$,
say $\|\DD\|_\eps$,
denote by $w\:\|\DD\|_\eps\to W$ the limit map.

Note that the projection $w\:\|\DD\|_\eps\to W$ is saddle.
Otherwise we could cut a hat from $w$ along some geodesic $\gamma$.
There should be a point, say $y$ in one of the convex sets in the decomposition of $\|\DD\|_\eps$ which disappear after cutting along $\gamma$.
Since the sequence $\gamma_n$ is dense in the space of all geodesics
we can choose a geodesic $\gamma_n$ which cuts a hat with $y$ inside, a contradiction.

It follows that the convex sets in the decomposition of $\|\DD\|_\eps$ are glued along geodesics,
if not a hat could be cutted from the map $w$.

Note that the space $\|\DD\|_\eps$ is $\CAT[0]$. 
Indeed it is glued from $\CAT[0]$ spaces along geodesics,
so we only need to check if the total angle around a vertex is at least $2\cdot\pi$.
If the latter does not hold then one could cut a hat from $w$. %???

To show that $\|\DD\|_s$ is $\CAT[0]$,
we need to show that $\CAT[0]$ comparison holds for any quadruple of points  in $\DD$.
Fix quadruple $(x,y,z,t)$, we can assume that these 4 points as well as sufficiently dense set of points on the

It follows that that $\|\DD\|_\eps$ converges to a $\CAT[0]$ space, say $L$ as $\eps\to 0$. 
Clearly there is a short map $L\to\ \|\DD\|_s$ which is length preserving on the boundary.
By Reshetnyak's majorization theorem, there is a the boundary of $L$ admits a majorization.
The projection of this majorization to $\|\DD\|_s$ gives a majorization of $\partial\|\DD\|_s$.

Hence the result follows.
\qeds












\begin{thm}{Proposition}
Let $Y\in\CAT(0)$ 
and 
$f\:\DD\to Y$ be a metric minimizing embedding.
Then $(\DD,\hat\rho_f)\in \CAT(0)$.
\end{thm}



\parit{Proof.}
Let us glue $\RR^2$ to $Y$ along $f$.
We obtain a metrizable space, say $\hat Y$
and the map $f\:\DD\to Y$ admits a natural extension to $\hat f\:\RR^2\to \hat Y$.

According to Proposition~\ref{prop:point-complement}, for any point $y\in \hat Y$ the complement 








Let $A\subset X$ be a closed subset.
A map $f\:X\to Y$ is called metric minimizing rel. $A$
if there is no map $h\:X\to Y$ such that $f\succ h$ and $h(a)=f(a)$ for any $a\in A$.


Let $X$ be a compact connected topological space
and $Y$ be metric spaces.
Given a map $f\:X\to Y$ and two points $x,y\in X$
set 
\[\rho_f(x,y)=\inf\{\diam K\},\]
where infimum is taken for all connected sets $K\subset X$
such that $K\supset x,y$.

Note that $\rho_f$ defines a pseudmetric on $X$;
that is it satifies all the condition for the metric except for $\rho_f(x,y)=0$ does not nesessury implies $x=y$.


Let us consider the equivalence realtion $\sim$ on $X$ 
defined by
$x\sim y$ if and only if $\rho_f(x,y)=0$.
The equivalence class of $x\in X$ with respect to $\sim$
will be denoted by $[x]_f$.
We will denote by $X_f$ 
the metric space of all the equvalence classes with the metric induced by $\rho_f$.

Note that the map $X\to X_f$ defiend as $x\mapsto [x]_f$
is continuous $f$;
in particular $X_f$ is compact.

Consider the induced length metric $\hat\rho_f$ on $X_f$.


Assume $X$ and $Y$ be two metric spaces 
and $f\:T\to M$ is an arbitrary map.

Let us define the intrinsic pullback premetric along $f$.

Fix $\eps>0$.
Given two points $x,y\in X$
set 
\[\ell_\eps(x,y)=\inf\left\{\sum_{i\ge 1}|f(x_i)-f(x_{i-1})|_Y\right\}\]
where the infimum is taken for all $\eps$-chains
$x=x_0,x_1,\dots,x_n=y$.

Set
\[\rho_f(x,y)=\lim_{\eps\to0}\ell_\eps.\]
Note that the function $\eps\mapsto\ell_\eps$ is nonincreasing;
therefore the limit is always defined although might be infinite.

It is straightforward to check that $\rho_f$ is a pseudometric;
that is it satisfies all the axioms of metric except $\rho_f(x,y)=0$ does not nesessury impy $x=y$.
On the other hand, $\rho_f$
defines genuine metric on the quoteint space $X/sim$,
where $x\sim y$ if and ony if $\rho_f(x,y)=0$.
The metric space $(X/\sim,\rho_f)$ will pullback space along $f$
and denoted as $?_f$.

\begin{thm}{Definition}
Let $f\:X\to Y$ be a map between metric spaces
and $A\subset X$.
The map $f$ is called \emph{metric minimizing rel.} $A$
if for any other map $h\: X\to Y$ 
such that $f(a)=h(a)$ for any $A$
the inequality
\[\rho_{h}(x,y)\le \rho_f(x,y)\]
for any pair $x,y\in X$
implies the equality
\[\rho_{h}(x,y)= \rho_f(x,y)\]
for any pair $x,y\in X$.
\end{thm}

For example a minimizing geodesic with any reparametrization is a metric minimizing map fro an interval rel. its ends.

If $X$ is a 2-dimesional disc $\DD$,
we will say that $\DD$ is metric minimizing if it is metric minimizing rel. its boundary

\begin{thm}{Proposition}
For any map $f\:X\to Y$ we have the inequality 
$\rho_f(x,y)\le |f(x)-f(y)|_X$
holds for any pair of points $x,y\in X$.
\end{thm}


\begin{thm}{Proposition}
Assume $f\:X\to Y$ be a continuous map.
Then the pullback pseudometric $\rho_f$ is lowe semicontinuous;
that is
\[\rho_h(x_0,y_0)
\le
\liminf_{|x-x_0|_X,|y-y_0|_X\to 0}
\rho_h(x,y).\]
\end{thm}