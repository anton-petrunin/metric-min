\documentclass[a4paper,12pt]{article}
\usepackage{amsmath}
\usepackage{amssymb}
\usepackage{amsthm}
\usepackage[all]{xy}

\newtheorem{theorem}{Theorem}
\newtheorem{claim}{Claim}
\newtheorem{lemma}{Lemma}
\newtheorem{proposition}{Proposition}
\newtheorem{obstruction}{Obstruction}
\theoremstyle{remark}\newtheorem*{remark}{Remark}
\theoremstyle{definition}\newtheorem{definition}{Definition}

\newcommand{\N}{\mathbb N}
\newcommand{\R}{\mathbb R}

\newcommand{\DD}{\mathbb D}

\newcommand{\len}{\operatorname{length}}
\newcommand{\im}{\operatorname{im}}
\newcommand{\ic}{intrinsically continuous}


\begin{document}

\title{Metric minimizing surfaces revisited}
\author{Anton Petrunin and  Stephan Stadler}
\date{}
\maketitle

%\begin{abstract}
%???
%\end{abstract}

\section{Definitions}



\begin{definition}[Induced metric, quotient space]
Let $f:Y\rightarrow Z$ be a continuous map from a topological 
space $Y$ to a metric space $Z$. Then the {\em induced metric}
$|\cdot,\cdot|_f$
on $Y$ is defined by
$$
|y,y'|_s:=\inf\{\len(f\circ\gamma)|\ 
\gamma \text{ a continuous path joining } y\text{ and }y'\}.
$$ 
The natural equivalence relation $y\sim y'$ if and only if 
$|y,y'|_f=0$ gives rise to a quotient space $Y_f:=Y/\sim$ and a
commutative diagram

\begin{displaymath}
    \xymatrix{ Y \ar[r]^f\ar[d]^{\pi_f} & Z  \\
               Y_f \ar[ru]_{\bar f} }.
\end{displaymath}


\end{definition}

We denote the closed unit disc in $\R^2$ equipped with the 
Euclidean metric by $\DD$.
As a rule, a {\em closed disc} $D$ is a space homeomorphic
to $\DD$. 

\begin{definition}
Let $D$ be a closed disc and $X$ a metric space. 
A {\em deformation} of a continuous map $f:D\rightarrow X$ is 
a continuous map $h:[0,1]\times D\rightarrow X$ such that 
$h_t:=h(t,\cdot)$ fulfills $h_t|_{\partial D}=f|_{\partial D}$ 
for all $t\in[0,1]$. 

The map $f$ is called {\em metric minimizing}, 
if there is no deformation $h$ such that 
$|\cdot,\cdot|_{h_t}\lneq|\cdot,\cdot|_f$ for some $t>0$.
\end{definition}




\begin{definition}[Intrinsic continuity]
A map $f:Y\to Z$ from a metric space $Y$ to a metric space
 $Z$ is called
{\em intrinsically continuous}, if for every $y\in Y$ 
and every $\epsilon>0$ there
exists $\delta>0$ such that any  $y'\in Y$
with $|y,y'|<\delta$ can be connected to $y$ by a path $\gamma$ 
with $\len(f\circ\gamma)<\epsilon$.
\end{definition}

\begin{remark}
\begin{enumerate}
 \item The main example of \ic\ maps are Lipschitz
maps from geodesic spaces to arbitrary metric spaces.
\item Restrictions of \ic\ maps to compact sets are uniformly \ic. 
\item If a map $f:Y\to Z$ is \ic, then 
the natural projection $\pi_f:Y\rightarrow Y_f$ is continuous. If
moreover $Y$ is compact, then $Y_f$ equipped with the induced 
metric is a geodesic space.
\end{enumerate}
\end{remark}

Our main goal is 

\begin{theorem}[Petrunin]
Let $X$ be a CAT(0) space and $f:\DD\rightarrow X$ an \ic\ map. If $f$ is metric minimizing and 
$f|_{\partial \DD}$ is an embedding, then $D_f$ is a CAT(0) disc.
\end{theorem}

\section{Existence}

\subsection{Planar graphs}
Let $\Gamma$ be a finite planar graph. If we specify an embedding of $\Gamma$ into a disc $D$, then 
we can define a {\em boundary} $\partial \Gamma$ as follows. A point $x\in\Gamma$ 
lies in $\partial\Gamma$ if and only if it can be connected to the 
boundary of the disc by a curve intersecting $\Gamma$ only at $x$. 
From now on, if we talk about the boundary of a finite planar graph $\Gamma$, then we
always refer to a fixed embedding $\Gamma\hookrightarrow D$.

Now let $\Gamma$ be a finite planar graph and $X$ be a proper CAT(0) space. 
For a continuous map $f:\Gamma\rightarrow X$ we define
$\mathcal{F}(f)$ to be the family of all maps $g:\Gamma\rightarrow X$ such that 
$g|_{\partial\Gamma}=f|_{\partial\Gamma}$ and
$\len(g(e))\leq\len(f(e))$ for all edges $e$ of $\Gamma$. Next, we specify a partial
order on  $\mathcal{F}(f)$ by assigning to an element $g\in \mathcal{F}(f)$ its total
length: $\len(g(\Gamma))$. If $f|_{\partial\Gamma}$ is rectifiable, then $\mathcal{F}(f)$
contains an element $\tilde f$ of finite total length. Moreover, by Arzel\`a-Ascoli, 
the sublevel $\{g\leq \tilde f\}$ is compact. Hence $\mathcal{F}(f)$ contains minimal 
elements. This yields the first half of

\begin{lemma}\label{lem:graphs}
Let $f:\Gamma\rightarrow X$ be a continuous map from a finite planar graph $\Gamma$ to 
a proper CAT(0) space $X$. If $f|_{\partial\Gamma}$ is rectifiable, then 
$\mathcal{F}(f)$ contains minimal elements. Every minimal element $s\in\mathcal{F}(f)$ fulfills:
\begin{enumerate}
 \item[(a)] For every interior\footnote{An edge is called {\em interior},
 if it does not contain a boundary point.} edge $e$ of $\Gamma$, $s(e)$ is either a geodesic
 or a point.
 \item[(b)] For every point $q\in X$ and every vertex $v\in\Gamma$ there is another vertex $\tilde v\in\Gamma$,
 adjacent to $v$, and such that $\measuredangle q v\tilde v\geq \tfrac\pi2$.
 \item[(c)] If $v\in\Gamma$ is a vertex, and $v_1,\ldots,v_k$ are all the vertices adjacent to $v$, then
 $$
 \sum_{i=1}^{k-1} \measuredangle v_i v v_{i+1}\geq 2\pi.
 $$
\end{enumerate}

\end{lemma}


\begin{proof}
 Item (a) is obvious, and item (b) follows from the first variation formula. At last, (c)
 follows from (b) and the fact that a loop of length $<2\pi$ in a CAT(1) space is contained in 
 a ball of radius stictly less than $\tfrac\pi2$.
\end{proof}

By part (a) above, we can think of a minimal element $s\in\mathcal{F}(f)$ as a map
$s:\Gamma'\rightarrow X$ where $\Gamma'$ results from $\Gamma$ by contracting all edges
$e$ which are mapped to single points by $s$. If $\Gamma\hookrightarrow D$ is the embedding
defining $\partial \Gamma$, then we obtain a commutative diagram

\begin{displaymath}
    \xymatrix{ \Gamma \ar[r]\ar[d] & D \ar[d] \\
               \Gamma' \ar[r] & D'  }
\end{displaymath}

where $D'$ is the quotient of $D$, again by contracting certain edges in $\Gamma$. The quotient map
$D\rightarrow D'$ is short and a uniform limit of homeomorphisms. In particular, $\Gamma'$ is planar.

  
\subsection{Discs}
Let $D$ be a closed disc and $X$ a proper metric space. Denote the set
of continuous maps $D\to X$ equipped with the compact-open topology 
by $\mathcal{C}(D,X)$. We define a partial 
order on $\mathcal{C}(D,X)$ by saying $f\leq g$ if and only if
$\len(f\circ\gamma)\leq \len(g\circ\gamma)$ for every continuous path
$\gamma:[0,1]\to D$.

We address the problem of finding a
metric minimizing disc below a certain a priori given one.

For a map $f\in\mathcal{C}(D,X)$ we 
define its relative sublevel $\mathcal{F}(f)$ consisting of all maps $g\in\mathcal{C}(D,X)$
with $g|_{\partial D}=f|_{\partial D}$ and $g\leq f$. If $f$ is \ic as a map from $\DD$, then
so is every element in $\mathcal{F}(f)$. In this case, the family
$\mathcal{F}(f)$ is equicontinuous. 
By Arzel\`a-Ascoli, it is precompact, since $X$ is proper and it is
closed by the semi-continuity of length structures. Hence, we conclude 
that $\mathcal{F}(f)$ contains a 
minimal element. This element is clearly metric minimizing, since the induced metric 
determines the length structure. This explains

\begin{proposition}
 If $X$ is a proper metric space, then for every \ic\ map $f:\DD\rightarrow X$,
 there is a metric minimizing map $s:\DD\rightarrow X$
 such that $|\cdot,\cdot|_s\leq|\cdot,\cdot|_f$.
\end{proposition}



\section{Intrinsic topology}

\begin{lemma}\label{lem:inttop}
Let $\gamma:S^1\rightarrow X$ be an embedded circle in a 
metric space $X$.
If $f:\DD\rightarrow X$ is an \ic\ map which
is metric minimizing with respect to 
$\gamma$, then $\pi_f:\DD\rightarrow D_f$ is a 
uniform limit of homeomorphisms.
\end{lemma}

\begin{proof}
By Moore's theorem (see \ref{}) it is enough to show that every 
fiber $\Pi_x:=f^{-1}(x)$, as well as every complement 
$\DD\setminus\Pi_x$, is connected.

For $x\in D_f$ and points $p,p'\in\Pi_x$ there is a sequence 
of paths $c_n$ joining $p$ and $p'$, and such that 
$\len(f\circ c_n)\to 0$. The images $\im(c_n)$ subconverge
in the Hausdorff topology to a compact connected subset 
$C\subset D$. Since $f$ is \ic, the set $C$
is contained in $\Pi_x$, and hence $\Pi_x$ is connected.

Next we show that $D_f$ has no cut point.

Since $\gamma$ is embedded, we know that 
$\partial \DD\setminus \Pi_x$ is connected. If $z$ would be a 
cut point in $D_f$, then $\pi_f(\partial \DD\setminus \Pi_x)$
is contained in a single component of $D_f\setminus \{z\}$.
Now we could alter the map $f$ by sending the inverse images of
all the other components to the point $\bar f(z)$. 
This would decrease the induced metric and hence contradict 
minimality. Therefore there is no cut point in $D_f$.

It follows that any two points in $D_f\setminus\{x\}$ are connected
by a piecewise geodesic since $D_f\setminus\{x\}$ is connected
and $D_f$ is geodesic. Consequently, $\DD\setminus \Pi_x$ is 
pathconnected. 
\end{proof}


\section{Approximating discs by finite graphs}

 
The following two auxiliary results will be important later. 
The first one enables us to approximate any
compact geodesic space by embedded graphs. For a proof see \cite{}.

\begin{proposition}\label{prop:app1}
 Let $Z$ be a compact geodesic space. Then for every $\epsilon>0$ there is a 
 finite graph $\Gamma$, embedded into $Z$, such that when $\Gamma$ is equipped with 
 the induced intrinsic metric, then $|Z,\Gamma|_{GH}<\epsilon$.
 Moreover, one can choose $\Gamma$ such that every edge is a geodesic in $Z$
 and such that for any given $\delta>0$ the vertices of $\Gamma$ form a $\delta$-net
 in $Z$.
\end{proposition}

The second one allows us, in the case of closed discs, to adjust the approximating graphs 
such that complementary regions have small diameter. It is a consequence ot the following cutting
lemma.

\begin{lemma}[Cutting discs]\label{lem:cut}
Let $Z$ be a geodesic space which is homeomorphic to a closed disc. 
The space $Z$ is contained in the tubular $\delta$-neighbourhood
 $N_\delta(\partial Z)$ of its boundary for some $\delta>0$. The boundary 
 $\partial Z$ is rectifiable with $\len(\partial Z)=L$.
 Then there is a geodesic in $Z$ which cuts $Z$ into two discs $Z^+$ and $Z^-$, 
 such that $\len(\partial Z^\pm)\leq\frac{3}{4}L+2\delta$.
\end{lemma}

\begin{proof}
 Divide $\partial Z$ into four paths $\alpha_i$ of equal lentgh, 
 and such that $\alpha_i$ is adjacent to $\alpha_{i+1}$.
 If for all $i$ the tubular $\delta$-neighbourhoods of $\alpha_i$ and $\alpha_{i+2}$
 are disjoint, then $\partial Z$ would not be contractible in $Z$. Indeed, the set
 $N_\delta (\alpha_i)\cap N_\delta (\alpha_{i+1})$ would be locally separating in $Z$. 
 Since it also separates the concatenation $\alpha_i*\alpha_{i+1}$, any path homotopic to 
 $\alpha_i*\alpha_{i+1}$ would have to intersect $N_\delta (\alpha_i)\cap N_\delta (\alpha_{i+1})$.
\end{proof}

Repeated application of the above lemma yields

\begin{proposition}\label{prop:app2}
 Let $Z$ be a closed disc with a compatible length metric and rectifiable boundary.
 Assume that for some $\delta>0$ the space $Z$ is contained in the tubular $\delta$-neighbourhood
 of its boundary $N_\delta(\partial Z)$. Then there is an embedded 
 finite graph $\Gamma\hookrightarrow Z$ with $\partial Z\subset \Gamma$, such that each component
 of $Z\setminus \Gamma$ has diameter less than $7\delta$.
\end{proposition}


\begin{proof}
 Assume that $\gamma\subset\Gamma$ is a Jordan curve in $Z$ of length $\leq n\cdot\delta$ which 
 surrounds a region $M$.
 By Lemma \ref{lem:cut}, we can subdivide $M$ into two domains $M^+$ and $M^-$ with 
 $\len(\partial M^\pm)\leq (\frac{3}{4}n+2)\delta$. By iterated cutting, we obtain 
 domains with circumferences at most a little over $8\delta$ and since every point
 in $Z$
 has distance less than $\delta$ from the boundary, the claim follows.
\end{proof}




\section{Intrinsic geometry}

We turn to our main result.


\begin{theorem}[Petrunin]
Let $X$ be a CAT(0) space and $f:\DD\rightarrow X$ an \ic\ map. If $f$ is metric minimizing and 
$f|_{\partial \DD}$ is an embedding, then $D_f$ is a CAT(0) disc.
\end{theorem}

\begin{proof}
We have the commutative diagram

\begin{displaymath}
    \xymatrix{ \DD \ar[r]^f\ar[d]^{\pi_f} & X  \\
               D_f \ar[ru]_{\bar f} }
\end{displaymath}

where $D_f$ is a closed disc by Lemma \ref{lem:inttop} and $\bar f$ is a length preserving map. 

Let $\epsilon>0$ and $\delta\ll\epsilon$.
By Propositions \ref{prop:app1} and \ref{prop:app2}
we can find a finite graph $\Gamma$, embedded into $D_f$, such that 
every edge is a geodesic of length $\leq 2\delta$,
every interior region has diameter $\leq 7\delta$ and $|D_f,\Gamma|_{GH}<\epsilon$ .
Using part (a) of Lemma \ref{lem:graphs} we replace $\bar f|_\Gamma$ by a minimal element 
$s_\epsilon\in\mathcal{F}(\bar f|_\Gamma)$, which is a map defined on a finite graph 
$\Gamma^\epsilon$. (See the discussion following Lemma \ref{lem:graphs}).
Extend $s_\epsilon$ to a map $S_\epsilon:D_f^\epsilon\to X$ via ruled surfaces
on every interior region. Denote $D_\epsilon$ the closed disc $D_f^\epsilon$ equipped with
the induced metric by $S_\epsilon$. Alexandrov's lemma on ruled surfaces in combination with 
part (c) of Lemma \ref{lem:graphs} shows that $D_\epsilon$ is a CAT(0) disc.
Note that the natural quotient map $\pi_\epsilon:D_f\to D_\epsilon$ is not short, but its restriction
to $\Gamma_\epsilon$ is.

Choose a sequence $\epsilon_i\to 0$. Then the sequence $S_{\epsilon_i}\circ\pi_{\epsilon_i}\circ\pi_f$
subconverges to a map $S_\infty:\DD\to X$ and the sequence $\pi_{\epsilon_i}$ subconverges to a short map
$\pi_\infty:D_f\to D_\infty$ where $D_\infty$ is $\DD$ equipped with the induced metric of $S_\infty$.
Since $S_\infty|_{\partial \DD}=f|_{\partial \DD}$ and $f$ is metric minimizing, 
$\pi_\infty$ is an isometry. As $D_\infty$ is a limit
of CAT(0) spaces, the theorem is proven.


\end{proof}





































\end{document}