\documentclass{article}
\usepackage{metric-min}
%\usepackage{showkeys}

\begin{document}

\title{Metric minimzing surfaces revisited}
\author{Anton Petrunin and Stephan Stadler}
\nofootnote{A.~Petrunin was partially supported by NSF grant DMS 1309340.
S.~Stadler was supported by DFG grants STA 1511/1-1 and SPP 2026.}

\newcommand{\Addresses}{{\bigskip\footnotesize
Anton Petrunin, \par\nopagebreak\textsc{Department of Mathematics, PSU, University Park, PA 16802, USA}
\par\nopagebreak
\textit{Email}: \texttt{petrunin@math.psu.edu}

\medskip
 
Stephan Stadler,
\par\nopagebreak\textsc{Mathematisches Institut der Universit\"at M\"unchen, Theresienstr. 39, D-80333 M\"unchen, Germany}
\par\nopagebreak
\textit{Email}: \texttt{stadler@math.lmu.de}
}}

\date{}

%%%%%%%???to do: def cat(0) disc retract, all graphs are connected

\maketitle

\begin{abstract}
A surface which does not admit a length nonincreasing deformation is called \emph{metric minimizing}.
We show that metric minimizing surfaces in $\CAT[0]$ spaces are locally $\CAT[0]$ with respect to their intrinsic metric. 
\end{abstract}

\section{Introduction}

Let us denote by $\DD$ the closed unit disc in the plane.
Its boundary $\partial \DD=\SS^1$ is the unit circle.

Let $Y$ be a metric space
and $s\:\DD\to Y$ be a continuous map.
Consider the induced length-pseudometric on $\DD$ defined as 
\[\<x-y\>_s=\inf_\alpha\{\length s\circ\alpha\},\]
where the greatest lower bound is taken over all paths $\alpha$ from $x$ to $y$ in $\DD$.
The distance $\<x-y\>_s$ might take infinite values.
We denote by $\<\DD\>_s$ the corresponding metric space;
see the next section for a precise definition.

The space $\<\DD\>_s$ comes with the projection $ \hat\pi_s\:\DD\to\<\DD\>_s$;
its restriction to $\SS^1=\partial\DD$ will be denoted by $\delta_s$.

Let $\gamma\:\SS^1\to Y$ be a closed rectifiable curve.
We say that $s\:\DD\to Y$ \emph{spans} $\gamma$,
if $\delta_s=\gamma$.

Assume that $s$ and $s'$ are two maps spanning the same curve.
We write $s\succcurlyeq s'$ if there is a \emph{majorization}
$\mu\:\<\DD\>_s\to \<\DD\>_{s'}$;
meaning that $\mu$ is a \emph{short} map with $\delta_{s'}=\mu\circ\delta_s$.

A map $s\:\DD\to Y$ will be called \emph{metric minimizing disc} if $s\succcurlyeq s'$ 
always implies that the corresponding majorization $\mu$ is an isometry.

\begin{wrapfigure}{r}{28 mm}
\begin{lpic}[t(-2 mm),b(-0 mm),r(0 mm),l(0 mm)]{pics/two-bry-curves(1)}
\end{lpic}
\end{wrapfigure}

A topological space $W$ together with a choice of a closed curve $\delta\:\SS^1\zz\to W$ is called \emph{disc retract} 
if the mapping cylinder of~$\delta$ 
\[W_\delta=W\bigsqcup_{\delta(u)\sim(u,0)}\SS^1\times[0,1]\]
is homeomorphic to the disc $\DD$.
The curve $\delta$ will be called \emph{boundary curve} of the disc retract $W$.
Note that a disc retract (for example the one shown on the picture) may have essentially different boundary curves,
in the sense that one is not a reparametrization of the other.

\begin{thm}{Main theorem}
Assume $Y$ is a $\CAT[0]$ space and $s\:\DD\to Y$ is a metric minimizing map.
Then $\<\DD\>_s$ is a $\CAT[0]$ disc retract with boundary curve $\delta_s$.
\end{thm}

Note that we did not make any assumptions on the continuous map $s$;
in particular, before hand, the space $\<\DD\>_s$ might have wild topology.

If we remove the condition that the boundary curve is rectifiable, then the space $\<\DD\>_s$ might have points of 
infinite distance from each other. However, our proof shows that all triangles with finite sides in $\<\DD\>_s$ are still thin; 
in particular each metric component of $\<\DD\>_s$ is a $\CAT[0]$ space.
In fact one can consider $\<\DD\>_s$ as a $\CAT[0]$ space where infinite distances between points are legal.

In the last section, we discuss the relation between saddle surfaces and metric minimizing discs.
Evidently smooth metric minimizing disc $s$ in a Euclidean space is saddle, we prove 
a local converse of this statement in dimension three (Proposition~\ref{prop:smooth}) and show by example that there is no global converse.
We expect that in the 4-dimensional case even the local converse does not hold.

The main theorem generalizes theorem of Alexander Alexandrov about ruled surfaces in \cite{A} and the main theorem by fist author in \cite{petrunin-metric-min}.
It is also closely related to the results of Samuel Shefel in \cite{shefel-2D} and \cite{shefel-3D}, see also \cite[Chapter 4]{akp} and \cite{petrunin-stadler}. 

\parbf{Acknowledgement.}
We want to thank 
Sergei Ivanov, 
Carlo Sinestrari, 
Peter Topping 
and Burkhard Wilking 
for help.
We also would like to thank Alexander Lytchak for explaining his recent work with Stefan Wenger and for several helpful discussions.


\section{Definitions}

\parbf{Metrics and pseudometrics.}
Let $X$ be a set.
A \emph{pseudometric} on $X$ 
is a function $X\times X\to[0,\infty]$ denoted as $(x,y)\mapsto |x-y|$
such that 
\begin{itemize}
\item $|x-x|=0$, for any $x\in X$;
\item $|x-y|=|y-x|$, for any $x,y\in X$;
\item $|x-y|+|y-z|\ge|x-z|$ for any  $x,y,z\in X$.
\end{itemize}
If in addition $|x-y|=0$ implies $x=y$ then the pseudometric $|{*}-{*}|$ is called a \emph{metric}; some authors prefer to call it \emph{$\infty$-metric} to emphasize that the distance between points might be infinite.
The value $|x-y|$ will also be called \emph{distance} form $x$ to $y$.

For any pseudometric on a set there is an equivalence relation ``$\sim$'' such that 
\[x\sim y\quad\iff\quad|x-y|=0.\]
The pseudometric induced  on the set of equivalence classes 
\[[x]=\set{x'\in X}{x\sim x}\] becomes a metric.
The obtained metric space will be denoted as $[X]$;
it comes with the projection map $X\to [X]$ defined as $x\mapsto [x]$.


For a metric space we can consider the equivalence relation ``$\approx$'' defined as 
\[x\approx y\quad\iff\quad|x-y|<\infty.\]
Its equivalence classes are called \emph{metric components}.
Note that by definition each metric component is a \emph{genuine metric space}, meaning that distance between points are finite.
Consequently, a metric space is a disjoint union of genuine metric spaces.

\parbf{Pseudometrics induced by maps.}
Assume $X$ is a topological space and $Y$ is a metric space.
Let $f\:X\to Y$ be a continuous map. 

Let us define the \emph{length pseudometric} on $X$ induced by $f$ as
\[\< x-y\>_f
=
\inf
\set{\length(f\circ\gamma)_Y}{\gamma\ \text{a path in}\  X\ \text{connecting}\ x\ \text{and}\ y}.\]
Denote by $\< X\>_f$ the corresponding metric space; that is 
\[\< X\>_f=[(X,\<{*}-{*}\>_f)].\] 

Similarly, define a \emph{connecting pseudometric} $|{*}-{*}|_f$ on $X$ in
the following way
\[|x-y|_f=\inf\{\diam f(K)\},\]
where the greatest lower bound is taken over all connected sets $K\subset X$ which contain $x$ and $y$;
if there is no such set we set $|x-y|_f=\infty$. 
The associated metric space will be
denoted as $|X|_f$;
that is 
\[| X|_f=[(X,|{*}-{*}|_f)].\]

For the projections 
\[\bar \pi_f\:X\to |X|_f\quad\text{and}\quad \hat \pi_f\:X\to \<X\>_f\]
we will also use the shortcut notations 
\[\bar x=\bar\pi_f(x) \quad\text{and}\quad  \hat x= \hat \pi_f(x).\]

\begin{thm}{Lemma}\label{lem:picont}
Let $X$ be a locally connected topological space and $Y$ a metric space. Assume that $f\:X\to Y$ is a continuous map. 
Then $\bar\pi_f\:X\to|X|_f$ is continuous.

In particular, if in addition $X$ is compact then so is $|X|_f$.
\end{thm}

\parit{Proof.}
For a point $x\in X$ and $\eps>0$ we denote by $U$ the connected component of $f^{-1}[B(f(x),\eps)_{Y}]$ which contains $x$.
Since $X$ is locally connected, the set $U$ is open.

Note that $\bar\pi_f(U)\subset B(\bar x,2\cdot \eps)_{|X|_f}$. Hence
$\bar\pi_f$ is continuous.
\qeds



\begin{wrapfigure}{r}{37 mm}
\begin{tikzpicture}[scale=1.5]

  \node[main node] (1) at (1,.5) {$|X|_f$};
  \node[main node] (2) at (1,1.5){$\<X\>_f$};
  \node[main node] (11) at (2,2){$Y$};
  \node[main node] (12) at (0,2) {$X$};
\draw[
    >=latex,
%   every node/.style={above,midway},% either
    auto=right,                      % or
    loop above/.style={out=75,in=105,loop},
    every loop,
    ]
   (2) edge node[right]{$\tau_f$}(1)
   (12) edge[bend left] node[above]{$f$}(11)
   (12) edge node[below]{$ \hat \pi_f$}(2)
   (2) edge node[below]{$ \hat f$}(11)
   (12) edge[bend right] node[left below]{$\bar \pi_f$}(1)
   (1) edge[bend right] node[below right]{$\bar f$}(11);
\end{tikzpicture}
\end{wrapfigure}

Note that $\tau_f\: \hat x\to \bar x$ defines a map $\tau_f\: \<X\>_f\zz\to |X|_f$ and by construction it is length-preserving.
Since $\<X\>_f$ is a length space, the latter implies that $\tau_f$ short.
The map $\tau_f$ might not induce an isometry
\[\<X\>_f\to\<|X|_f\>_{\bar f}.\]
Moreover, $\tau_f$ does not have to be injective, an example is given in \cite[4.2]{petrunin-intrinisic}.
However, for metric minimizing discs $f\:\DD\to Y$ both statements hold true; see
Proposition~\ref{prop:|D|}.

The maps $\bar f\:|X|_f\to Y$ and $ \hat f\:\<X\>_f\to Y$ are uniquely defined by the identity
\[f(x)=\bar f(\bar x)=  \hat f( \hat x)\] for any $x\in X$.
By construction, the diagram commutes.

\parbf{Metrics induced by metrics.}
%???formally this par is irrelevant, but I think to include it here to provide a better picture 
If $X$ is a metric space, the two constructions above can be applied to the identity map $\id\:X\to X$.
In this case the obtained spaces $\<X\>_\id$ and $|X|_\id$ will be denoted by $\<X\>$ and $|X|$ correspondingly.
The space $\<X\>$ is $X$ equipped with induced intrinsic metric.
All three spaces $\<X\>$, $|X|$ and $X$ have the same underlying set;
in other words they can be considered as a single space with different metrics and tautological maps between them.
Both tautological maps 
\[\<X\>\to |X|\to X\]
are short and length-preserving.

In particular the tautological map $\<|X|_f\>_{\bar f}\zz\to \<|X|_f\>$ is an isometry;
that is, for any continuous map $f\:X\to Y$ the induced length metric on $|X|_f$ coincides with the length metric induced by 
$\bar f\:|X|_f\to Y$.


Recall that a geodesic in a metric space is a curve whose length coincides with the distance between its endpoints.
A metric space is called \emph{geodesic} if any two points at finite distance can be joined by a geodesic.

\begin{thm}{Lemma}\label{lem:geospace}
Let $X$ be a compact metric space. 
Then $\<X\>$ is a complete geodesic space.
\end{thm}

The second statement is classical (see for example \cite[II-\S8 Thm. 3]{KF}) but 
we were not be able to find the first one in the literature.

\parit{Proof.}
Assume contrary $\<X\>_f$ is not complete.
Fix a sequence $(x_n)$ in $\<X\>_f$ converging in itself, but not converging in $\<X\>_f$.
After passing to a subsequence, we can assume that the points of the sequence appear on a rectifiable curve $\hat\gamma\:[0,1)\to\<X\>$ in the same order.

The corresponding curve $\gamma\:[0,1)\to X$ has the same length.
Since $X$ is compact we can extend it to a path $\gamma_+\:[0,1]\to X$.
The curve 
\[\hat\gamma_+=\hat\pi\circ\gamma_+\:[0,1]\to\<X\>\]
has the same length.
Therefore $\hat\gamma_+(1)$ is the limit of $(x_n)$, a contradiction.

It remains to show that $\<X\>$ is geodesic.
Assume $\gamma_n$ is a sequence of constant speed paths from $x$ to $y$ in $X$
such that $\length(\hat\gamma_n)\to \<x-y\>$ as $n\to\infty$.
Since $X$ is compact, we can pass to a partial limit $\gamma$ of  $\gamma_n$.
The corresponding curve $\hat\gamma=\hat \pi\circ\gamma$ is the needed geodesic from $\hat x$ to $\hat y$ in $\<X\>$.
\qeds

\parbf{Monotone-light factorization.} %???any reason you prefer decomposition???
Let $f\:X\to Y$ be a map between topological spaces.
Recall that 
\begin{itemize}
\item $f$ is called \emph{monotonic} if the inverse image of each point is connected,
 \item $f$ is called \emph{light} if the inverse image of any point totally disconnected.
\end{itemize}
Since a connected set is nonempty by definition, any monotonic map is onto.

\begin{thm}{Lemma}\label{cor:fiberconnected}
Assume $X$ is a locally connected compact metric space and $Y$ is a metric space.
Let $f\:X\to Y$ be a continuous map.
Then the map $\bar \pi_f$ is monotonic and $\bar f$ is light.
In particular 
\[f=\bar f\circ\bar\pi_f\]
is a monotone-light factorization. 
\end{thm}

\parit{Proof.}
First we prove the monotonicity of $\bar\pi_f$.

Assume the contrary;
that is, for some $x\in X$ the equivalence class 
\[K=\bar\pi_f^{-1}(\bar x)=\set{x'\in X}{|x-x'|_f=0}\]
is not connected. Since $X$ is normal, we
can cover $K$ by disjoint open sets $U,V\subset X$ such that both intersections
$K\cap U$ and $K\cap V$ are nonempty.

By Lemma~\ref{lem:picont}, $K$ is closed.

Assume that $x\in U$ and pick $x'\in K\cap V$.
Then there is a sequence of connected sets $K_n\ni x,x'$ such that $\diam f(K_n)<\tfrac1n$.
For each $n$ we choose a point $k_n\in K_n\backslash (U\cup V)$.
Let $k$ be a partial limit of the sequence $(k_n)$.
It follows that $k\in K\backslash (U\cup V)$, a contradiction. 

Assume $\bar f$ is not light;
that is, the inverse image of some $y\in Y$ contains a closed connected set $C$ with more than one point.  
Note that the inverse image $Z:=\bar\pi_f^{-1}C$ is connected. 
Indeed, if $Z$ is the disjoint union of two compact sets $Z_1$ and $Z_2$, then the sets 
$C_i=\{\,\bar x\in C\mid\bar\pi_f^{-1}(\bar x)\subset Z_i\,\}$ are closed.
It follows that $|z-z'|_f=0$ for any two points $z,z'\in Z$, a contradiction.
\qeds

\section{Disc retracts}\label{Metric minimizing discs}

\begin{thm}{Proposition}\label{prop:|D|}
Let $s\:\DD\to\ Y$ be a metric minimizing disc.
Then $|\DD|_s$ is a disc retract with boundary curve $\delta_s=\bar\pi_s|_{\SS^1}$.
Moreover the map $\tau_s\:\<\DD\>_s\to |\DD|_s$ is injective and defines an isometry
$\<\DD\>_s\to \<|\DD|_s\>$;
that is $\tau_s$ is an isometry from $\<\DD\>_s$ to $|\DD|_s$ equipped with induced intrinsic metric.
\end{thm}

Let $Y$ be a metric space and
$s\:\DD\to Y$ be a continuous map.
We say that $s$ has \label{page:no-bubble}\emph{no bubbles}
if for any point $p\in Y$ every connected component of the complement $\DD\backslash s^{-1}\{p\}$ contains a point from $\partial \DD$.

\begin{thm}{Lemma}\label{prop:point-complement}
Let $Y$ be metric spaces and $s\:\DD\to Y$ be a metric minimizing map.
Then $s$ has no bubbles.
\end{thm}

\parit{Proof.}
Assume the contrary;
that is, there is $y\in Y$ such that the complement $\DD\backslash s^{-1}(y)$ contains a connected component $\Omega$ such that $\partial \DD\cap \Omega=\emptyset$.

Let us define the new map $s'\:\DD\to\ Y$ by setting $s'(z)=y$ for any $x\in \Omega$ and $s'(x)=s(x)$ for any $x\notin \Omega$.

By construction $s'$ and $s$ agree on $\partial\DD$ and moreover $s\succcurlyeq s'$.

Note that
\[\<x-x'\>_{s}>0=\<x-x'\>_{s'}\]
for a pair of distinct points $x,x'\in \Omega$, a contradiction.
\qeds



\begin{thm}{Lemma}\label{prop:disc-moore}
Let $Y$ be a metric space and a map $f\:\DD\to Y$ has no bubbles.
Then $|\DD|_f$ is homeomorphic to a disc retract with boundary curve $\bar\pi_f|_{\SS^1}$.
%Moreover, if $f|_{\partial\DD}$ is an embedding, then $|\DD|_f$ is homeomorphic to a disc.
\end{thm}

The proposition above is a disc version of Moore's theorem \cite{moore} proved in \cite{LW3}.

\parit{Proof.}
Form Lemma \ref{lem:picont} we know that $\bar\pi_f$ is continuous and hence $|\DD|_f$
is a compact metric space. 
The fibers of $\bar\pi_f$ are connected by Corollary \ref{cor:fiberconnected}.
Since $f$ is a no-bubble map, we see that $\bar\pi_f$ is cell-like. 
Therefore, the claim follows from \cite[Corollary 7.12]{LW3}.
%If the restriction $f|_{\partial\DD}$ is an embedding, then $|\DD|_f$ is homeomorphic to a disc by Schoenflies theorem.
\qeds

\parit{Proof of Proposition~\ref{prop:|D|}.}
The first two statement follows from the Lemmas.

Since $|\DD|_s$ is a disc retract, the mapping cylinder over the boundary curve of $|\DD|_s$ is homeomorphic to $\DD$.
Denote by $\theta\:\DD\to |\DD|_s$ the projection.

Note that $\<\DD\>_{\bar s\circ\theta}$ is isometric to $|\DD|_s$ equipped with the induced intrinsic metric.
Recall that the map $\tau_s\:\<\DD\>_s\to|\DD|_s$ is short;
in particular the induced map $\<\DD\>_s\to\<\DD\>_{\bar s\circ\theta}$ is also short.
Therefore $s\succcurlyeq \bar s\circ\theta$;
since $s$ is metric minimizing the statement follow.
\qeds

\section{Compactness of planar CAT[0] spaces}\label{Compactness}

A sequence of pairs $(X_n,\gamma_n)$, where $X_n$ is a metric space and $\gamma_n\:\SS^1\to X_n$ is a 
closed curve is said to \emph{converge} to $(X_\infty,\gamma_\infty)$ if there is a convergence of $X_n$ to $X_\infty$ 
in the sense of Gromov--Hausdorff for which $\gamma_n$ converges to $\gamma_\infty$ pointwise.

More precisely, we ask the following.
\begin{enumerate}[(1)]
	\item There is a metric $\rho$ on the disjoint union 
\[\bm{X}=X_\infty\sqcup X_1\sqcup X_2\dots\]
which restricts to the given metric on each $X_\alpha, \alpha\in\{1,2,\dots,\infty\}$, 
and such that $X_n$ converge to $X_\infty$ in the sense of Hausdorff as subsets in $(\bm{X},\rho)$.
\item  The sequence of compositions $\gamma_n\:\SS^1\to X_n \hookrightarrow\bm{X}$ 
converges to $\gamma_\infty\:\SS^1\zz\to X_\infty \hookrightarrow\bm{X}$ pointwise.
\end{enumerate}
Consider the class $\mathcal{K}_\ell$ of $\CAT[0]$ disc retracts whose marked
boundary curves have Lipschitz constant $\ell$.


\begin{thm}{Compactness lemma}\label{lem:compact}
$\mathcal{K}_\ell$ is compact in the topology described above.
\end{thm}

The lemma follows from the two lemmas below.

\begin{thm}{Lemma}\label{lem:precompact}
$\mathcal{K}_\ell$ is precompact in the topology described above.
\end{thm}

\parit{Proof.}
Let $K$ be a metric space with the isometry class in $\mathcal {K}_\ell$.

Denote by $\area A$ the two-dimensional Hausdorff measure of $A\subset K$.
By the Euclidean isoperimetric inequality we have 
\[\area K \le \pi\cdot\ell^2.\]

Fix $\eps>0$. 
Set $m=\lceil 10\cdot\tfrac\ell\eps\rceil$.
Choose $m$ points $y_1,\dots,y_m$ on $\partial K$
which divide $\partial K$ into arcs of equal length.

Consider the maximal set of points $\{x_1,\dots,x_n\}$ such that $d(x_i,x_j)>\eps$ and $d(x_i,y_j)>\eps$.

Note that the set $\{x_1,\dots,x_n,y_1,\dots,y_m\}$
is an $\eps$-net in $(K,d)$.
Further note that the balls $B_i=B_{\eps/2}(x_i)$
do not overlap.

By comparison,
\[\area B_i\ge \tfrac{\pi\cdot\eps^2}{4}.\]
It follows that $n\le 4\cdot\left(\tfrac\ell\eps\right)^2$.
In particular, there is a integer valued function $N(\eps)$, such that any  
$K$ as above contains an $\eps$-net
with at most $N(\eps)$ points.

The latter means that the class $\mathcal{K}_\ell$ is uniformly totally bounded.
By the selection theorem \cite[7.4.15]{BBI}, the class of metrics with this property is precompact in the Gromov--Hausdorff topology.

Since the set of $\ell$-Lipschitz maps defined on $\SS^1$ with compact target is compact 
with respect to pointwise convergence, we conclude that $\mathcal{K}_\ell$ is precompact in the topology defined above. 
\qeds





\begin{thm}{Lemma}\label{lem:closed}
$\mathcal{K}_\ell$ is closed in the topology described above.
\end{thm}

\parit{Proof.}
Let $(X_n,\gamma_n)$ be a sequence in $\mathcal{K}_\ell$.
Assume $X_n\to X$ and $\gamma_n\to\gamma$. 
Choose a point $p_n\in X_n$ and define
$f_n:\DD\to X_n$ by sending the geodesic $[0,\theta]$ for $\theta\in\partial \DD=\SS^1$ to the geodesic path $[p_n,\gamma_n(\theta)]$ with constant speed. 

By comparison, $f_n$ is $(5\cdot\ell)$-Lipschitz. %??? explain
The limit map $f\:\DD\to X$ is also $(5\cdot\ell)$-Lipschitz.

Note that if two different points $x$  and $y$ map to the same point $q$ under $f$, then $q$ is a branch point in $X$ 
and there is an arc in $\DD$ connecting 
$x$ and $y$ which also maps to $q$. 
In particular, $f$ is monotone.

On the other hand, the map $f$ has no bubbles.
This follows since in the mapping cylinder $X_\gamma$, the concatenation of the geodesic path $[p,\gamma(\theta)]$ 
with the vertical line $\theta\times [0,1]$ can be lifted to $\DD$.

Therefore, by Lemma~\ref{prop:disc-moore}, $X$ is a disc retract with the boundary curve $\gamma$.
\qeds

\section{Metric minimizing graphs}\label{Metric minimizing graphs}

\emph{Metric minimizing graphs} are defined analogously to metric minimizing discs.

Namely, let $Y$ be a metric space, $\Gamma$ be a finite graph and $A$ be a subset of its vertexes.
Given two maps $f,f'\:\Gamma\to Y$, we write $f\succcurlyeq f'\rel A$ if $f$ and $f'$ agree on $A$ 
and there is a majorization $\mu\:\<\Gamma\>_f\to \<\Gamma\>_{f'}$
such that $f(a)\zz=f'(\mu(a))$ for any $a\in A$.

A map $f\:\Gamma\to Y$ is called \emph{metric minimizing relative to $A$} if $f\succcurlyeq f'\rel A$ implies that the majorization $\mu$ is an isometry.

\begin{thm}{Proposition}\label{prop:metric-min-graph-exist}
Let $Y$ be a $\CAT(0)$ space, 
$\Gamma$ be a finite graph and $A$ be a subset of its vertexes.

Given a continuous map $f\:\Gamma\to Y$ there is a map $h\:\Gamma\to Y$ 
that is metric minimizing relative to $A$ and $f\succcurlyeq h\rel A$.
\end{thm}

\parit{Proof.}
Let us parametrize each edge of $\Gamma$ by $[0,1]$.
A map $h\:\Gamma\to Y$ will be called \emph{straight} if it
sends each edge of $\Gamma$ to a constant-speed geodesic path in $Y$.

If $h\:\Gamma\to Y$ is straight, then $f\succcurlyeq h\rel A$ if and only if 
\[|f(v)-f(w)|_Y\ge |h(v)-h(w)|_Y\]
for any two adjacent vertexes $v$ and $w$ in $\Gamma$.
In particular, we can assume that the given map $f$ is straight.

By finiteness of the number of vertexes and Zorn's lemma,
it is sufficient to prove that for any ordered sequence of straight maps $f_1\succcurlyeq f_2\succcurlyeq \dots$ there exists a map $f\preccurlyeq f_n$ for all $n$.

Assume contrary; let us apply the \emph{ultralimit+projection construction}.

Namely, fix an ultrafilter $\omega$; denote by $Y^\omega$ the ultrapower of $Y$. 
Then $Y^\omega$
is a $\CAT(0)$ space that contains $Y$ as a closed convex subset. 
The $\omega$-limit $f_\omega\:\Gamma\to Y^\omega$ is well defined, since all $f_n$ are Lipschitz continuous. 
Denote $f'$ the composition of $f_\omega$ with the nearest point projection $Y^\omega\to Y$.
The nearest point projection to a closed convex set in $\CAT(0)$ space is short.
Therefore $f_n\succcurlyeq f'$ for any $n$.
Let $f''$ denote the straightening of $f'$.
Then $f'\succcurlyeq f''$ and the claim follows.
\qeds


\begin{thm}{Proposition}\label{prop:metric-min-graph}
Let $Y$ be a $\CAT(0)$ space, 
$\Gamma$ be a finite  graph and $A$ be a subset of its vertexes.
Assume that the assignment $v\mapsto v'$ is a metric minimizing map $\Gamma\to Y$ relative to $A$.
Then
\begin{enumerate}[(a)]
\item each edge of $\Gamma$ maps to a geodesic;
\item\label{prop:metric-min-graph:x} for any vertex $v\notin A$ and any $x\ne v'$
there is an edge  $[v,w]$ in $\Gamma$ such that
$\measuredangle[v'\,^{w'}_x]\ge \tfrac\pi2$;
\item\label{sum>=2pi} for any vertex $v\notin A$ and any cyclic order $w_1,\dots,w_n$ of adjacent vertexes we have
\[\measuredangle[v'\,^{w'_1}_{w'_2}]+\dots+\measuredangle[v'\,^{w'_{n-1}}_{w'_n}]+\measuredangle[v'\,^{w'_n}_{w'_1}]\ge 2\cdot\pi.\]
\end{enumerate}
\end{thm}

\begin{wrapfigure}{r}{22 mm}
\begin{lpic}[t(-8 mm),b(-0 mm),r(0 mm),l(0 mm)]{pics/not-sufficient(1)}
\end{lpic}
\end{wrapfigure}

\parit{Remark.}
The conditions in the proposition do not guarantee that the map $f$ is metric minimizing.
An example can be guessed from the diagram, where the solid points form the set~$A$. 


\parit{Proof.}
The first condition is evident.

Assume the second condition does not hold at a vertex $v\zz\notin A$;
that is, there is a point $x\in Y$ such that
$\measuredangle[v'\,^{w'}_x]< \tfrac\pi2$
for any adjacent vertex $w$.
In this case moving $v'$ toward $x$ along $[v',x]$ decreases the lengths of all edges adjacent to $v$, a contradiction.

Assume the third condition does not hold; 
that is, the sum of the angles around a fixed interior vertex $v'$ is less than $2\cdot\pi$.

Recall that the space of directions $\Sigma_{v'}$ is a $\CAT(1)$ space.
Denote by $\xi_1,\dots,\xi_n$ the directions of $[v',w'_1],\dots, [v',w'_n]$ in $\Sigma_{v'}$.
By assumption, we have
\[|\xi_1-\xi_2|_{\Sigma_{v'}}+\dots+|\xi_n-\xi_1|_{\Sigma_{v'}}<2\cdot\pi.\]
By Reshetnyak's majorization theorem,
the closed broken line $[\xi_1,\dots,\xi_k]$ is majorized by a convex spherical polygon $P$.

Note that $P$ lies in an open hemisphere with pole  at some point in $P$.
Choose $x\in Y$ so that the direction from $v'$ to $x$ coincides with the image of the pole in $\Sigma_{f(v)}$.
This choice of $x$ contradicts (\emph{\ref{prop:metric-min-graph:x}}).
\qeds
\section{Key lemma}\label{Key Lemma}


\begin{thm}{Lemma}\label{lem:graph}
Let $Y$ be a $\CAT(0)$ space and $s\:\DD\to Y$ 
be a metric-minimizing disc.
Assume $F\subset \DD$ is a finite set such that $\hat\pi_s(F)$ has finite
diameter in $\<\DD\>_s$.
Then there exists a finite piecewise geodesic graph $\Gamma$ embedded in $\<\DD\>_s$ that contains a geodesic between any pair of points in $\hat\pi_s(F)$.
\end{thm} 

\parit{Proof.} 
For any pair $x,y\in F$, connect $\hat x$ to $\hat y$ by a minimizing geodesic in $\<\DD\>_s$. 
We can assume that the constructed geodesics 
are either disjoint or their intersection is formed by finite collections of arcs and points.

Indeed, if some number of geodesics $\gamma_1,\dots,\gamma_n$ already has this property and we are given points $x$ and $y$, then
we choose a minimizing geodesic $\gamma_{n+1}$ from $x$ to $y$ that maximizes the time it spends in $\gamma_1,\dots,\gamma_n$  
in the order of importance.
Namely, 
\begin{itemize}
\item  among all minimizing geodesics connecting $x$ to $y$
choose one that spends maximal time in $\gamma_1$ --- in this case, $\gamma_{n+1}$ intersects $\gamma_1$ along the empty set, 
a one-point set, or a closed arc.
\item among all minimizing geodesics as above
choose one that spends maximal time in $\gamma_2$ --- in this case, $\gamma_{n+1}$ intersects $\gamma_2$ along at most two arcs and points.
\item and so on.
\end{itemize}

It follows that together the constructed geodesics form a finite graph $\Gamma$ as required.
\qeds



\begin{thm}{Key lemma}\label{lem:key}
Let $Y$ be a $\CAT(0)$ space and $s\:\DD\to Y$ 
be a metric-minimizing disc.
Given a finite set $F\subset \DD$
there is 
\begin{enumerate}[(1)]
	\item a $\CAT(0)$ disc-retract $W$ with boundary curve $\delta$;
	\item a map $p\:F\to W$ such that
\[|p(x)-p(y)|_W\le \<x-y\>_s\] 
for $x,y\in F$ and $p(x)=\delta(x)$ for $x\in F\cap \partial\DD$;
  \item a short map $q\:W\to Y$ such that
\[s(x)=q\circ p(x)\] 
for any $x\in\partial\DD\cap F$.
\end{enumerate}
 
\end{thm} 

\parit{Proof.} If $\partial \DD\cap F= \emptyset$,
then one can take a one-point space as $W$ and arbitrary maps $p\:F\to W$ and $q\:W\to Y$.
So suppose $\partial \DD\cap F\ne\emptyset$.

Without loss of generality, we may assume that the distance $\<x-y\>_s$
between any pair of points $x,y\in F$ is finite.
Indeed, since the boundary curve $s|_{\partial\DD}$ is rectifiable,
this always holds for pairs of points in $\partial \DD\cap F$.
Consider the subset $F'\subset F$ that lies at finite $\<{*}-{*}\>_s$-distance from one (and therefore any) point in $\partial \DD\cap F$.
Suppose $p'\:F'\to W$ and $q\:W\to Y$ are maps satisfying the proposition for $F'$.
Extend $p'$ to $F$ by sending $F\backslash F'$ to one point in $W$.
The resulting map $p$ together with $q$ will then satisfy the proposition for $F$.

By Lemma~\ref{lem:graph}, there exists a finite piecewise geodesic graph $\Gamma$ embedded in $\<\DD\>_s$ that contains $F$ as a subset of its vertexes.
According to Proposition~\ref{prop:|D|},
 $\tau_s$ embeds $\Gamma$ in $|\DD|_s$.
By Proposition~\ref{prop:|D|},
$|\DD|_s$ is a disc-retract.
Therefore $\Gamma$ can be (and will be) considered as a graph embedded into the plane.

By Proposition~\ref{prop:metric-min-graph-exist}, there is a map 
$u\:\Gamma\to Y$ metric-minimizing relative to $A=F\cap\partial\DD$ such that
\[s|_\Gamma\succcurlyeq u\rel A.\eqlbl{eq:>=}\]

Fix an open disc $\Delta$ cut by $\Gamma$ from $|\DD|_s$.
By Reshetnyak's theorem, the closed curve $u|_{\partial\Delta}$
is majorized by a convex plane polygon, possibly degenerating to a point or a line segment.
Note that the angle of the majorizing polygon cannot be smaller than the angle between the corresponding edges in $u(\Gamma)\subset Y$.

Let us glue the majorizing polygons into $\<\Gamma\>_u$;
denote by $W$ the resulting space.
According to Proposition~\ref{prop:metric-min-graph}(\ref{sum>=2pi}), the angle around each inner vertex has to be at least $2\cdot\pi$.
Clearly, $W$ is a disc-retract;
in particular, it is simply connected.
It follows that $W$ is a $\CAT(0)$ space.

The short map $q\:W\to Y$ is constructed by gluing together the maps provided by Reshetnyak's majorization theorem.
The space $W$ comes with a natural short map $\<\Gamma\>_u\to W$.

Define $p(x)$ for $x\in F$ as the image of the corresponding vertex of $\Gamma$ in $W$.
By \ref{eq:>=}, 
\[|p(x)-p(y)|_W\le \<x-y\>_s\]
for any $x,y\in F$.

By construction, the pair of maps $p,q$ meet all conditions.
\qeds

The following establishes a connection between the 
key lemma (\ref{lem:key}) and the extension lemma (\ref{lem:finite-whole}).

\begin{thm}{Lemma}\label{lem:S-closed}
Let $Y$ be a $\CAT(0)$ space and $s\:\DD\to Y$ 
be a metric-minimizing disc. Let $W$ be a
$\CAT(0)$ disc-retract with boundary curve $\delta$. For a given 
finite set $F\subset \DD$ we define $\mathfrak{S}_F$ to be the family of maps  
 $p\:F\to W$ such that
\[|p(x)-p(y)|_W\le \<x-y\>_s\] 
for $x,y\in F$ with $p(x)=\delta(x)$ for $x\in F\cap \partial\DD$ and such that there exists
a short map $q\:W\to Y$ with
\[s(x)=q\circ p(x)\] 
for any $x\in\partial\DD\cap F$.
Then $\mathfrak{S}_F$ is closed under pointwise convergence. 
\end{thm}

The proof is an application of the ultralimit+projection construction;
we used it before and will use it again later.

\parit{Proof.}
Consider a converging sequence $p_n\in  \mathfrak{S}_F$;
denote by $p_\infty$ its limit.
For each $p_n$ there is a short map $q_n\:W\to Y$ satisfying the condition above.
Pass to its ultralimit $q_\omega\:W\to Y^\omega$.
Recall that $Y$ is a closed convex set in $Y^\omega$.
In particular, the nearest point projection $\nu\:Y^\omega\to Y$ is well-defined and short.
Therefore, the composition $q=\nu\circ q_\omega$ is short.
Finally note that the maps $p_\infty\:F\to W$ and $q\:W\to Y$ satisfy the condition above.
\qeds


\section{Extension lemma}\label{Finite-whole extension lemma}

\begin{thm}{Extension lemma}\label{lem:finite-whole}
Suppose that $X$ is a set 
and $Y$ is a compact topological space.
Assume that for any finite set $F\subset X$ 
a nonempty set $\mathfrak{S}_F$ of maps  $F\to Y$ is given, such that
\begin{itemize}
\item $\mathfrak{S}_F$ is closed under pointwise convergence;
\item for any subset $F'\subset F$ and any map $h\in \mathfrak{S}_F$
the restriction $h|_{F'}$ belongs to $\mathfrak{S}_{F'}$. 
\end{itemize}

Then there is a map $h\: X\to Y$ such that $h|_F\in \mathfrak{S}_F$ for any finite set $F\subset X$.
\end{thm}

\parit{Proof.}
Consider the space $Y^X$ of all maps $X\to Y$ equipped with the product topology.


By assumption, the sets $\mathfrak{S}_F\subset Y^X$ are closed and any finite interection of these sets is nonempty.

According to Tikhonov's theorem, $Y^X$ is compact.
By the finite intersection propery, the intersection $\bigcap_F\mathfrak{S}_F$ for all finite sets $F\subset X$ is nonempty.
Hence the stattement follows.
\qeds

Note that if $X$ and $Y$ are metric spaces and $A$ is a subset in $X$
then one can take as $\mathfrak{S}_F$ short maps $F\to Y$ which coincide with a given short map $A\to Y$ on $A\cap F$.
This way we obtain the folloing corollary; it is closely related to \cite[Proposition 5.2]{lang-shroeder}.
In a similar fashion, we will use the lemma in the proof of our main theorem.

\begin{thm}{Corollary}
Let $X$ and $Y$ be metric spaces, $A\subset X$ and $f\:A\to Y$ a short map.
Assume $Y$ is compact and for any finite set $F\subset X$ there is a short map $F\to Y$ which agrees with $f$ in $F\cap A$.
Then there is a short map $X\to Y$ which agrees with $f$ in $A$.
\end{thm}
\section{Proof assembling}\label{Main theorem}

The following lemma will be used in the final step in the proof of the main theorem.

\begin{thm}{Lemma}\label{lem:maj is isom}
Let $Y$ be a $\CAT[0]$ space and $s\:\DD\to Y$ be a metric minimizing map.
Assume that there is a $\CAT[0]$ disc retract $W$ with boundary curve $\delta$ and a short map $f\:\<D\>_s \to W$
such that $f\circ \delta_s=\delta$. If there exists a short map 
$q\: W\to Y$ with $q\circ \delta=s|_{\partial \DD}$, then the map $f$ is an isometry.
\end{thm}

\parit{Proof.}
Let $r\:\DD\to W$ be the projection from the mapping cylinder $\DD=W_\delta$. 
Note that $r$ is a retraction and $r|_{\partial \DD}=\delta$.
The composition $\DD\xrightarrow{r}W\xrightarrow{q} Y$ fulfills \[q\circ r|_{\partial \DD}=s|_{\partial \DD}.\]

Note that  $\<W\>_q=\<\DD\>_{q\circ r}$ and the natural projection $\rho\: W\to \<W\>_q$ is short.
It follows that $\rho\circ f\: \<\DD\>_s\to \<\DD\>_{q\circ r}$ is a majorization.
Since $s$ is metric minimizing, $\rho\circ f$ is an isometry. 

Therefore $f$ is an isometric embedding which contains $\delta$
in its image. 
By Lemma \ref{lem:geospace} and Proposition \ref{prop:|D|} $\<\DD\>_s$ is a complete geodesic space.
So $f$ has to be surjective and therefore an isometry.
\qeds

\parit{Proof of the main theorem.}
It is sufficient to consider
the case that $|\DD|_s$ is  homeomorphic to a disc.
Indeed, it is sufficient to show that a triangle $[xyz]$ in $\<\DD\>_s$ is thin. 
Once the above case has been proven, the closure of each connected open component of $\<\DD\>_s\backslash ([x,y]\cup[y,z]\cup[z,x])$
surrounded by the triangle is $\CAT[0]$.
From this, the thinness of the triangle $[xyz]$ follows; see \cite{bishop}.

%Since $|\DD|_s$ is homeomorphic to the disc, we can identify $\DD$ and $|\DD|_s$ and therefore we can assume that the map $s$ is light; that is, inverse image of any point is zero-dimensional.

Given a finite set $F\subset \DD$,
denote by $\mathcal{W}_F$
the set of isometry classes of spaces $W$ which meet the conditions of the Key Lemma~\ref{lem:key}
for $F$;
according to the Key Lemma~\ref{lem:key} $\mathcal{W}_F\ne\emptyset$.
Note that for two finite sets $F\subset F'$ in $\DD$,
we have $\mathcal{W}_F\supset \mathcal{W}_{F'}$.

According to the Compactness Lemma \ref{lem:compact} $\mathcal{W}_F$ is compact.
Therefore 
\[\mathcal{W}
=
\bigcap_{F}\mathcal{W}_F\ne \emptyset\]
where the intersection is taken over all finite subsets $F$ in $\DD$. 


Fix a space $W$ from $\mathcal{W}$;
the space $W$ is a $\CAT[0]$ disc retract,
such that given a finite set $F\subset \DD$ there is a map $h_F\:F\to W$ which is short with 
respect to $\<{*}-{*}\>_s$ 
and a short map $q_F\:W\to Y$ such that $q_F\circ h_F$ agrees with $s$ on $\partial\DD\cap F$.

Given a finite set $F\subset \DD$,
denote by $\mathfrak{S}_F$ the set of all maps $h_F\:F\to W$ described above.

By Lemma \ref{lem:S-closed}, $\mathfrak{S}_F$ is closed.
The condition on the restriction of $h_F\in  \mathfrak{S}_F$ in the finite-whole extension lemma (\ref{lem:finite-whole}) is evident.
That is, $\mathfrak{S}_F$ satisfies the assumption of the lemma.

Applying the finite-whole extension lemma \ref{lem:finite-whole},
we get a map $h\:\DD\to W$ such that $h|_F\in \mathfrak{S}_F$
for any finite set $F\subset \DD$.

Our next aim is to show that there is a single map $q$ such that
for all finite sets $F$ the composition $q\circ h|_F$ agrees with
$s$ on $\partial\DD\cap F$.
This is done by applying the ultralimit+projection construction.

Choose a sequence of finite sets $F_n$ such that the intersections $F_n\cap\partial \DD$ get denser and denser in $\partial \DD$, 
denote by $q_n$ the corresponding maps.
Let $q_\omega\:W\to Y^\omega$ be the ultralimit of $q_n$ and set $q=\nu\circ q_\omega$,
where $\nu\:Y^\omega\to Y$ is the closest-point-projection.
By construction $q\:W\to Y$ is short and $q\circ h$ agrees with $s$ on $\partial \DD$.
Note that we can't quite conclude $s\succcurlyeq q\circ h$ because $h$ might not be continuous.

By construction, the map $h$ induces a short map $\hat h\:\<\DD\>_s\to W$ 
such that $\hat h\circ\delta_s$ is the boundary curve of $W$.
By Lemma \ref{lem:maj is isom}, $\hat h$ is an isometry and the statement follows.
\qeds

\section{Saddle surfaces}\label{sec:smooth}

In this section we will discuss the relation between metric minimizing discs and saddle discs.

Recall that a map $s\:\DD\to \RR^m$ is called \emph{saddle} if for any hyperplane $\Pi\subset\RR^m$ each of the connected components of $\DD\backslash s^{-1}\Pi$ meets the boundary.

If $s$ is smooth embedding in $\RR^3$ then its saddle if and only if the  obtained surface has nonpositive Gauss curvature. 
An old conjecture of Samuel Shefel states that any saddle disc in a $\RR^3$ is $\CAT[0]$ with respect to its intrinsic metric, see \cite{shefel-3D}.

It is evident that any metric minimizing disc $s$ in a Euclidean space is saddle.
 

\parbf{Three-dimesional case.}
In general a saddle disc may not be globally metric minimizing.
An example is shown in the picture.
It is a saddle polyhedral disc made from 10 triangles
with a hexagon boundary forming a Y-shape marked with bold lines.


\begin{center}
\begin{lpic}[t(-0 mm),b(-0 mm),r(0 mm),l(0 mm)]{pics/not-sufficient-disc(1)}
%\lbl[lb]{12.5,11;$W_0$}}
\end{lpic}
\end{center}

By smoothing this example one can produce a smooth disc which is not metric minimizing, but \emph{strictly saddle} meaning the principle curvatures at each interior point have opposite signs. 
As the following proposition states, there are no local examples of that type.

\begin{thm}{Proposition}\label{prop:smooth}
Any smooth strictly saddle surface in $\RR^3$ is locally metric minimizing.
\end{thm}

Let $s\:\DD\to\RR^3$ be a smooth map.

Fix an array of vector fields $\bm{v}=(v_1,\dots,v_k)$ on $\DD$. 
Consider the energy functional 
\[E_{\bm{v}}s
\df
\sum_i\int\limits_\DD |v_is|^2\cdot d_x\area,\]
where $vs$ denotes the derivative of $s$ in the direction of the field $v$.
Set 
\[\Delta_{\bm{v}}s=\sum_iv_i(v_is).\]
It is convenient to think of the operator $s\mapsto \Delta_{\bm{v}}s$
as an analog of the Laplacian.

Note that 
\begin{enumerate}[(i)]
\item $E_{\bm{v}}$ is well defined for any Lipschitz map $s$.
\item $E_{\bm{v}}$ is convex, that is
\[E_{\bm{v}}[t\cdot s_1+(1-t)\cdot s_2]
\le 
t\cdot E_{\bm{v}} s_1+(1-t)\cdot E_{\bm{v}} s_2.\]
\item If $s$ is a smooth $E_{\bm{v}}$-minimizing map in the class of Lipschitz maps with given boundary data then $s$ is metric minimizing.
\item A smooth map $s\:\DD\to\RR^3$ is a $E_{\bm{v}}$-minimizing map among the class of Lipschitz maps with given boundary if and only if
\[\Delta_{\bm{v}}s=0.\]

\end{enumerate}

The discussion above reduces the proposition above to the following.

\begin{thm}{Claim}
Assume $s\:\DD\to \RR^3$ is a smooth strictly saddle surface. 
Then for any interior point $p\in\DD$ there is an array of 4 vector fields $\bm{v}=(v_1,v_2,v_3,v_4)$ such that the equation \[\Delta_{\bm{v}}s=0\eqlbl{eq:laplasian}\]
holds in an open neighborhood of $p$.
\end{thm}

\parit{Proof.}
Denote 
by $\kappa_1,\kappa_2$ the principal curvatures,
and by $e_1,e_2$ the corresponding unit principal vectors. 
Further, denote by by $a_1,a_2$ a pair of asymptotic vectors; we can assume that $a_1,a_2$ form coordinate vector fields in a neighborhood of $p$.


Set $v_1=\tfrac 1{\sqrt{|\kappa_1|}}\cdot e_1$ and $v_2=\tfrac 1{\sqrt{|\kappa_2|}}\cdot e_2$. 
It remains to show that one can choose smooth functions  $\lambda_1$ and $\lambda_2$ 
so that \ref{eq:laplasian}
holds in a neighborhood of $x$ for $v_3=\lambda_1\cdot a_1$ and $v_4=\lambda_1\cdot a_1$.

Note that the sum $v_1(v_1s)+v_2(v_2s)$ has vanishing normal part.
That is \[v_1(v_1s)+v_2(v_2s)\] is a tangent vector to the surface.

Since the $a_i$ are asymptotic,
the vectors $a_1(a_1s)$ and $a_2(a_2s)$ have vanishing normal part.
Therefore, for any choice of $\lambda_i$,
the following two vectors are also tangent
\begin{align*}
v_3(v_3s)&=\lambda_1^2\cdot a_1(a_1s)+\tfrac12\cdot a_1\lambda_1^2\cdot a_1s
\\
v_4(v_4s)&=\lambda_2^2\cdot a_2(a_2s)+\tfrac12\cdot a_2\lambda_2^2\cdot a_2s.
\end{align*}

Set $w=(\lambda_1^2,\lambda_2^2)$.
Note that the system \ref{eq:laplasian} can be rewritten as 
\[\left(\begin{smallmatrix}
   1&0\\0&0
  \end{smallmatrix}\right)
w_x
+
\left(\begin{smallmatrix}
   0&0\\0&1
  \end{smallmatrix}\right)
w_y=h(x,y,w),\]
where $h\:\RR^3\to\RR^2$ is a smooth function.

Change coordinates by setting $x=t+z$ and $y=t-z$.
Then the system takes the form 
\[w_t+\left(\begin{smallmatrix}
   1&0\\0&-1
  \end{smallmatrix}\right)
w_z=h(t+z,t-z,w),\]
which is a semilinear hyperbolic system.
According to \cite[Theorem 3.6]{bressan}, it can be solved locally for smooth initial data at $t=0$.

It remains to choose $v_3$ and $v_4$ for solution so that $\lambda_1, \lambda_2>0$ in a small neighborhood of $p$.
\qeds

\parbf{Four-dimensional case.}
Except for the constructing an energy as we did above,
we do not see any way to show that a given smooth surface is metric minimizing.
Locally, the appropriate energy functional can be described by three functions defined on the disc.
These three functions are subject to certain differential equations.
Straightforward computations show that on generic smooth saddle surfaces in $\RR^4$ 
there is no solution even locally.

By that reason we expect that generic smooth saddle surfaces in $\RR^4$ are not locally metric minimizing. 
That is, arbitrary small neighborhoods of any point admit deformations which shrink 
the intrinsic metric and keep the boundary fixed.
On the other hand we do not have an examples of a saddle surface for which this condition would hold at single point.


\begin{thebibliography}{52}

\bibitem{A}
\begin{otherlanguage}{russian}
Александров, А. Д. 
\textit{Линейчатые поверхности в метрических пространствах.}
Вестник ЛГУ 2 (1957): 15---44.
\end{otherlanguage}
%Alexandrov, A. D. ``Ruled  surfaces  in  metric  spaces,'' Vestnik Leningrad. Univ., 12:5-26, 1957 (Russian).

\bibitem{akp}
Alexander, S.; Kapovitch, V. and Petrunin, A.,
\textit{Invitation to Alexandrov geometry: CAT [0] spaces.}
\texttt{arXiv:1701.03483v2 [math.DG]}.

\bibitem{bishop}
Bishop, Richard L.
\textit{The intrinsic geometry of a Jordan domain.}
Int. Electron. J. Geom. 1 (2008), no. 2, 33--39. 

\bibitem{bressan} Bressan, A.
\textit{Hyperbolic systems of conservation laws.
The one-dimensional Cauchy problem.}
Oxford Lecture Series in Mathematics and its Applications, 20, 2000.

\bibitem{BBI}Burago, D., Burago, Y., Ivanov, S.
\textit{A course in metric geometry.}
Graduate Studies in Mathematics, 33, 2001.

%\bibitem{GS} Gromov, Mikhail, and Richard Schoen. "Harmonic maps into singular spaces and p-adic superrigidity for lattices in groups of rank one." Publications Mathématiques de l'IHÉS 76.1 (1992): 165-246.

%\bibitem{H} Hamilton, R. S. ``Harmonic Maps of Manifolds with Boundary,'' Lecture Notes in Mathematics, Springer, 1975, ISBN 978-3-540-37530-2.

%\bibitem{HKST} Heinonen, J.;  Koskela, P.;  Shanmugalinga, N.; Tyson, J. ``Sobolev spaces on metric measure spaces,''
%volume 27 of New Mathematical Monographs. Cambridge University Press, Cambridge, 2015.

%\bibitem{jost} Jost, J.
%Univalency of harmonic mappings between surfaces.
%J. Reine Angew. Math. 324 (1981), 141--153. 

\bibitem{KF}
\begin{otherlanguage}{russian}
Колмогоров, А. Н.;
Фомин, С. В.,
\textit{Элементы теории функций и функционального анализа.}
Издание седьмое, 2004.
\end{otherlanguage}

%\bibitem{KS}Korevaar, N. J.; Schoen, R. M. ``Sobolev spaces and harmonic maps for metric space targets,'' Comm. Anal. Geom., 1(3-4):561-659, 1993.

%\bibitem{LSW} Lytchak, A.; Stadler, S.; Wenger, S.  ``On conformal changes of CAT(0) spaces'', in preparation.

%\bibitem{LW}Lytchak, A.; Wenger, S. ``Area minimizing discs in metric spaces,'' preprint arXiv:1502.06571, 2015.

%\bibitem{LW2}Lytchak, A.; Wenger, S. ``Energy and area minimizers in metric spaces,'' preprint  arXiv:1507.02670, 2015.

\bibitem{LW3}
Lytchak, A.; Wenger, S.
\textit{Intrinsic structure of minimal discs in metric spaces.} 
\texttt{arXiv:1602.06755 [math.DG]}

%\bibitem{LW4}Lytchak, A.; Wenger, S. ``Regularity of harmonic discs in spaces with quadratic isoperimetric inequality  ,'' preprint  arXiv:1512.01060, 2016.

\bibitem{moore}
Moore, R. L.,
\textit{Concerning upper semi-continuous collections of continua.}
Trans. Amer. Math. Soc. 27 no. 4 (1925) pp. 416--428.

\bibitem{petrunin-metric-min} Petrunin, A.
\textit{Metric minimizing surfaces.}
Electron. Res. Announc. Amer. Math. Soc. 5 (1999), 47--54 

\bibitem{petrunin-intrinisic} Petrunin, A.
\textit{Intrinsic isometries in Euclidean space.}
St. Petersburg Math. J. 22 (2011), no. 5, 803--812 

\bibitem{petrunin-stadler} Petrunin, A., Stadler, S.
\textit{Monotonicity of saddle maps.} 
\texttt{arXiv:1707.02367 [math.DG]}.

%\bibitem{petrunin-orthodox} Petrunin, A. 
%``Exercises in Orthodox Geometry''
%{\tt arXiv:0906.0290 [math.HO]}

%\bibitem{R}Reshetnyak, Yu. G. ``Sobolev classes of functions with values in a metric space,'' II, Sibirsk. Mat. Zh. 45 (2004), no. 4, 855-870. MR 2091651 (2005e:46055)

%\bibitem{Se} Serbinowski,  T. ``Boundary regularity of harmonic maps to nonpositively curved metric spaces,''
%Comm. Anal. Geom. , 2(1):139-153, 1994.

%\bibitem{S}Schoen, R. ``Analytic Aspects of The Harmonic Map Problem'' Chapter IX in  
%``Lectures on harmonic maps'' by Schoen, R.; Yau, S. T.,  
%Conference Proceedings and Lecture Notes in Geometry and Topology, II. International Press, Cambridge, MA, 1997. vi+394 pp. ISBN: 1-57146-002-0

\bibitem{shefel-2D} 
\begin{otherlanguage}{russian}
Шефель, С. З.,
\textit{О седловых поверхностях ограниченной спрямляемой кривой.}
Доклады АН СССР, 162 (1965) №2, 
294---296.
\end{otherlanguage}
%\v{S}efel', S.,
%\textit{On saddle surfaces bounded by a rectifiable curve,} 
%Dokl. Akad. Nauk SSSR 
%162 
%(1965), 
%294--296.

\bibitem{shefel-3D} 
\begin{otherlanguage}{russian}
Шефель, С. З., 
\textit{О внутренней геометрии седловых поверхностей.}
Сибирский математический журнал, 5 (1964), 1382---1396
\end{otherlanguage}
%\v{S}efel', S., 
%\textit{On the intrinsic geometry of saddle surfaces,} Sibirsk. Mat. \v{Z}. 
%5 
%(1964), 
%1382--1396

%\bibitem{schoen-yau} Schoen, Richard; Yau, Shing Tung
%On univalent harmonic maps between surfaces.
%Invent. Math. 44 (1978), no. 3, 265--278. 

%\bibitem{St} Stadler, S. ``Harmonic discs in CAT(0) spaces'', in preparation.

%\bibitem{W1}Whyburn, G. T., ``On sequences and limiting sets,'' Fund. Math. vol. 25 (1935) pp. 408-426.

%\bibitem{W2}Whyburn, G. T., ``Analytic topology,'' Amer. Math. Soc. Colloquium Publications, vol. 28, 1942.

%\bibitem{Wi}Wilder, R. L., ``Topology of Manifolds,'' American Mathematical Society Colloquium Publications, vol. 32. American Mathematical
%Society, New York, N. Y., 1949.
\end{thebibliography}

\Addresses

\end{document}